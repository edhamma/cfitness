%% Generated by Sphinx.
\def\sphinxdocclass{book}
\IfFileExists{luatex85.sty}
 {\RequirePackage{luatex85}}
 {\ifdefined\luatexversion\ifnum\luatexversion>84\relax
  \PackageError{sphinx}
  {** With this LuaTeX (\the\luatexversion),Sphinx requires luatex85.sty **}
  {** Add the LaTeX package luatex85 to your TeX installation, and try again **}
  \endinput\fi\fi}
\documentclass[a5paper,10pt,english]{book}
\ifdefined\pdfpxdimen
   \let\sphinxpxdimen\pdfpxdimen\else\newdimen\sphinxpxdimen
\fi \sphinxpxdimen=.75bp\relax
\ifdefined\pdfimageresolution
    \pdfimageresolution= \numexpr \dimexpr1in\relax/\sphinxpxdimen\relax
\fi
%% let collapsible pdf bookmarks panel have high depth per default
\PassOptionsToPackage{bookmarksdepth=5}{hyperref}
%% turn off hyperref patch of \index as sphinx.xdy xindy module takes care of
%% suitable \hyperpage mark-up, working around hyperref-xindy incompatibility
\PassOptionsToPackage{hyperindex=false}{hyperref}
%% memoir class requires extra handling
\makeatletter\@ifclassloaded{memoir}
{\ifdefined\memhyperindexfalse\memhyperindexfalse\fi}{}\makeatother

\PassOptionsToPackage{booktabs}{sphinx}
\PassOptionsToPackage{colorrows}{sphinx}

\PassOptionsToPackage{warn}{textcomp}

\catcode`^^^^00a0\active\protected\def^^^^00a0{\leavevmode\nobreak\ }
\usepackage{cmap}
\usepackage{fontspec}
\defaultfontfeatures[\rmfamily,\sffamily,\ttfamily]{}
\usepackage{amsmath,amssymb,amstext}
\usepackage{polyglossia}
\setmainlanguage{english}


\usepackage{fontspec}\setmainfont{TeX Gyre Pagella}



\usepackage{sphinx}
\sphinxsetup{
        HeaderFamily=\bfseries,
        TitleColor={rgb}{0,0,0},
        InnerLinkColor={rgb}{0,0,0},
        hmargin={1.5cm,2cm},
        vmargin={2cm,2cm},
    }
\fvset{fontsize=auto}
\usepackage{geometry}


% Include hyperref last.
\usepackage{hyperref}
% Fix anchor placement for figures with captions.
\usepackage{hypcap}% it must be loaded after hyperref.
% Set up styles of URL: it should be placed after hyperref.
\urlstyle{same}


\usepackage{sphinxmessages}



        \usepackage{emptypage}
        \usepackage{titling}
        \makeatletter
        \fancypagestyle{normal}{
          \fancyhf{}
          \fancyfoot[LE,RO]{\thepage}
          \fancyfoot[RE,LO]{}
          \fancyhead[RE]{\releasename}
          \fancyhead[LE]{\@title}
          \fancyhead[RO]{\emph{\leftmark}}
          \renewcommand{\headrulewidth}{0.4pt}
          \renewcommand{\footrulewidth}{0pt}
        }
        \fancypagestyle{plain}{
          \renewcommand{\footrulewidth}{0pt}
          \fancyhead{}
          \renewcommand{\headrulewidth}{0pt}
        }
        \makeatother
        \renewcommand{\chaptermark}[1]{\markboth{#1}{}}
        \usepackage[numbered]{bookmark}
    

\title{Contemplative Fitness}
\date{Nov 13, 2024}
\release{e175053}
\author{Kenneth Folk}
\newcommand{\sphinxlogo}{\vbox{}}
\renewcommand{\releasename}{e175053}
\makeindex
\begin{document}

\pagestyle{empty}

\makeatletter%
\hypersetup{pdfauthor={\@author}, pdftitle={\@title}}%
\makeatother%
\begin{titlepage}%
    \vspace*{\baselineskip}
    \vfill
    \hbox{%
        \hspace*{0.15\textwidth}%
        \rule{1pt}{.95\textheight}
        \hspace*{0.05\textwidth}%
        \parbox[b]{0.8\textwidth}{
            \vbox to.95\textheight{%
                \vspace{.05\textheight}
                {
                    \noindent\Huge\bfseries Contemplative
                    \\[0.5\baselineskip]
                    Fitness
                }
                \\[4\baselineskip]
                {\Large\emph{Kenneth Folk}}\par
                \vfill % space{0.3\textheight}
                Other formats (PDF, HTML, ePub, …) available from \href{https://github.com/edhamma/cfitness}{github.com/edhamma/cfitness}.
                \\[\baselineskip]
                {\noindent Revision \releasename, built \today.}
                \\[\baselineskip]
                \emph{Kenneth Folk © 2013, All Rights Reserved.}
            }% end of vbox
        }% end of parbox
    }% end of hbox
    \vfill
\end{titlepage}

\pagestyle{plain}
\sphinxtableofcontents
\pagestyle{normal}
\phantomsection\label{\detokenize{index::doc}}


\frontmatter
\bgroup
\def\thesection{\arabic{section} }

\sphinxstepscope


\chapter{Introduction}
\label{\detokenize{front-intro:introduction}}\label{\detokenize{front-intro::doc}}
\setcounter{section}{0}

\sphinxAtStartPar
There is a kind of human development that has gone largely unnoticed in
the West. Even as mindfulness meditation makes its way into hospitals,
schools, and outpatient treatment facilities, our culture doesn’t yet
have an over\sphinxhyphen{}arching concept of how meditation is relevant to our lives.

\sphinxAtStartPar
Meditation is much more than stress reduction. Meditation changes your
brain. Do it enough and it will change your life.

\sphinxAtStartPar
In this book, I hope to present a conceptual framework in which to place
meditation and related contemplative practices, and to show how this
particular branch of human development is as essential to a human life
well\sphinxhyphen{}lived as psychological health, emotional maturity, or physical
fitness. In fact, the various kinds of mental and physical fitness work
together such that the whole is greater than the sum of the parts; one
step forward on any of the lines of human development makes it easier to
access any of the others.

\sphinxAtStartPar
Just as we can speak of physical fitness and mental fitness, we can
identify a branch of human development that we might call \sphinxstyleemphasis{contemplative
fitness}. Contemplative fitness has to do with the kind of growth that
comes from meditation and related contemplative practices. Its ultimate
manifestation is a kind of persistent well\sphinxhyphen{}being that is independent of
external circumstances. At its essence, contemplative fitness is the art
of being OK. And from the platform of being OK, the stage is set for the
very best of humanity to emerge. When you are OK, an enormous amount of
energy is freed up to find out what it means to be truly human. When you
don’t have to work so hard to protect yourself, you have, perhaps for
the first time, the luxury of considering the needs of others. It is
from this stable place of equanimity and self\sphinxhyphen{}acceptance that we can
learn to access levels of sensitivity, creativity, spontaneity, and
empathy that we didn’t know existed. One aspect of contemplative fitness
has to do with what has often been called spiritual awakening or
enlightenment. This phenomenon is real, and it happens to real people in
our own time; it happened to me, and the first part of this book
describes that process. But rather than thinking of this awakening as a
panacea, a magical wand to wave away our difficulties, we can take a
more nuanced view, a more realistic and balanced understanding that
takes into account what we now know about psychology, physics, and
neurobiology. In order to embrace the benefits of contemplative fitness,
we don’t have to believe that at some point, if we meditate enough, we
will behave impeccably, glow in the dark, or suddenly have access to the
90\% of our brain that we imagine is now lying dormant.

\sphinxAtStartPar
Contemplative fitness, as I teach it, does not require the adoption of
any philosophical or religious beliefs. I will not make claims about the
structure of the universe or the ultimate nature of reality. I don’t
teach how to be right, smart, perfect, sanitized, or holy. I don’t teach
super\sphinxhyphen{}powers or extra\sphinxhyphen{}sensory perception. I don’t teach religion,
guru\sphinxhyphen{}worship, dogma, or doctrine. I don’t teach people to uncritically
accept what I say. I simply teach practical, hands\sphinxhyphen{}on techniques that,
when practiced diligently, can be utterly transformative to a human
life. To the extent that I offer concepts and ideas, they are intended
not as doctrine, but as conceptual frameworks within which to understand
your own experience. The second part of this book, Theory, will help you
orient yourself as you practice.

\sphinxAtStartPar
Contemplative fitness is spiritual enlightenment for the 21st century.
And although no less valuable than the mythologized notions of the past,
the modern version does not require us to suspend common sense. Rather,
it requires us to do some work; progress comes as a result of effective
training. The third part of the book, Method, will give you the tools to
find this out for yourself.


\section{The physical fitness revolution}
\label{\detokenize{front-intro:the-physical-fitness-revolution}}
\sphinxAtStartPar
When I was a boy growing up in Southern California in the 1960s, the
concept of physical fitness, as we know it today, did not yet exist.
There had always been sports and athletics, of course, but those
pursuits were for a special minority; most of us were not athletes and
did not consider exercise to be particularly relevant to our lives
except in our capacity as spectators.

\sphinxAtStartPar
And yet, change was in the air. A few visionaries had taken it upon
themselves to bring exercise to the people. And it was from these early
efforts that the modern concept of physical fitness emerged and began to
penetrate the consciousness of everyday folks. One early pioneer was
Jack Lalanne, who would eventually come to be seen as the “godfather of
fitness.” I remember Jack Lalanne well. He had his own television show,
was impressively muscular, charmingly energetic, and always dressed in
his trademark black or navy blue jumpsuit. And he always seemed to be
doing jumping jacks. The popular image of physical fitness in the 60s
was Jack Lalanne doing jumping jacks. At a time when there were only a
handful of television stations to choose from, Jack Lalanne was
impossible to miss, and he soon became a fixture in every living room in
America. Before long, we were all doing jumping jacks in front of the
TV.

\sphinxAtStartPar
We can look back now and chuckle at how unsophisticated we were compared
to our current understanding of physical fitness. There has been an
explosion of sophistication in both the theory and application of
exercise science. It is now possible to train with great precision to
achieve virtually any kind of physique, always taking into account your
own natural strengths and limitations. Whether you want to be a ballet
dancer or a power lifter, a tennis player or a marathoner, you can find
a fitness trainer who can help you do it. Most importantly, everyday
folks who have no intention of ever playing a competitive sport
understand the value of physical fitness and often make it a priority in
their lives. Physical fitness is real, the concept is well established,
and the benefits are well accepted, not just for athletes, but for
everyone. And all of this has happened within a single human lifetime.

\sphinxAtStartPar
We are ready for a parallel revolution in contemplative fitness. Years
from now, we will look back and chuckle at how unsophisticated we were
in our understanding of meditation and its benefits way back in the
opening years of the 21st century. Even for those of us who accept that
such a thing as spiritual awakening is possible, the field tends to be
shrouded in religion, superstition, hero\sphinxhyphen{}worship, and unrealistic
expectations. We look for inspiration to the Buddha, the Zen Patriarchs,
Ramana Maharshi, or some saintly figure from our own time, rarely
allowing ourselves to consider that in order for awakening to become
real, we must make the transition from spectators to participants. The
physical fitness revolution exploded when we stopped watching Jack
Lalanne on TV and started doing jumping jacks of our own. The
contemplative fitness revolution will begin when we stop looking to our
spiritual heroes and start meditating. In this book, I will tell my own
story first. It is the story of a depressed and addicted young man who
found his way in the world through a single\sphinxhyphen{}minded obsession with
meditation. Next, I will present a theory of contemplative development
based on my formal Buddhist training as well as my thirty\sphinxhyphen{}plus years of
dedicated practice and my twenty\sphinxhyphen{}plus years of teaching meditation and
awakening. Finally, I will offer a detailed method that has been
successful for dozens of ordinary people as I have guided them through
their own process of discovery. My hope is that twenty years from now,
contemplative fitness will be as much a part of mainstream culture as
physical fitness is today. I believe that a great deal of good will come
from such a revolution.

\egroup
\mainmatter
% promote sections for the main text
% (unlike in frontmatter and appendix)
\let\subsubsection\subsection
\let\subsection\section
\let\section\chapter
\let\chapter\part

\sphinxstepscope


\chapter{Book One: Kenneth’s Story}
\label{\detokenize{main-1:book-one-kenneth-s-story}}\label{\detokenize{main-1::doc}}

\section{The Setup}
\label{\detokenize{main-1:the-setup}}
\sphinxAtStartPar
In 1982, I was a suicidally depressed cocaine addict. A 23\sphinxhyphen{}year old
musician in Los Angeles, I had a lot of free time to sit around being
depressed and wondering how my life had gone so terribly wrong. I was
trying to kick my cocaine habit, and failing. One night, alone at home,
having exhausted all the cocaine in the house and spiraling into
despair, I took four hits of LSD. And while I’m neither advocating drugs
nor taking a moral stance against them, this is what happened.

\sphinxAtStartPar
I put the LSD in my mouth and turned on the television. I watched part
of the \sphinxstyleemphasis{Shogun} miniseries about a 17th Century English ship pilot who
was shipwrecked in Japan and adopted Samurai culture. There is a scene
in which John Blackthorne, the protagonist, who has now become a
Samurai, decides to commit \sphinxstyleemphasis{seppuku}, Japanese ritual suicide by
disembowelment. Just as Blackthorne is tensing his muscles to plunge his
short sword into his own abdomen, another Samurai reaches out and grabs
Blackthorne’s hand, preventing his suicide. Watching this scene on
television, I wondered about the changes that might take place in the
mind of someone who had completely accepted death in a moment and yet
didn’t die. I was fascinated by the question, and this theme of death
and rebirth would set the tone for the evening.

\sphinxAtStartPar
I went into my bedroom, closed the door, and lay down on the bed, face
up. I had nothing left to do but reflect upon the unsatisfactoriness of
my own life. Still pondering the question of death, I remembered another
movie I had seen in which an old Native American Indian chief climbs a
hill and lies down on a funeral pyre. The pyre is not lit; it’s just a
bunch of sticks. The old man lies down on the pyre and says to himself.
“Today… is a good day to die.”

\sphinxAtStartPar
Tired, defeated, and yet inspired by the possibility of surcease, I said
to myself, “Yes. Today \sphinxstyleemphasis{is} a good day to die.” In that moment, my mind
felt so powerful, so focused… I was absolutely convinced that I could
will myself to death.

\sphinxAtStartPar
Flat on my back, I began to meditate, using a technique I had learned
from my older brother a couple of years before. It was a simple
concentration exercise, nothing more than looking at the backs of my
eyelids with eyes closed, and falling into the blackness there. In the
past, I had practiced it occasionally in an effort to relax, and to have
an interesting experience of an altered state of consciousness. Now, I
was meditating with a purpose. And as I was thus engaged, earnestly
attempting to will myself to death, an odd thing happened; it occurred
to me that if I \sphinxstyleemphasis{did} die, I would be opening myself up to whatever
negative forces were out there in the ether. I had a visceral fear that
there was some kind of malevolent force, some kind of evil that would
wash over me and take control if I let down my guard. I think I also
understood in that moment that I had never let down my guard before. So
here I was, 23 years old, and somehow I had managed throughout my entire
life to maintain a wall, to keep something, who knows what, from
entering my consciousness and taking it over. I could feel this
unspeakable evil clamoring outside the gates, trying to get in. I was
both terrified and bemused.

\sphinxAtStartPar
I wondered if this was what Christians meant by “the Devil,” the very
personification of evil. Interestingly, I wasn’t the least bit
religious. I thought religion was foolish. I didn’t believe in God. I
didn’t believe in the Devil. But somehow here I was, thinking “the
Devil’s gonna get me.” Ridiculous, on hindsight, like something out of a
seventies\sphinxhyphen{}era comedy skit. At the time, though, it didn’t feel like a
joke. Far from it, in fact; I had never been so frightened. This fear
lasted for a few moments, and then I began to ponder a kind of equation
of good and evil: if indeed there were such a thing as the Devil, then
there must also be such a thing as a God, in which case, if I opened
myself up entirely, they would either cancel themselves out or God would
win. Somehow, this childlike idea of symmetry in the universe gave me
just the courage I needed to take the leap. So I did. I opened up
entirely and surrendered to death.

\sphinxAtStartPar
This absolute and unquestioned surrender to my own death… no, even more,
\sphinxstyleemphasis{commitment} to it… inspired by the movie scene I’d seen earlier of John
Blackthorne’s abortive suicide attempt, triggered a remarkable series of
events.

\sphinxAtStartPar
Immediately upon acceptance of my own death came the recognition that
the “malevolent forces” barely held at bay for so many years by my own
dogged unwillingness to admit them, were none other than my own fears. I
was protecting myself from \sphinxstyleemphasis{myself}. This recognition, so surprising and
stark, brought, all of itself, enormous relief. The insufferable burden
of a lifetime was seen as an illusion fueled by a misconception. Indeed,
the fears themselves were tolerable; it was the effort to avoid them
that I could not endure. With the shattering of the illusion, a burden
was lifted and the need to die was gone, but the event now had a
momentum of its own and continued to unfold even though all thoughts of
self\sphinxhyphen{}destruction had evaporated.

\sphinxAtStartPar
Next was a kind of instantaneous life review. A thousand images flowed
through my mind in a single moment, images of things I had done, both
“good” and “bad.” The theme was that actions have consequences; it was
immediately and intuitively obvious that actions motivated by good will
had led to positive results while actions motivated by ill will had led
to sorrow. This insight was matter\sphinxhyphen{}of\sphinxhyphen{}fact, with no implied judgment or
moralism; here is everything I’ve done, and here are the consequences of
each action. Here was my very own mechanistic, non\sphinxhyphen{}moralistic judgment
day.

\sphinxAtStartPar
The experience continued to unfold in stages. Next, I found myself being
drawn up into the sky through what appeared to be a long glass tube. I
was fascinated, riveted by this experience. Suddenly, there appeared a
flock of small, translucent, quasi\sphinxhyphen{}intelligent, possibly unfriendly
beings on the other side of the glass tube, trying to get my attention,
and keeping pace with me as I was sucked up toward the sky. I had the
impression that they wanted to get inside the tube, to go where I was
going, and that they were frustrated by being stuck outside. I was aware
that I had taken acid and was hallucinating, but the kind of cohesion
and consistency of these visions, this entirely new world created out of
thin air, was unlike anything I had experienced before, with or without
drugs. As I was floating up the glass tube alongside these roundish,
colorful beings, I remember thinking to myself that this must be some
kind of challenge or quest: “\sphinxstyleemphasis{I’ve got to find a way to communicate
with these things, but we don’t have a language in common. How can I
communicate with them?}” I felt that if I could find some common
ground with the creatures, we would be able to establish a basis for
communication. Well, it didn’t happen. My mind was a blank. If it was a
quest, I failed it. I soon outpaced the creatures and they disappeared.

\sphinxAtStartPar
I was being sucked up into the sky faster and faster now, and was able
to see that there was an end to the tube, and at the end of it was white
light… blinding, glorious, perfect light beyond imagining. I was moving
so fast now that almost immediately after first glimpsing the light, I
was pulled into it and merged with it. And this was far and away the
most ecstatic experience of my life so far. Because now I was one with
what felt like universal consciousness. It was an utterly mind\sphinxhyphen{}blowing
experience. I thought \sphinxstyleemphasis{this} must be what the Christian mystics meant
when they said “God”. But it wasn’t the personal God of a Michelangelo
painting. It wasn’t a man up in the sky who was like me, only big and
powerful; it was everything that was or had been or could ever be, and
it was self\sphinxhyphen{}aware. And in that moment of merging with what seemed to be
universal consciousness, it was as though I knew everything there was to
know; everything that needed to be known was known, and yet there was no
need to ask. This felt really good. Beyond good. Perfect, exquisite,
ecstatic, flawless… superlatives fail to capture it. I marveled to
myself, “\sphinxstyleemphasis{Everything up until now, my entire life, has been a dream.
Only now am I awake. Only this is real.}” And almost immediately I
realized that it was going to end. I was going to be kicked out of the
garden. Later, I wrote in my journal, “As I lay naked beneath God’s
crushing foot, I asked Him to throw me a bone: ‘\sphinxstyleemphasis{Nobody is going to
believe this. I’m going to need some proof. Give me something to take
back with me.’}” This experience of merger with something infinitely
larger than myself made everything else pale in comparison, and already
I could see that it would end and that I would have nothing to show for
it. There was a moment of profound grief. A moment later, I found myself
back in my room, lying on my bed facing upward, exhilarated, exhausted,
annihilated and reborn.

\sphinxAtStartPar
Now, as it happened, my cocaine addiction vanished in that moment; I
have not used cocaine since. There was no aversion, no negative feeling
about the drug. I just wasn’t interested anymore. My reaction to being
offered cocaine was similar to what I imagine might happen if someone
offered me a plate of cold, raw tofu: “No thanks, I don’t much like
cold, raw tofu.” There was no offense and no judgmentalism, only
disinterest. It later occurred to me that this might be the “bone” I had
asked for, the objective proof that something remarkable had happened.
As to who it was that granted me this boon of the bone, my thinking has
changed over the years. I no longer believe I was having a conversation
with an “Eternal Being,” or even that there is such a thing, although I
did believe that for a long time after the event. My current speculation
is that what happened that day was an internal event, a function of the
interplay between a brain, a psychoactive substance, meditation, and a
traumatic life situation. One way or the other, the experience was
deeply moving and may have saved my life in addition to setting me on a
new course.

\sphinxAtStartPar
The experience of union showed me a reality beyond my ordinary self, but
it was only a short\sphinxhyphen{}lived glimpse. I found myself on a quest to
understand what had happened to me and to “get it back.” With the
assumption that what I had glimpsed was somehow truer than my ordinary
life, I wanted to be able to access it again, and ultimately find a way
to feel like that all the time. I was now officially a seeker, but I
didn’t know how to seek. For many years following that experience, I
couldn’t escape the feeling that I was somehow “doing it wrong.” I
experienced myself as alien, but I remembered that it was possible to be
complete, and I was determined to feel that way again.

\sphinxAtStartPar
Although both meditation and drugs were involved in that first big
opening, my intuitive sense was that the way forward was through
meditation, not drugs; it seemed to me that while drugs might
temporarily open windows in the mind, a more systematic approach would
be required to keep them open. So, I began meditating each day while I
did some research.

\sphinxAtStartPar
I began by reading self\sphinxhyphen{}improvement books, a genre I had previously
regarded with contempt. I read the likes of Dr. Wayne Dyer about how to
realize your potential as a human being. That was a place to start, but
it wasn’t zoomed in enough on where I wanted to go. On a recommendation
from a friend, I bought a copy of the Ram Dass book \sphinxstyleemphasis{Be Here Now}. This
was getting closer. Ram Dass made vague but tantalizing references to
spiritual awakening, spinning interesting and implausible yarns about
his guru, who he considered to be a “fully realized being.” From there,
I began reading Buddhist books, getting ever closer to what I really
wanted, which was an instruction manual. I read Alan Watts on Zen wisdom
and then \sphinxstyleemphasis{The Three Pillars} of Zen by Philip Kapleau. I also read
Ouspensky’s book about Gurdjieff, and took a brief detour into New Age
books like \sphinxstyleemphasis{Seth Speaks} and Richard Bach’s \sphinxstyleemphasis{Illusions}. I found all
sorts of hints, a drop of wisdom, a dollop of childish nonsense, and a
large portion of snake oil, but no method. In 1989, I read Ken Wilber’s
\sphinxstyleemphasis{Spectrum of Consciousness}. Wilber was the first author I had found who
bridged the gap between a nebulous and impractical pursuit of spiritual
enlightenment and a more concrete understanding that could be approached
systematically and reconciled with common sense and science. By talking
about levels of mind that could be targeted by specific practices,
Wilber made spiritual awakening/enlightenment sound like a realistic
project. But he did not offer a method. I had been an almost daily
meditator for seven years, ever since my big opening in 1982; I was
willing to do the work if someone could give me the instructions. In
\sphinxstyleemphasis{Spectrum of Consciousness}, Wilber mentioned in passing that he was
offering a conceptual framework as opposed to a method, and that
resources abounded for those who sought a more hands\sphinxhyphen{}on approach. I was
frustrated. I had no idea what resources he was referring to. I
continued to practice without a teacher.

\sphinxAtStartPar
Fast forward to 1990, eight years after my first unitive experience. My
depression had returned. I made my living as a bass player in a dance
rock band. Sometimes I would find myself onstage in front of a hundred
people, on the verge of tears for no reason that I could name. I
couldn’t play music anymore. I quit the band in North Carolina, where I
had moved two years earlier to pursue my musical career, and moved back
to Southern California. I promised myself I would never again perform
music for money. All I wanted to do was meditate. My most cherished
fantasy involved checking myself into a cave in the Himalayas and living
as a monk for the rest of my life.


\section{Bill Hamilton}
\label{\detokenize{main-1:bill-hamilton}}
\sphinxAtStartPar
When I moved back to Southern California, I had my mail forwarded from
the post office in Chapel Hill. A few weeks after arriving in
California, a postcard arrived, forwarded from my old address. It was a
simple white postcard with some dot matrix computer\sphinxhyphen{}printed text on it,
advertising a series of audiocassette recordings of discussions between
the Dalai Lama and western scientists. It sounded intriguing, so I
decided to order the tapes, which would set me back about twenty bucks.
Reading the phone number on the postcard, I noticed that the area code
and prefix were from a town not more than a half hour’s drive from where
I was now living. Excited, I called the number and said I’d like to
order the tape set, and that I was nearby, just half an hour away. The
man on the other end of the line, whose name was Bill Hamilton, said he
was a meditation teacher and invited me for a visit.

\sphinxAtStartPar
My first impulse was to impress Bill with what I knew about Buddhism and
spirituality, because I was used to thinking of myself as a big deal;
I’d had this thing happen to me that most people hadn’t had, or at least
weren’t talking about. But within two minutes of meeting Bill, I
realized he wasn’t speculating; he knew far more about meditation and
awakening than I did. I stopped talking and started listening.

\sphinxAtStartPar
Bill was twenty\sphinxhyphen{}five years older than I. He was gawky and tall, about
6’1”. He had white hair, a Prince Valiant haircut, and a short white
beard. He was affable, humorous, and just slightly socially awkward,
with a tendency toward malapropisms. In spite of the occasional mis\sphinxhyphen{}used
word, though, Bill Hamilton was a masterful communicator. He was
eloquent, articulate, creative, and had a special way with concepts. And
he was the king of the one\sphinxhyphen{}liners. When I asked Bill what it felt like
to be enlightened, he said, “Suffering less. Noticing it more.” Bill was
a natural entrepreneur. He had founded the Dharma Seed Tape Library as a
volunteer at Insight Meditation Society (IMS) in 1983 but had since
moved on and was now subsisting solely on the proceeds of his own mail
order cassette tape business, Insight Recordings. He had a couple of
professional dubbing machines in his apartment, and did all of his own
promotion and bookkeeping on his computer. Bill liked to modify his own
computers; he had several, and they were always breaking down. He liked
to laugh about “computer follies,” which was his way of referring to all
the time he spent jury\sphinxhyphen{}rigging his machines. He also had a 35mm SLR
camera that he would drag out randomly to shoot pictures. Bill had been
married and divorced three times, and was now alone.

\sphinxAtStartPar
Bill became my mentor. I drove the 20 miles to his apartment every
Sunday afternoon for a personalized dharma talk, a hangout, and 45
minutes of formal sitting meditation. The first thing Bill taught me was
to use Mahasi Sayadaw’s (Footnote here with link to noting definition
and instructions.) mental noting technique instead of the Zen breath
counting exercise I’d learned from a book. And every Sunday evening,
when I left Bill’s house to drive home, I would be on cloud nine, full
of hope and optimism, and a deep calm that felt like the most precious
gift in the world, and the only thing worth pursuing in an otherwise
confusing and pointless existence. I did not understand why spending
time with this old man affected me so profoundly.

\sphinxAtStartPar
One of the things that struck me about Bill was his willingness to walk
his talk. The first day I met him, he needed to go to his storage unit
to get something out of it. Since Bill drove an old yellow Volkswagon
bug, he asked me to drive him there in my Honda wagon, which had more
cargo space. At the storage unit, rummaging around in boxes, Bill found
a black widow spider. I would’ve just killed it, but Bill left it alone.
When I asked him why, he told me that one of the five precepts of
Theravada Buddhism was to avoid killing. I was impressed by the fact
that he not only knew about these precepts, but actually followed them,
unwilling to kill so much as a bug. Inspired by Bill’s example, I too
adopted the precept to avoid killing “sentient beings,” and for years I
didn’t kill insects either. Incidentally, a few years later I was on
retreat at Bill’s Whidbey Island Retreat, which was his retreat center
(made available by a generous friend) in Washington State, consisting of
20 acres of pine forest and Bill’s tiny trailer, along with an extra
motor home for a yogi or two to stay in. I was the only student there at
the time. One day, I saw Bill smack a mosquito. I said, “I see you’re no
longer abstaining from killing insects.” Bill said, “Last time I was {[}on
meditation retreat{]} in Burma I felt like killing them. So I did.” Bill
had been following the non\sphinxhyphen{}killing precept for years. I interpreted this
not as backsliding, but as progress. Notwithstanding the beauty of a
life without killing, Bill had come to a place in his practice and his
life where he could question even his own dogma.

\sphinxAtStartPar
Compared to everything I’d heard and read previously, Bill’s model of
enlightenment was simple and clear. He told me about the four “Paths of
Enlightenment” of Theravada Buddhism, discrete developmental landmarks
that could be attained by systematically applying the vipassana
technique. {[}The first of the four paths is called stream entry, and is
discussed in greater detail in Chapter X as part of the method.{]}
Together, the four paths form a map of what can happen when a meditator
does vipassana practice. {[}I will be presenting my interpretation of
these stages in Chapter X, “Get Stream Entry”, as one tried and true
programs for developing contemplative fitness, a method I have seen work
for dozens of students.{]} If you read traditional descriptions of these
four paths, you will find references to future rebirths (and lack
thereof), saints, “fetters,” and “purification.” If, on the other hand,
you strip away the jargon, magical thinking, gratuitous mythology, and
hero\sphinxhyphen{}worship, while making healthy allowances for hyperbole and
hagiography, the four\sphinxhyphen{}path model can be interpreted as describing an
organic process of human development. (Editor’s note: Link to a section
describing my interpretation of the four path model explaining why I
reject the idea that I am “redefining” the model; my contention is that
there is no One Correct Way (orthodoxy) to interpret ancient teachings,
by which all other interpretations must be judged (and found lacking).)

\sphinxAtStartPar
In other words, the ancient Buddhists were \sphinxstyleemphasis{onto} something. Bill was
talking about something do\sphinxhyphen{}able, and he believed there were many people
living today who had these attainments, including \sphinxstyleemphasis{arahatship} (fourth
of the four paths), or “full enlightenment” according to Theravada
Buddhism. Bill gave me to understand through indirect speech that he
himself had attained at least the second of these four paths of
enlightenment. Finally, after eight years of reading fairy tales culled
from a Zen master’s fantasy, I was sitting across the table from a man
who asserted that there \sphinxstyleemphasis{was} something called enlightenment, and that
he had it. Or at least that he had some significant amount of it, and
was working towards getting more of it. (Purists, don’t despair at the
irony of \sphinxstyleemphasis{getting} enlightenment as though it were a side of bacon in
the butcher shop window. We’ll discuss the pros and cons of
“\sphinxhref{https://www.amazon.com/Cutting-Through-Spiritual-Materialism-Chogyam/dp/1570629579}{spiritual materialism}“
in a later section. For now, suffice to say that it was precisely the
clarity of language made possible by the acquisitive approach to
awakening that made it possible for me to jump in with both feet.)


\section{First Long Retreats}
\label{\detokenize{main-1:first-long-retreats}}
\sphinxAtStartPar
Within a few months of meeting Bill, he had convinced me to commit to a
three\sphinxhyphen{}month\sphinxhyphen{}long intensive meditation retreat at Insight Meditation
Society in Massachusetts. When he first suggested it, I balked; the
prospect of spending three months in silence, meditating all day long,
every day, was daunting. But I soon warmed up to the idea. The
guidelines for the three\sphinxhyphen{}month retreat called for several weeks of
prerequisite meditation retreats before attending such a long program.
But Bill had connections at IMS, having spent most of the 80s there as a
“long term yogi,” living in the unfinished basement of the facility,
attending all the retreats and recording the dharma talks. He pulled
some strings and got me signed up. I spent the fall of 1991 on retreat.
I kept a journal of the entire ordeal. When I returned home around
Christmas time, I sat down with Bill to tell him about it, reading
directly from my notes. I read for two hours straight, and literally put
Bill to sleep at one point. I pretended not to notice that he was
snoring and kept reading. After listening to my report, Bill told me
that I had gone through ten of the sixteen stages leading up to stream
entry or “first path,” the first level of enlightenment according to the
Theravada map. I had not attained stream entry, but I was close.
Remarkably, Bill was able to extract some useful information from my
long\sphinxhyphen{}winded story. It took me years to figure out that my vipassana
teachers didn’t care what I \sphinxstyleemphasis{thought} about my meditation. No matter how
important it seemed to me, they couldn’t glean much information from my
opinions and psychological or philosophical commentary. They wanted to
hear about what \sphinxstyleemphasis{actually happened}, in clear, simple terms. The ability
to distinguish experience from thoughts about experience is key to both
effective practice and effective reporting. By comparing
phenomenological descriptions of my experience to the developmental map
they carried in their heads, they could tentatively place me on that map
and give me targeted advice on how to develop further.

\sphinxAtStartPar
Based on my report, Bill was able to neatly line up my experiences with
the Progress of Insight map. (Footnote to P of I essay.) He showed me,
point by point, were I was and where I had been. About halfway through
the retreat, for example, I’d fallen into a notoriously difficult
stretch of territory, and languished in it for the remainder of my time
in Massachusetts, meditating less, sleeping more, ruminating and
worrying, journaling, shuffling about the retreat center in a funk, and
generally wasting time. Bill explained that all of this was predictable,
and that if I’d spent more time meditating and less time thinking and
writing during the second half of the retreat, I might well have moved
through this difficult stage and on to next, which was, by the way, a
distinctly more agreeable state. I might even have attained stream
entry! I pointed out that this would have been valuable information to
have in real time. “Why didn’t my interview teachers tell me what \sphinxstyleemphasis{you}
just told me?”

\sphinxAtStartPar
Bill grinned. “IMS is a mushroom factory.”

\sphinxAtStartPar
I didn’t catch the reference, so he explained: ”Keep ‘em in the dark and
feed ‘em shit.”

\sphinxAtStartPar
How to explain the impact of one comment on my entire life? The IMS
teachers had treated us, the students, like “mushrooms.” I was stunned,
later enraged. I found it appalling that teachers would withhold such
valuable information. Surely, if I had known that my discouragement,
confusion, and lack of motivation were normal, typical, temporary, and
the entirely predictable consequence of a particular phase of
developmental that was first mapped over 2,000 years ago, I would have
practiced differently and had a more successful retreat.

\sphinxAtStartPar
My commitment to full disclosure about states and stages was born with
Bill Hamilton’s “mushroom” comment. Much of my teaching since 1991 has
been a reaction to what I came to think of as the “mushroom culture” of
mainstream Western Buddhism. The antidote to the mushroom culture was
the simple dissemination of information. I railed (and still rail)
against the presumption, patriarchy, and authoritarianism that leads a
few teachers to withhold information from their students. This theme
later became a movement, when my friend Daniel Ingram, having heard my
story and later having experienced the mushroom culture for himself,
wrote about it in his 2003 book, \sphinxstyleemphasis{Mastering the Core Teachings of the
Buddha}.

\sphinxAtStartPar
Bill suggested that I go to Asia, check myself into a monastery, and get
stream entry. So that’s what I decided to do. At the time, I only cared
about meditation. The rest of my life didn’t matter. Bill told me:
“Everything you do in order to make your next retreat possible is part
of your practice.” I found this vastly empowering; now it was possible
to view all aspects of my life as supporting my practice, rather than
getting in the way of it. I had a job delivering pizzas at Domino’s
Pizza, which was demoralizing due to the low wage. Again, Bill came to
the rescue: “Just figure out how many pizzas you need to hustle to buy a
ticket to Burma. Then, get busy.” I got busy. Although I didn’t earn
much at the pizza store, my goal was concrete, and I could see progress
each day as my piggy bank filled up. I sold my car and bought a one\sphinxhyphen{}way
ticket to Malaysia, understanding that I was going to meditate at a
Burmese\sphinxhyphen{}style monastery in Penang while applying for a visa to continue
on to Burma. I didn’t know when or whether I would return home. In fact,
returning home was the furthest thing from my mind. I planned to get
enlightened. I was steeped in the four paths model of enlightenment that
I’d learned from Bill, and I wanted to get not just first, but also
second path before returning home from Asia, however long that might
take. {[}I no longer view the four paths model and the progress of insight
map as the only way to model contemplative development, but it is a
useful tool (lens) for diagnostics and teaching that consistently leads
to results.{]}

\sphinxAtStartPar
I remember saying to Sayadaw U Rajinda, the Burmese monk who was both my
interview teacher and abbot of Malaysian Buddhist Meditation Centre in
Penang, “I’m going to stay in Asia until I get 2nd path.” U Rajinda
smiled approvingly, and in his deep, resonant voice, said, “Good plan.”
This was powerful validation; now, both Bill Hamilton and Sayadaw U
Rajinda were on my team, and both of them took this as seriously as I
did. Enlightenment was real and doable. I stayed in Malaysia for six
months. U Rajinda was my teacher throughout that time. I saw him later
that year in Burma when he came for a visit, and again in Malaysia after
that. He and I formed a bond. He once drew a picture of me on a scrap of
paper and gave it to me as a gift. It was an image of a shaven\sphinxhyphen{}headed
meditator sitting cross\sphinxhyphen{}legged in meditation, with the caption “Mr.
Kenneth” printed underneath in English.

\sphinxAtStartPar
Being on retreat in the Mahasi Sayadaw tradition is intense, immersive,
and often grueling. There’s very little to do other than meditate. You
go to sleep at 9:00pm, wake up at 3:00am and meditate, alternating one
hour of sitting with one hour of walking meditation. Sometimes I would
wake up at 2:00am. If you whittle it down to four hours of sleep by
going to bed at 10:00pm and waking up at 2:00am, you earn a smile from
the monks and a sense of macho satisfaction. After breakfast, there is a
work period. They give you a piece of a plant, something like a palm
frond, to use as a broom, and you might spend 10 minutes sweeping the
floor of the meditation hall. An ordinary broom would be more effective,
but there is apparently some sort of ceremonial significance to the
frond, and after all, it’s not as though we were pressed for time. After
work period, it’s back to meditation, all day long, with a break for
lunch at 10:00am. Lunch is the last meal of the day; monks are not
allowed to eat after noon, and there were no special provisions for
those of us practicing as lay people to sneak in an extra meal.

\sphinxAtStartPar
A little bit of talking is allowed, especially if it is about
meditation, but anything beyond five or ten minutes a day is met with
disapproval. As such, most of the adventures occur internally. Looking
at the workings of your own mind is rarely dull, and you encounter the
whole range of experience, from “this is the most wonderful, amazing
experience possible for a human and I never want to leave retreat” to “I
hate everything about this and I’ve got to get out of this hell hole
immediately.” The deep compulsion to let the process run its course, to
find out where it was going, was so strong that I stayed for an entire
year the first time, and for months at a time in two subsequent trips to
Southeast Asian monasteries.

\sphinxAtStartPar
Within about two months of starting my retreat in Malaysia, meditation
had become uneventful. All of the big, exciting, “wow” things of my
earlier practice had passed and I was just sitting, quiet and
comfortable. This is the stage called “insight knowledge of equanimity”
on the Progress of Insight map, the stage just beyond where I had gotten
on my IMS retreat in Massachusetts. One day, sitting after lunch,
something changed. I fell so deeply into meditation, it was almost as
though I went to sleep, or lost consciousness for a moment. And then,
suddenly, I perked up and said to myself, “Was that it? I think that was
it.”

\sphinxAtStartPar
According to the Mahasi interpretation, stream entry and subsequent path
moments are signaled by an event called a “cessation.” {[}More on this in
Chapter X.{]} A cessation, by this interpretation, is a blip out, a loss
of consciousness, typically for just a brief moment, although in some
cases it might last longer. Now you’re here, now you’re not, now you’re
back, with no sense of the passage of time and no memory of what
happened in the interim. The first time this occurs, it signals stream
entry. I instantly recognized my experience that afternoon as stream
entry, based on what I had heard and read about the phenomenon. Bill
Hamilton had characterized stream entry as a great anticlimax compared
to experiences that often precede it, like my first mind\sphinxhyphen{}shattering
opening in 1982. Although such powerful unitive experiences are often
assumed to be enlightenment by those who experience them, they are, at
least according to the Theravada Buddhist tradition, preliminary stages;
Bill said that the initial unitive opening is to stream entry as the
germination of a seed is to the blossom that eventually grows from it.

\sphinxAtStartPar
After attaining stream entry in Malaysia, I got up from the cushion and
walked around the monastery laughing for a day or two. I felt free. Life
was good. I suddenly had access to jhanas. Jhanas are pleasant,
discrete, reproducible altered states of consciousness, each more
refined and exquisite than the next. {[}I will present the jhanas as part
of the method, in Chapters X, Y.{]} I found that I suddenly had access to
four of these states. I had heard a little bit about the jhanas and what
they were supposed to be like, but this was my first experience with
them. The first four jhanas would normally arise in order during a
sitting: one, two, three, four. But I also found that I had random
access, and could jump to any jhana from any other, just by intending to
do so. The jhanas appeared as discrete channels to which I could attune
the mind, much like moving the dial of an old\sphinxhyphen{}fashioned radio. The depth
and clarity of these new meditative states was completely different from
the day before. I took this as further validation of stream entry.

\sphinxAtStartPar
Within hours of the event, I went to Sayadaw U Rajinda’s room and
knocked on his door to request an impromptu interview, something I had
never done before. I told him what had happened and hinted that I
understood this to be stream entry. U Rajinda hinted that he thought so
too, and gave me the new instruction to note “pleasant” and “unpleasant”
while sitting, and sent me back out to meditate some more. “Pleasant
experiences may arise in your sittings and they may stay for a long
time,” he said. “Be sure to note ‘pleasant’ when this happens.” After
six months in Malaysia (and a great deal of pleasantness), I flew to
Burma, where I would spend another six months at Panditarama Meditation
Center in Rangoon. My teacher there was the famous and cantankerous
Sayadaw U Pandita, abbot of Panditarama, highly decorated scholar, and
celebrated master of the technical aspects of meditation. In his
community, he was affectionately known as “Sayadawgyi” (pronounced
“sigh\sphinxhyphen{}a\sphinxhyphen{}dow JEE”) meaning “Big Sayadaw.” \sphinxstyleemphasis{Sayadaw} is itself an
honorific meaning “elder monk.” There were lots of Sayadaws in Burma,
but there was only one Sayadaw Gi at Panditarama.

\sphinxAtStartPar
U Pandita was interested in spreading the Buddhist teachings outside of
Burma, so he spent a lot of time with the foreign yogis, charging his
lieutenants with the supervision of the local Burmese students. We
foreigners (non\sphinxhyphen{}Burmese yogis, both Asian and Western), interviewed with
Sayadaw U Pandita several times a week, and we heard dharma talks by him
on the days we weren’t interviewing. The interviews were done through an
interpreter, even though U Pandita was able to understand a bit of
English. Interviews were one\sphinxhyphen{}on\sphinxhyphen{}one, but were done in front of the
entire group of 10–15 foreigners, so we all got to hear all the
interviews.

\sphinxAtStartPar
Several times a week, I would spend half an hour or so talking privately
with U Vivekananda, a German monk and disciple of U Pandita who has
since become a Sayadaw in his own right, and abbot of Panditarama
Lumbini in Nepal. U Vivekananda spoke perfect English and was more
forthcoming than Sayadaw U Pandita. Monk’s rules prevented him from
being completely open in our discussions, especially with regard to his
own experience, but I could ask him questions that I couldn’t ask U
Pandita.

\sphinxAtStartPar
Much about the authoritarian structure and hierarchy of the monastery
was difficult for me. Having to bow three times, on hands and knees,
forehead to the floor, before and after each interview with Sayadaw U
Pandita seemed like a charming custom at first, but eventually it was
just annoying. I wanted to engage U Pandita in discussion, and in my
more grandiose moments I even fantasized about educating him about what
I considered to be certain superior aspects of Western culture. Once, I
brought this up to U Vivekananda after a frosty encounter with U
Pandita. The German monk said. “Never argue with Sayadaw. He simply
can’t \sphinxstyleemphasis{tolerate} it.” This was obviously true, and part of the challenge
of monastery life was suppressing my own psychological need for open
engagement on a level playing field; there was no opportunity for that
whatsoever. I reminded myself again and again that it was worth the
pain; I was getting something that I couldn’t have gotten anywhere else.
So I stayed on.

\sphinxAtStartPar
There is a common misconception that a high level of contemplative
development will necessarily transform a human being into a lovable,
likable, caring, infinitely compassionate, and utterly sanitized cartoon
saint. Sayadaw U Pandita was living proof that this was not so; he
displayed the whole range of emotion. Although he could at times be
loving, kind, and supportive, more often he appeared angry, irritated,
cutting and sarcastic. In short, he was a mean old man. Between my
instinct for self\sphinxhyphen{}preservation and the powerful taboo against outright
disclosure, I judged it unwise to simply tell U Pandita that I believed
I had attained stream entry. All I could tell him was what I was
experiencing in my individual meditation sessions.

\sphinxAtStartPar
When I first arrived in Burma, I was still in the review phase after
stream entry, a kind of afterglow that follows attainment of a path.
This left me without much motivation for precise reporting; my
meditations were often so blissful that I would just sit and bask in
pleasure for an hour or more at a time. I wasn’t able to adequately
describe the precise phenomenology of those experiences, so to U Pandita
I was just being sloppy. He shouted, “You are dull, dreamy, drifty! This
is not acceptable!” He would angrily hold forth on the inferiority of
western yogis in general and Americans in particular. “You Americans,
you think you can do it your way! But here in Burma, there’s only \sphinxstyleemphasis{one}
way, and that is \sphinxstyleemphasis{my} way!” It would have been funny, had it not been so
intimidating. Kneeling on the floor of a monastery in a foreign land,
with the legendary Sayadaw U Pandita sitting cross\sphinxhyphen{}legged on his throne
above me, surrounded by his disciple monks, my Western notions of
equality did not apply.

\sphinxAtStartPar
I became obsessed with U Pandita. His presence filled my world. Every
waking moment was spent reflecting on our conversations and his
criticisms, along with imagined conversations in which I would
skillfully refute his attacks. But there was no future in this and I
knew it. My two options were to either follow Sayadaw’s instructions, or
to leave the retreat. After several weeks of internal turmoil and barely
restrained resentment during interviews, my resistance collapsed. In my
mind, I bowed to U Pandita and said, “I surrender. You’re the king. What
do you want me to do?” U Pandita recognized the change at our very next
interview. As soon as I began to do things his way, I saw the kind and
supportive side of the man. He smiled. “So now… you look like a yogi!”

\sphinxAtStartPar
What Sayadaw U Pandita wanted was for me to be almost painfully simple
in my reporting. He wanted me to say, for example, “when I observe the
rise and fall of the abdomen, I feel pressure, tightness, coolness,
warmth, softness. I feel mind states of fear, annoyance, joy,
equanimity.” These kinds of bare\sphinxhyphen{}bones explanations gave him the
information he needed to gauge my progress, place me on a map of
development, and give targeted advice. It was very important to him that
I not space out or get into sleepy, dreamy, states, and that I report in
simple, concrete terms, with little or no interpretation or commentary.

\sphinxAtStartPar
I did not attain 2nd Path on this first Asian retreat, in spite of my
earlier promise to Sayadaw U Rajinda during my stay in Malaysia. After
about a year of intensive meditation in the austere conditions of
Buddhist monasteries, I was sick and exhausted. I had lost 60 pounds of
bodyweight, going from 200 to 140 in 12 months. It was time to fly back
to the States and rest.


\section{Alaska}
\label{\detokenize{main-1:alaska}}
\sphinxAtStartPar
While I was in Asia, my parents bought a cabin by a lake near Haines,
Alaska, and got to know some people there, including a locally famous
artist and woodcarver. He had been to India as a spiritual seeker when
he was young, and understood the culture shock of returning home after
immersion in another culture. When he heard that I was returning from a
year\sphinxhyphen{}long retreat in Southeast Asia, he suggested to my parents that I
come live with him and his family and work in his art gallery as I
reintegrated into American culture. And that’s what I did. The artist
taught me the art of woodcarving in his unique style, which was in
heavily influenced by Northwest Coast Indian Art.

\sphinxAtStartPar
I was also involved in community theater in Haines. We performed a
melodrama every Saturday and Sunday night in the summertime for the
tourists who came on cruise ships. Occasionally we’d do a big production
as well, and when we performed \sphinxstyleemphasis{Fiddler on the Roof}, I played Fyedka,
the Russian boyfriend. My Alaskan adventure was a magical time, a
creative time, and the calm and clarity of mind resulting from a year of
intensive meditation practice made the breathtaking natural beauty of
Southeast Alaska seem all the more exquisite.

\sphinxAtStartPar
Shortly after arriving in Alaska, a local woman recruited me to teach
meditation, and I led a weekly sitting group. I would give talks and
teach basic techniques like following the breath and noting in the
Mahasi style. I also taught the model of the four paths model of
enlightenment that I’d learned from Bill Hamilton and my Asian teachers,
and spoke of stream entry as a realistic goal. I made no secret of the
fact that I believed I had attained stream entry. These views were met,
for the most part, with resistance or indifference, but there were a
handful of people who became my friends and came regularly to the weekly
sittings.

\sphinxAtStartPar
Most people in that tiny Alaskan town thought of me as a kind of odd
quasi\sphinxhyphen{}monk. I didn’t date throughout the time I spent in Alaska, mainly
because the women I was interested in attended my sitting group, and it
was clear to me that it wasn’t a good idea for teachers to date their
students. So I continued my monkish ways. My parents were spending the
summers in their cabin 26 miles outside of town, so every weekend my
father would pick me up and we would drive out to the cabin together. My
dad was an avid, even obsessive, fisherman. He and I would go fishing
every weekend, both Saturday and Sunday, on a nearby river or on the
lake just outside the cabin’s back door. Mother was an intellectual and
a reader; she and I would discuss ideas in the evening, sitting at the
table in the one\sphinxhyphen{}room cabin as Dad lovingly cared for his fishing
tackle. In the wintertime, Mom and Dad would drive back to their place
in Oregon, and I spent my first Alaskan winter alone in the cabin by the
now frozen lake. There was no electricity. Illumination was courtesy of
gaslights, and the cabin’s only heat source was a leaky old Franklin
stove that required constant attention and had an insatiable appetite
for firewood. I spent a lot of time alone that first winter, splitting
wood, feeding the stove, reading, and meditating.

\sphinxAtStartPar
I returned to Southeast Asia twice for retreats during the period I
called Alaska home; two months in Malaysia for the first trip, and four
months for the second, half in Malaysia and half in Burma. My teacher in
Burma on my third Asian retreat was the lovable Sayadaw U Kundala, a
direct disciple of Mahasi Sayadaw who was as famous as U Pandita in
Burma, but less well known in the West. It was on this retreat with U
Kundala that I attained second path. Both stream entry and second path
were so obvious to me that I didn’t require validation from my teachers.
Nontheless, Sayadaw U Kundala validated me. I described the
phenomenology of the
cessations I was experiencing, which at the time I experienced as
cluster of visual freeze frames, often in quick succession, and U
Kundala declared: “Oh, this is \sphinxstyleemphasis{magga phala}!” \sphinxstyleemphasis{Magga} and \sphinxstyleemphasis{phala} are
the Pali words for “path” and “fruition,” respectively. I explained that
I had experienced this once before, two years earlier in Malaysia, and
had been through the stages of the Progress of Insight before as well.
Sayadaw U Kundala acknowledged this and even taught me how to use
resolutions to re\sphinxhyphen{}experience the cessations of stream entry and second
path, which were subtly different.

\sphinxAtStartPar
I was impressed by how open Sayadaw U Kundala and many of the Burmese
Buddhists were in talking about meditative attainments and progress.
There seemed to be a whole culture of acknowledging attainment at U
Kundala’s monastery. According to Buddhist tradition, giving a gift to
someone who has attained some level of enlightenment accrues spiritual
merit to the giver, so as word got out (various people were able to
overhear my interviews with U Kundala) that a Western student was making
progress, people began to come to my room and offer me gifts, including
some beautiful silk sarongs and a warm Russian\sphinxhyphen{}style furry hat with
earflaps for chilly Rangoon mornings in the cool season.

\sphinxAtStartPar
When I finished the retreat and was ready to leave Burma, I got a ride
to the airport from a man who was a board member at the monastery. He
was clearly well connected in Rangoon, because when we get to the
airport, he waved his hand at a bunch of soldiers with assault rifles,
causing them to stand back and let me pass, so that I didn’t have to
wait in the customs line with the other hapless tourists. As I was
walking away, the man from the monastery waved goodbye and shouted to me
across the crowded airport, “You got two! Come back for a third,” in a
less\sphinxhyphen{}than\sphinxhyphen{}veiled reference to the four paths of enlightenment.

\sphinxAtStartPar
I had remained in touch with Bill Hamilton throughout my time in Alaska,
and would periodically go to California or Bill’s Whidbey Island Retreat
in Washington to spend time with him or to do a silent retreat.
Returning from my third Asian trip, I stopped by Whidbey Island. Bill
confirmed my attainment of second path in his characteristic indirect
style; I later heard from a mutual friend, “Bill said you attained
second path!” This was part of the odd, roundabout way in which Bill
liked to communicate about attainments. In my own teaching, I have taken
the Burmese willingness to talk about attainments one step further; I
speak openly about my own attainments and freely give my opinion to my
students about where I think they can be placed on a developmental map.
{[}It’s worth pointing out that I don’t consider myself the final arbiter
of other people’s attainments. I’m not in a position to validate or
invalidate other people’s attainments; I can only give my opinion. An
attainment either happens or it doesn’t, irrespective of anyone’s
opinion including the teacher and the student herself.{]}


\section{Deeper Into Jhana}
\label{\detokenize{main-1:deeper-into-jhana}}
\sphinxAtStartPar
After attaining second path and returning from Asia, I spent several
months meditating at Bill Hamilton’s Whidbey Island Retreat. I had
gained access to the first four jhanas with stream entry, and I now set
out to develop jhanas 5\sphinxhyphen{}8, four more altered states commonly described
in Theravada Buddhist literature. I thought of jhana practice as support
for my vipassana practice as well as being interesting in its own right,
and I wanted to be able to access these states on demand. I planned to
do \sphinxstyleemphasis{kasina} practice, a type of pure concentration practice in which the
attention is held steady on a single object or concept. {[}I had secretly
done some of this kind of practice on my Asian retreats too, without
telling anyone.{]} I had a brown plastic sluicing bowl from Burma that I
had bought to bathe from water tanks when I didn’t have access to a
shower, and this bowl was to serve as my kasina. I propped up the bowl
in my tent on Whidbey Island and got to work; the work in this case was
as simple as staring at the bowl for hours at a time. {[}I was probably
following instructions from the Visudimagga, of which Bill had a copy,
and the “One by one as they occur” sutta (MN 111).{]} Using this
technique, I was able to develop the fifth jhana within a few days, and
the others followed one after the other until I had access to eight
distinct altered states.

\sphinxAtStartPar
Having attained the first two levels of enlightenment according to the
four paths model, my next landmark would be third path. I had heard a
little about third path from Sayadaw U Pandita and from Bill Hamilton.
It was said that first and second path were fairly straightforward and I
had even heard from Bill Hamilton that first and second path were “a
dime a dozen” in Buddhist circles. But third path was considered more
difficult, rare, and harder to diagnose. And in fact, although my own
attainment of stream entry and second path were self\sphinxhyphen{}evident, the exact
moment in which I attained third path is not clear to me.

\sphinxAtStartPar
Some time in the mid–90s, I discovered a new set of jhanas beyond the
eight commonly taught within Buddhism. I was sitting in my car after a
grocery shopping expedition. I remembered reading about a vow that a
Buddha named Amitabha had made. According to what I remembered of the
mythology, Amitabha once vowed that anyone who sincerely invoked his
name would be instantly transported to the Pure Land, a kind of Buddhist
heaven. Notwithstanding my initial experience of mystical union in 1982
and a few brief flirtations with religious concepts, I had remained as
skeptical as ever, and didn’t believe for a moment that there was a
magical Buddha named Amitabha up in the sky, poised to intervene on my
behalf. Still, I well understood the power of metaphor and suggestion in
human experience, so I decided to try an experiment. With as much
sincerity as I could muster, I invoked Amitabha Buddha by repeating the
phrase “Namo Amitabha” over and over. Almost immediately, I entered a
state of boundless gratitude and happiness that I hadn’t felt before.
This was a discrete altered state, but not one of the eight jhanas I was
already familiar with. This new state was so pleasant and profound that
one of my first thoughts was I would happily toss away all of the
previous eight jhanas in return for this one. I found that I could
conjure up this new jhana at will by picturing Amitabha Buddha in his
traditional red robes, by recalling the sense of boundless gratitude, or
by focusing on the “third eye” area in the middle of my forehead, which
was experienced prominently in this state. I dubbed this new state the
“Pure Land jhana,” since it had come from the Pure Land Buddhist
practice of invoking Amitabha Buddha. A few months later, I went on
another retreat with Bill at Whidbey Island and discovered another
altered state in the same mental territory as the Pure Land jhana. It
wasn’t the same state as the Pure Land jhana, but was of similar
character, so I began thinking of these states as Pure Land One and Pure
Land Two.

\sphinxAtStartPar
While I was cultivating the Pure Land jhanas on Whidbey Island, I
received a letter from my good friend and former meditation student
Daniel Ingram in which he claimed access to a state called nirodha
samapatti. {[}The Pure Land jhanas and nirodha samapatti are discussed in
more detail in Chapter X.{]} Nirodha samapatti (NS) is a special
meditative phenomenon that is said to only be accessible to \sphinxstyleemphasis{anagamis}
(those who had attained the third path of enlightenment) and \sphinxstyleemphasis{arahats}
(fourth path practitioners, the “fully enlightened”). I once heard U
Pandita describe NS during a dharma talk in Rangoon as “a way of
accessing nibbana” {[}Nibbana is the Pali word for the Sanskrit Nirvana.{]}
that “those nobles ones, the \sphinxstyleemphasis{anagamis} and \sphinxstyleemphasis{arahats}” had in their
bag of tricks. The developmental aspect got my attention; if only 3rd
path yogis and beyond had access to nirodha samapatti, then access to NS
was necessarily a key diagnostic criterion. To access NS was to be an
anagami, a developmental attainment supposedly so lofty that most modern
Buddhist practitioners did not consider it a reasonable goal. And here
was my friend Daniel claiming to have access to NS, and accordingly
claiming to have attained third path. I didn’t believe him. I wrote back
to Daniel suggesting that he get over himself and keep practicing. Years
later, though, perhaps in 2003, I found that I also was able to access
to a curious state that seemed to line up with textual descriptions of
nirodha samapatti. Daniel and I compared notes and seemed to be
experiencing the same thing. Together and separately, we have since
heard many of our students describe a similar phenomenon.

\sphinxAtStartPar
In the face of the prevailing Buddhist culture, which holds that we live
in a degenerate age and that it is not possible for modern humans to
achieve the same levels of awakening as the great mystics of the past,
it is natural to ask whether what I identify as nirodha samapatti is the
same phenomenon described by the ancients. Unfortunately, I cannot know
the answer to this question. I can never be sure that any of the
experiences described by others correspond exactly with my own, and this
kind of comparative mind\sphinxhyphen{}mapping becomes all the more difficult if the
other people involved are dead or unwilling to talk openly. The larger
issues here are dear to me and have become a mainstay of my practice and
teaching. Are modern people capable of attaining the high levels of
contemplative development spoken of in ancient texts as “awakening”? I
believe we are. In fact, it’s hard for me to imagine what would prevent
it. To the extent that the accomplishments of ancient meditators seem
beyond our reach, I suspect it has more to do with hyperbole and
hagiography than with any inadequacy on the part of modern humans. I
believe it is realistic for us to reach and even go beyond the
achievements of the ancients and I practice and teach accordingly.
Contemplative fitness is within everyone’s reach, and contemplative
excellence is there for those of us willing to dedicate our lives to its
pursuit. In this, we see yet another parallel between physical and
contemplative fitness. As for the conventional wisdom that awakening not
only leads to but is defined by moral perfection, omniscience, etc., my
answer is simple; I don’t believe there ever was a morally perfected or
omniscient human. Whatever contemplative development the ancients were
describing, it did not entail perfection. And yet, I am convinced that
ancient advocates of meditation were pointing to something real and
infinitely valuable. My ongoing efforts to separate reality from
fantasy, abandoning childish notions of perfection while continuing to
cultivate contemplative excellence are a large part of what motivates my
teaching (and this book.)


\section{Third Path}
\label{\detokenize{main-1:third-path}}
\sphinxAtStartPar
Years after I first came across the two Pure Land jhanas, I found a list
of the 31 realms of existence of Buddhist cosmology on the Internet.
{[}\sphinxurl{http://www.accesstoinsight.org/ptf/dhamma/sagga/loka.html}{]} The 31
mythical realms were mapped to jhanas. I’d seen a poster years before,
hanging on the wall in the Malaysian Buddhist Meditation Centre, that
lined up the 31 realms to jhanas in a similar way. In the online map,
there were five realms said to be accessible to anagamis and arahats
only. These were labeled as \sphinxstyleemphasis{suddhavasa} realms or ”Pure Abodes.” This
was a great “aha” moment for me. From here, it wasn’t much of a stretch
to connect the states I had independently named “Pure Land” jhanas with
these “Pure Abode” realms. Bill Hamilton often spoke of mapping our mind
states to the Buddhist realms. For Bill, the Buddhist realms of
existence were, above all, a mind map; irrespective of whether one
believed they had any independent existence, we could view the realms as
corresponding to layers of mind that were stable enough to be taken as
object and accessed as jhanas. It also seemed plausible that the reason
no one seemed to be talking about or teaching any jhanas beyond the
first eight was that the Pure Abodes were developmental; most people
could not access them, and admitting that you \sphinxstyleemphasis{could} was tantamount to
claiming to be an anagami, which neither monks (because of the rules of
their order) nor neo\sphinxhyphen{}Buddhists steeped in the culture of non\sphinxhyphen{}disclosure
were likely to do. There is, to this day, very little information online
about the Pure Land jhanas. Ten years ago, there was even less. So it
was left to me to explore this territory on my own, and see what I could
see. Since I was initially able to access only two such states, the fact
that five pure abodes were listed on the 31 Realms map was provocative
in the extreme. My interest in nirodha samapatti, which was also said to
be accessible only to anagamis and arahats, tied in with this; I
wondered if NS might be one of the five realms in question.

\sphinxAtStartPar
I worked on expanding my understanding of these realms for about a year,
and went on a retreat at the Forest Refuge for this purpose. While
meditating formally, I would ride what I called the “jhanic arc” up and
down through the available strata of mind, and open to the possibility
that there might be another layer above the ones I knew, i.e., above the
second Pure Land jhana. {[}“Riding the jhanic arc”, a method I invented
for accessing and developing strata of mind, is discussed in Chapter X.{]}
As a result of this targeted exploration, I found that there was such a
state! A new layer opened up, a new jhana that felt as though it were
from the same family as the two Pure Land jhanas. I still considered the
possibility that nirodha samapatti was one these realms, but I
eventually accessed five discrete Pure Land jhanas, none of which were
nirodha samapatti. I asked myself if nirodha samapatti fell naturally
into the sequence of the set of the pure land jhanas, and as far as I
could tell, it did not. So I ended up with a set of states that can only
be accessed by practitioners who have attained to at least third path:
the five pure land jhanas, and nirodha samapatti. I’ve taught a number
of people to access all five pure land jhanas. As far as I know, no one
has come up with any additional jhanas, so to my knowledge this is the
complete set: the four material jhanas, the four immaterial or formless
jhanas, and the five Pure Land jhanas.


\section{Disillusionment}
\label{\detokenize{main-1:disillusionment}}
\sphinxAtStartPar
I had been led to believe that stream entry and certainly second and
third path were so lofty and quasi\sphinxhyphen{}holy that by the time you had them,
you’d basically be on easy street; if your life wasn’t yet a cosmic
bliss out, it was certainly on the way. If anyone had said I would still
be depressed after the second path of enlightenment I wouldn’t have
believed it. But as it happened, by the standard diagnostic criteria I
learned from the Mahasi system, by 1994 I \sphinxstyleemphasis{did} have second path and I
was \sphinxstyleemphasis{still} depressed. By 2003, I believed I had attained third path
too, but my life was still in shambles. There was a rift between what
was happening and what I thought \sphinxstyleemphasis{ought} to be happening. On the one
hand, I was a meditation expert; I had a high level of facility with
altered states, knew a great deal of Buddhist theory, and had had myriad
fascinating and profound experiences. I could easily access jhanas, and
use them to temporarily remedy my problematic mind states, but it wasn’t
enough. Depression and anxiety continued. It seemed to me that my brain
chemistry was seriously fouled up, and this movement via my meditation
practice through what I thought of as an organic, somehow biological
spectrum of development was not addressing my mental health issues. I
was becoming resigned to the conclusion that meditation would help me
accept my depression but would not help me overcome it. I bitterly came
to terms with my depression as a long\sphinxhyphen{}term, chronic problem that might
be with me for the rest of my life; in 1999, I begged a friend to take
me by the hand to the county mental health clinic and help me ask the
doctor for antidepressant medication.

\sphinxAtStartPar
My spiritual opening on LSD in 1982 had sent me on a quest for
enlightenment, and I was still caught up in that current. The feeling
that I was on a ride towards enlightenment consumed me. Although I
wouldn’t have been able to articulate it at the time, what I really
wanted was to be done with it; I wanted the ride to stop. My meditation
practice was a blessing and a curse, because I was moving along this
developmental continuum in ways that were rich and fulfilling, and yet
it was torture to wake up each morning with the feeling that something
important remained to be done. I didn’t know how to proceed. My practice
had given me access to entirely new categories of pleasant mind states,
but this access was not reliable. Instability was the curse of third
path. I kept practicing only because I didn’t know what else to do.


\section{Off the Ride}
\label{\detokenize{main-1:off-the-ride}}
\sphinxAtStartPar
In June of 2004, I went on a retreat at Southwest Sangha in New Mexico.
One day, walking under a pepper tree in the desert, I gave myself
permission to be enlightened. I had been practicing obsessively for
twenty\sphinxhyphen{}two years, including a cumulative three years on intensive
retreat. I thought of myself as a professional yogi. On this day in New
Mexico, reflecting on the question “have I suffered enough?” I gave
myself permission to be done. I was acutely aware of everything around
me — the sights and sounds of the desert, the feeling of heat on my
skin, the warm breeze on my face, the pulsing in my veins. It suddenly
occurred to me that I \sphinxstyleemphasis{was} done. The current that had carried me for so
many years had relaxed. The ride that had begun the day I first saw the
white light in 1982, this thing that had taken hold of me and had been
the most important thing in my life for these twenty two years, was
over.

\sphinxAtStartPar
I felt like Dorothy in the Wizard of Oz, clicking her heels three times,
and then waking up to find that she’d been home in her own bed the whole
time, safe and sound. I called my mother the next day and told her what
had happened. “I think I’ve just wasted twenty\sphinxhyphen{}two years of my life. The
ride is over and nothing much has changed. But I have never been
happier. There is peace.”

\sphinxAtStartPar
The essential realization that comes from this process is that there
isn’t anyone here to get enlightened. You work tirelessly for years to
get enlightened, only to find out that you couldn’t possibly get
enlightened, because there isn’t anybody here for it to happen to.
Contemplative development, in its purest sense, is learning to see
yourself as process.

\sphinxAtStartPar
I walked back into my little trailer in the desert and wrote on the
calendar, “I see the elephant.” This was a reference to the parable of
the blind men and the elephant.
{[}\sphinxurl{http://en.wikipedia.org/wiki/Blind\_men\_and\_an\_elephant}{]} I’d been
able to see parts of the puzzle before, but now it came together. I saw
the elephant. My depression went away. I weaned myself off of
antidepressants and anti\sphinxhyphen{}anxiety medication over a period of several
months. I stopped having trouble sleeping. It does not happen this way
for everyone, but this is what happened to me.

\sphinxAtStartPar
By the way, what is an \sphinxstyleemphasis{arahat}? According to Theravada Buddhism, an
arahat is a “fully enlightened” being. This person has attained all four
of the Four Paths of enlightenment. Some say arahats are extremely rare,
although in the time of the Buddha, they were apparently as common as
ants at a picnic. Whether there are few arahats or many, or for that
matter, any at all, depends entirely upon the definition of the word. By
one popular definition, an arahat is a kind of superman who has
transcended human emotions. He has “overcome greed, hatred, and
delusion.” In other words, he does not experience \sphinxstyleemphasis{or express} fear,
anger, hate, lust, envy, nor any other “afflictive” emotion. By this
definition, it’s not surprising that there don’t seem to be many around.
In fact, I doubt there ever was a person like that, Siddhatta Gotama
Buddha included.

\sphinxAtStartPar
My own preferred definition is much less ambitious and, I believe, much
more useful. Moreover, I believe it is what the people who originally
coined the word meant when they said it. An arahat is someone who has
come to the end of a particular developmental process. The process of
which I speak is familiar to anyone who has had a spiritual opening.
Once it is set in motion, there is a kind of visceral pull that propels
one to practice more. There is the feeling that one is moving
toward…something…one knows not what. But there is the pull. It will not
be denied, and ignore it at your peril. Almost all yogis know this pull.
But some yogis also know the end of it. These yogis are arahats.

\sphinxAtStartPar
An arahat is not a superman. An arahat is off the ride. Viewed through
this lens, the old stories suddenly make sense. According to the suttas
{[}Suttas (Pali) or sutras (Sanskrit) are the Buddhist scriptures that
record the oral teachings of the Buddha.{]}, it was fairly routine for
someone to walk up to the Buddha and say something like “Done is what
needed to be done.” Why did they say it like that? Because that’s what
it feels like. How do I know? Because it happened to me on June 13th,
2004, while walking under a pepper tree in New Mexico. A circuit was
completed that day. A palpable energy that had been working its way
through my body for 22 years completed its circuit and has been
recycling ever since, stable, without any sense that anything else needs
to be done.

\sphinxAtStartPar
It would be impossible to overstate what a profound change this caused
in my understanding of my own life. The pull I spoke of earlier, the
sense of “being on a ride,” and needing to see it through to its
conclusion, had formed the backdrop for nearly my entire adult life.
Suddenly, it was over. What should I do now? At the very least, I would
have to find another project. All of this was clear in a moment. I
chuckled, turned to an imaginary Buddha standing next to me and said,
“Done is what needed to be done. You got nothin’ on me now.” I
understood that there was not, had never been a Buddha outside of me. I
was finally free… and yet it wasn’t me. It was just a constellation of
thoughts and sensations conveniently designated Kenneth.

\sphinxAtStartPar
There is infinite opportunity for misunderstanding here, so I want to be
as clear as possible. Being done refers only to the attainment of a
particular landmark along a natural developmental continuum. It does not
mean, contrary to hyperbolic legend, that the arahat has “erased all
karma,” “perfected him or herself,” etc. Those are children’s stories,
told by charlatans or starry\sphinxhyphen{}eyed apologists (or pre\sphinxhyphen{}industrial
quasi\sphinxhyphen{}biographers who depended on mythic deeds as a vehicle for their
stories).

\sphinxAtStartPar
Simply being enlightened will not magically transform you into a
glow\sphinxhyphen{}in\sphinxhyphen{}the\sphinxhyphen{}dark saint. It won’t necessarily even make you a good
person. The evidence for this is all around us, as we see that it is not
exceptional, but rather the norm, for enlightened teachers to get caught
with their pants down. If we were to give up our childish expectations
of saintly behavior from our sages, we would not have to feign surprise
when they succumb to the same human temptations that plague the rest of
us.

\sphinxAtStartPar
I understand that many will not accept my definition of enlightenment.
They will mumble something about “higher standards” and go on believing
in superheroes. But I think there may be some who are ready to take a
mature and realistic look at what enlightenment can and cannot do for us
as individuals and as a society. For them, the empowerment of knowing
that enlightenment, even to the level of arahat, is possible, will
outweigh the disappointment of having to give up the fantasy of infinite
wisdom, moral perfection, and steady\sphinxhyphen{}state bliss.


\section{Full Circle}
\label{\detokenize{main-1:full-circle}}
\sphinxAtStartPar
After my New Mexico retreat, I drove to Barre, MA, and worked at Insight
Meditation Society on and off for about a year and a half in the
maintenance and IT departments. Almost immediately after arriving at
IMS, I met my wife, Beth, who worked in the retreat center as a cook. My
chronic depression, which had left me dysfunctional for months at a time
throughout my adult life, was gone, and I no longer felt the need to
subordinate everything else to my spiritual quest. I was able to get my
life together. I went back to school and earned a bachelor’s degree in
Spanish literature and culture, and a master’s degree in second language
education, both from the State University of New York. While still in
school, I began teaching meditation over Skype and I’ve since become a
full\sphinxhyphen{}time meditation teacher, making my living doing something I enjoy
and passionately believe in. Beth and I stayed together, and were
married in 2008. The relief that comes from having gotten off the ride
isn’t what I thought it would be; it is not a cosmic bliss\sphinxhyphen{}out or a
perpetual beatific smile, but rather a deep, abiding sense of peace, and
the feeling that there’s no longer anything missing from my life or the
universe. This doesn’t prevent my taking on projects, having goals and
motivations and seeing them through, or caring about my life and the
people and things around me. Nor does it erase the difficulties of an
ordinary human life. Life continues as before, but with less sting.
Contentment underlies all, much as the deep sea underlies the froth on
the surface of the waves. Even the most violent storms do not disturb
those depths.

\sphinxAtStartPar
No description of awakening is adequate. If you get there, you will be
surprised, no matter what you hear or read in the meantime. In response
to my questions about enlightenment during the early years of my
practice, Bill Hamilton used to say, “Highly recommended. Can’t tell you
why.” I’ll end my story here, not because my story has ended, but
because it hasn’t; my story is ongoing and will not tidily fit within
these pages. Indeed, my life has changed as much in the last nine years
(since that day in the desert) as in any nine\sphinxhyphen{}year period of my life. I
have not retired or put myself out to pasture. I teach, learn, meditate,
spend time with my wife, family, and community, and run a business. But
my purpose in teaching meditation is not to make clones of myself; I see
contemplative fitness as analogous to physical fitness. Every individual
is unique. Your contemplative fitness will be your own. There is no
universal ideal, and no predetermined outcome.

\sphinxAtStartPar
If you want to be strong, lift weights. If you want to be well\sphinxhyphen{}educated,
go to school. If you want to awaken, meditate. The rest of this book
will show you how.

\sphinxstepscope


\chapter{Book Two: Theory}
\label{\detokenize{main-2:book-two-theory}}\label{\detokenize{main-2::doc}}

\section{Watering Down the Dharma}
\label{\detokenize{main-2:watering-down-the-dharma}}
\sphinxAtStartPar
I have been accused of “watering down the dharma.” By defining an
\sphinxstyleemphasis{arahat} (also \sphinxstyleemphasis{arhat} and \sphinxstyleemphasis{arahant}) as someone who has “gotten off the
ride” and can see experience as process, as opposed to a cartoon saint,
I have ruffled more than a few feathers. Here are some questions, along
with my responses:
\begin{quote}

\sphinxAtStartPar
Why are you redefining the Four Paths of Theravada Buddhism?
\end{quote}

\sphinxAtStartPar
There is an old joke in which a man is asked, “Do you still beat your
wife?” The person being asked is put into an untenable situation by the
false assumption built into the question. The assumption is that you
have beaten your wife in the past. If you answer “no,” the questioner
will follow up with “when did you stop?”

\sphinxAtStartPar
Best to reject the question entirely.
\begin{quote}

\sphinxAtStartPar
Why are you redefining the Four Paths?
\end{quote}

\sphinxAtStartPar
I reject the question. It is false to assume that there is One Right Way
to interpret ancient texts, providing an infallible orthodoxy against
which all other interpretations must be compared and inevitably found
lacking. There is no One Right Way.

\sphinxAtStartPar
The authors who penned the early Buddhist texts are no longer available
for comment. We can only guess at their intentions. Modern commentators
who insist that they know the original meaning of \sphinxstyleemphasis{arahat} are
overplaying their hand, regardless of how scholarly or ostensibly
traditional their arguments.

\sphinxAtStartPar
Like everyone else who has an opinion about this, I am simply throwing
my hat into the ring; I offer one possible interpretation of the Four
Paths model. This interpretation is based on the Pali Canon and
commentaries, rooted in observed reality, and nurtured by pragmatism.
Implausible claims are sooner discarded than taken at face value. But
even after discarding the implausible, the unprovable, and the downright
silly, what is left is too good to ignore; enlightenment is much more
than a myth, it happens to real people in our own time, and it can be
systematically developed through known practices.

\sphinxAtStartPar
It seems likely that the Buddhist definition of “fully enlightened”
changed over time in a kind of slow motion frenzy of one\sphinxhyphen{}upmanship. Here
is a passage from palikanon.com, attributed to W.G. Weeraratne, author
of several books on Buddhism and editor\sphinxhyphen{}in\sphinxhyphen{}chief of the prodigiously
researched Encyclopaedia of Buddhism, published by the government of Sri
Lanka:
\begin{quote}

\sphinxAtStartPar
In its usage in early Buddhism the term {[}arahant{]} denotes a person
who had gained insight into the true nature of things
(yathābhūtañana). In the Buddhist movement the Buddha was the first
arahant… The Buddha is said to be equal to an arahant in point of
attainment, the only distinction being that the Buddha was the
pioneer on the path to that attainment, while arahants are those who
attain the same state having followed the path trodden by the
Buddha. \sphinxurl{http://www.palikanon.com/english/pali\_names/ay/arahat.htm}
\end{quote}

\sphinxAtStartPar
Note that “insight into the true nature of things” sounds as though it
might be within reach of anyone. (In a moment, we’ll discuss what the
early Buddhists believed this “true nature” to be.) And indeed it was
not the least bit unusual for people practicing the Buddha’s system to
become “fully enlightened arahats” according to early Buddhist texts.
But look what happened next:
\begin{quote}

\sphinxAtStartPar
But, as time passed, the Buddha\sphinxhyphen{}concept developed and special
attributes were assigned to the Buddha. A Buddha possesses the six
fold super\sphinxhyphen{}knowledge (chalabhiññā); he has matured the thirty\sphinxhyphen{}seven
limbs of enlightenment (bodhipakkhika dhamma); in him compassion
(karunā) and insight (paññā) develop to their fullest; all the major
and minor characteristics of a great man (mahāpurisa) appear on his
body; he is possessed of the ten powers (dasa bala) and the four
confidences (catu vesārajja); and he has had to practise the ten
perfections (pāramitā) during a long period of time in the past.

\sphinxAtStartPar
When speaking of arahants these attributes are never mentioned
together, though a particular arahant may have one, two or more of
the attributes discussed in connection with the Buddha (S.II.217,
222). In the Nidāna Samyutta (S.II.120\sphinxhyphen{}6) a group of bhikkhus who
proclaimed their attainment of arahantship, when questioned by their
colleagues about it, denied that they had developed the five kinds
of super\sphinxhyphen{}knowledge—namely, psychic power (iddhi\sphinxhyphen{}vidhā), divine ear
(dibba\sphinxhyphen{}sota), knowledge of others’ minds (paracitta\sphinxhyphen{}vijānana), power
to recall to mind past births (pubbenivāsānussati) and knowledge
regarding other peoples’ rebirths (cutū\sphinxhyphen{}papatti)—and declared that
they had attained arahantship by developing wisdom (paññā\sphinxhyphen{}vimutti).
\sphinxurl{http://www.palikanon.com/english/pali\_names/ay/arahat.htm}
\end{quote}

\sphinxAtStartPar
Hmmm… So it looks as though the meanings of the words \sphinxstyleemphasis{Buddha} and
\sphinxstyleemphasis{arahat} changed over time, with more and more powers and attributes
layered on. Eventually, the lists of things arahats could do and the
lists of things they had left behind became so long that no living
person, past or present could reasonably be expected to make the cut.
This is where we find ourselves today, assuming we believe the currently
popular (among Buddhists) kitchen\sphinxhyphen{}sink version of enlightenment.

\sphinxAtStartPar
Let’s go back to the beginning for a moment.
\begin{quote}

\sphinxAtStartPar
In its usage in early Buddhism the term {[}arahat{]} denotes a person who had gained insight into the true nature of things.
\end{quote}

\sphinxAtStartPar
It would be useful to know what the early Buddhists may have meant by
the “true nature of things.” Here is more from Weeraratne:
\begin{quote}

\sphinxAtStartPar
At the outset, once an adherent realised the true nature of things,
i.e., that whatever has arisen (samudaya\sphinxhyphen{}dhamma) naturally has a
ceasing\sphinxhyphen{}to\sphinxhyphen{}be (nirodhā\sphinxhyphen{}dhamma), he was called an arahant…
\sphinxurl{http://www.palikanon.com/english/pali\_names/ay/arahat.htm}
\end{quote}

\sphinxAtStartPar
Are you seeing what I’m seeing? Not only is full enlightenment
(arahatship) a perfectly reasonable thing for ordinary people to aspire
to and attain, the Buddha himself was initially considered just another
enlightened man, albeit the first of his group. All that was required
was to see that anything that “has arisen, naturally has a
ceasing\sphinxhyphen{}to\sphinxhyphen{}be.” (And may I humbly submit that this is precisely what I
mean when I advocate learning to see this experience as process. While
trivial as a mere concept, the ability to see this in real time is
life\sphinxhyphen{}changing.) I find this empowering beyond words. Although I would be
perfectly willing to dispense with Buddhism entirely if it did not have
anything to offer us at this point in our history, I love the fact that
2500 years ago, humans discovered a technology for mental development
that still works today. And I love the fact that once you strip away the
accretions of thousands of years of can\sphinxhyphen{}you\sphinxhyphen{}top\sphinxhyphen{}this storytellers, it
all seems perfectly do\sphinxhyphen{}able to us ordinary folks. It \sphinxstyleemphasis{is} perfectly
do\sphinxhyphen{}able, of course, and this is my entire point.

\sphinxAtStartPar
In interpreting ancient Buddhist maps, it is necessary to begin with a
few assumptions. Here are mine: I begin with the assumption that the
chroniclers of early Buddhism were pointing to something that was
happening around them (or to them), but were limited by the obligatory
biography\sphinxhyphen{}as\sphinxhyphen{}hagiography storytelling style of their day. I continue
with the assumption that what was possible in the 5th Century BCE is
still possible today. Next, I strip away the implausible and preserve
the plausible. It is implausible that ancient meditators defied gravity,
traveled through time, performed miracles, or overcame their human
biology. On the other hand, it is plausible that awakening, as it was
then understood, was commonplace among meditators in the time of the
Buddha. (A common theme of early Buddhist documents is that nearly
everyone who did the Buddha’s practice became fully enlightened.) I
conclude that there is an organic process of development that results
from meditation. It need not be mystical or magical, and we can just as
easily think of it as brain development. Finally, and most importantly,
I reality\sphinxhyphen{}test these assumptions with observations of present day
humans, using my subjective experience, interviews with other
meditators, and the carefully documented reports of present\sphinxhyphen{}day
meditators available in books and online forums.

\sphinxAtStartPar
Before I present a side\sphinxhyphen{}by\sphinxhyphen{}side comparison of two competing models of
arahatship, we might reasonably ask whether a stage model of
contemplative development has any value at all. I believe it does.
Humans learn best when they are given discrete goals and regular
assessments of progress. I have heard the protestations of those who
believe that meditation must never be a goal\sphinxhyphen{}oriented activity, and that
this holy truth renders all stage models either counterproductive or
irrelevant. I refer such people to the success of my students. And for
those who crave a more authoritative (authoritarian?) voice, I would
point out that according to that most definitive of Buddhist sources,
the Pali Canon, the dying words of the Buddha were “Strive diligently.”

\sphinxAtStartPar
We can compare and contrast my model (let’s call it the \sphinxstyleemphasis{Pragmatic
Model}) with a model that is currently in vogue among Buddhists, and
which we might reasonably call the \sphinxstyleemphasis{Saint Model}. First, the
definitions:


\subsection{The Pragmatic Model of Arahatship}
\label{\detokenize{main-2:the-pragmatic-model-of-arahatship}}
\sphinxAtStartPar
These people know they are done; they have come to the end of seeking.
Although they may continue to meditate with great enthusiasm, and
continue to deepen and refine important aspects of their understanding
throughout their lives, they do not feel there is anything they need to
do vis a vis their own awakening. This is in marked contrast to the
pre\sphinxhyphen{}arahat meditator, who tends to be obsessed with meditation and
progress. Equally important, the Pragmatic Model arahat is able to see
experience as process. There is no enduring sense of self at the center
of experience. The Buddhist ideal of insight into not\sphinxhyphen{}self has been
completely realized and integrated.


\subsection{The Saint Model of Arahatship}
\label{\detokenize{main-2:the-saint-model-of-arahatship}}
\sphinxAtStartPar
This person does not suffer. No negative emotion is felt \sphinxstyleemphasis{or expressed}.
Ever. (I have emphasized the \sphinxstyleemphasis{expression} of negative emotions because
there will always be individuals who claim not to feel negative emotions
even while expressing them in a way that is obvious to everyone around
them. Doesn’t count.) No anger, resentment, annoyance, irritation,
aversion, impatience, or restlessness is allowed. There is no sensual
desire, and this applies to both food and sex. This person cannot
compare himself/herself with others, either favorably or unfavorably.
This person is unable to lie, steal, speak harshly, or kill a sentient
being, including insects. Did I mention omniscience and diverse psychic
powers including mind reading? This person is a saint by the most
rigorous interpretation of the word.


\subsection{Comparing the models}
\label{\detokenize{main-2:comparing-the-models}}
\sphinxAtStartPar
For a developmental model to be relevant to modern humans, it should
describe something that actually happens and can be observed today. It
should happen often enough to form a reasonable sample size for study.
The Pragmatic Model does this. I estimate that I have communicated with
20\sphinxhyphen{}30 people who might be considered arahats by this model. Since I
personally know only a tiny fraction of the humans on Earth, it is
reasonable to assume that this is only the tip of the iceberg, and there
are many hundreds or thousands of such people whom I have not yet met.

\sphinxAtStartPar
By contrast, the Saint Model is a high bar indeed. I have never met
anyone who could approach it, in spite of the fact that in the natural
course of my life, first as dedicated seeker, and later as meditation
teacher, I have met many highly accomplished and/or revered meditators.
As for dead saints, in many cases there is little record of the
phenomenology of their day\sphinxhyphen{}to\sphinxhyphen{}day experience, either subjective or as
observed by others. In cases where there \sphinxstyleemphasis{is} such a record, candidates
are quickly eliminated from the Saint Model for displaying or reporting
unseemly amounts of human behavior.

\sphinxAtStartPar
A useful model describes a repeatable process and has clear metrics for
success. The Pragmatic Model identifies specific phenomena that are
experienced by the meditator at each stage along a typical sequence of
events. (See, for example, the Progress of Insight section of this book,
and my criteria for attainment of each of the Four Paths.) The Saint
Model, on the other hand, does not offer clear metrics for success,
either in the beginning or the middle. In the end, you will know you
have achieved it because you will never experience or express
irritation, and you will lose your enjoyment of food.


\subsection{The Hercules Effect: Why “watering down” a high standard might be a good idea}
\label{\detokenize{main-2:the-hercules-effect-why-watering-down-a-high-standard-might-be-a-good-idea}}
\sphinxAtStartPar
In Greco\sphinxhyphen{}Roman mythology, Hercules was the very embodiment of physical
fitness. He did a great deal of slaying and capturing in his illustrious
career, and even had time to hold up the world for a moment when Atlas
needed a break. Hercules was invincible and almost infinitely strong.
Compared to Hercules, the most decorated athletes of our own day are
scarcely worth mentioning. Hercules would outbox Mike Tyson with one
hand while simultaneously defeating Serena Williams at tennis and
Michael Jordan (in his prime!) at basketball. Are we watering down our
definition of physical fitness by not believing in Hercules? Or are we
simply acknowledging that Hercules is but a myth and is therefore not
relevant to us as we probe the limits of human excellence?

\sphinxAtStartPar
Similarly, we can dispense with the myth of enlightened sainthood and
concentrate on what actually happens to flesh and blood humans when they
meditate. We can define enlightenment/awakening in a way that comports
with observed reality. A four paths model that is teachable and
learnable is infinitely more interesting than one that never happens. We
stopped believing in Hercules some time ago. Perhaps it’s time to stop
believing in magical cartoon saints. This is an eminently practical
step, as letting go of our fantasies allows us to focus on meditation in
earnest. And effective meditation practice allows us to realize the
remarkable benefits of awakening for ourselves, rather than through the
intermediary of an imagined champion.


\section{Fluency with the the Ladder of Abstraction}
\label{\detokenize{main-2:fluency-with-the-the-ladder-of-abstraction}}
\sphinxAtStartPar
Neuroscience informs us that everything we experience is a
representation in the brain. We have no direct pipeline to the external
world. I see a wall across the room. It is beige, with white trim, and
littered with purple sticky notes in book outline form. But my
experience of that wall is a mental construct based on photons hitting
my eyes or pressure sensations hitting my fingers. Some mystics have
intuitively realized this and concluded that external reality is
therefore an illusion. I find this conclusion fallacious. I have every
reason to believe that 20 years from now other people will still be able
to see and touch the wall across the room, and cover it with their own
sticky notes. The external world is not an illusion. But my experience
of it is an abstraction. What this means is that even when I go as low
as possible on a ladder of abstraction, it is \sphinxstyleemphasis{still} an abstraction.
Fine. Fair enough. For our purposes, it is sufficient to identify a
continuum of abstraction from lower to higher.

\sphinxAtStartPar
Lower levels of abstraction are, by definition, more grounded in the
five physical senses. Higher levels allow the naming of things,
memories, projections of imaginary worlds, and manipulation of concepts.
Dogs, cats, birds, lizards, and snails have access to lower levels of
abstraction, but cannot go as high as we can on the ladder. They can
experience input from the five senses, and create a mental
representation of their environment. Some non\sphinxhyphen{}human animals can even
abstract to the level of assigning labels to things. But they presumably
cannot do math or spin multiple elaborate scenarios about the future.
They cannot be architects or diagnosticians. The ability to move high on
the ladder of abstraction is uniquely human (at least on this planet)
and it has served us well. We are fruitful. We multiply. And there is
the individual payoff; if you can out\sphinxhyphen{}plan your neighbor, you will
prosper. But there is a cost. There is a cost! Higher levels of
abstraction are inherently agitating. We are happy to pay the cost
because the payoff is so great. Still… the cost. Our inability to return
to low levels of abstraction makes us sick and kills us early. We are
awash in a sea of stress and anxiety. We must re\sphinxhyphen{}learn the art of
climbing back down the ladder of abstraction. We must learn to be simple
sometimes. Not all the time. Sometimes. One of the benefits of
meditation, one of the specialties held under the over\sphinxhyphen{}arching umbrella
of contemplative fitness, is the art of simplicity. To go low on the
ladder of abstraction. To breathe. To relax.

\sphinxAtStartPar
With this in mind, we can identify \sphinxstyleemphasis{fluency} as a core value and a core
competency within contemplative fitness. We can train ourselves to
access the ladder of abstraction in its entirety, from low to high, and
back down again.


\section{The three speed transmission}
\label{\detokenize{main-2:the-three-speed-transmission}}
\sphinxAtStartPar
The three speed transmission is a conceptual framework for understanding
the ways in which contemplative practices from various traditions
complement one another. It grew out of my need to make sense of the
different, often contradictory teachings and techniques I encountered
from various contemplative traditions and teachers. Think of it as a
tree to hang your knowledge on. It will help you organize your thoughts.
This kind of knowledge tree is called a schema. Here is the three speed
transmission schema in a nutshell:
\begin{enumerate}
\sphinxsetlistlabels{\arabic}{enumi}{enumii}{}{.}%
\item {} 
\sphinxAtStartPar
First gear: What?

\item {} 
\sphinxAtStartPar
Second gear: Who?

\item {} 
\sphinxAtStartPar
Third gear: Stop practicing; this is it.

\end{enumerate}

\sphinxAtStartPar
At a slightly higher level of detail, here it is again:
\begin{enumerate}
\sphinxsetlistlabels{\arabic}{enumi}{enumii}{}{.}%
\item {} 
\sphinxAtStartPar
First Gear: Bring attention to the experience of this moment.
Objectify (take as object) the simple phenomena of the six sense
doors, which are seeing, hearing, tasting, touching, smelling, and
thinking. Pure concentration practices also fall under the First Gear
heading.

\item {} 
\sphinxAtStartPar
Second Gear: Bring attention to the apparent knower of this
experience. Typical guiding questions are “Who am I?” or “To whom is
this happening?”

\item {} 
\sphinxAtStartPar
Third Gear: This is it. It’s over. Surrender to the situation as it
is in this moment. Then, go beyond even surrender, to the simple
acknowledgement that this moment is as it is, with or without your
approval. Even your effort to surrender is a presumption, a
last\sphinxhyphen{}ditch effort to control the situation; by agreeing to surrender,
you imply that you have a choice, as though you could choose \sphinxstyleemphasis{not} to
surrender. This is not so. You are not in charge. You are the kid in
the the back seat with the plastic steering wheel. This moment is
already here and nothing you can do or not do in this moment will
change it.

\end{enumerate}

\sphinxAtStartPar
Here is a third iteration of the schema with a partial list of practices
that correspond to each gear:
\begin{enumerate}
\sphinxsetlistlabels{\arabic}{enumi}{enumii}{}{.}%
\item {} 
\sphinxAtStartPar
First gear:

\end{enumerate}
\begin{enumerate}
\sphinxsetlistlabels{\arabic}{enumi}{enumii}{}{.}%
\item {} 
\sphinxAtStartPar
Vipassana meditation, with or without following the breath, noting
aloud or silently, Burmese Mahasi\sphinxhyphen{}style, interactively or alone; body
scanning vipassana, as taught in the U Ba Kin/Goenka tradition of
Burma.

\item {} 
\sphinxAtStartPar
Pure concentration practices like mantra (repeating a word); gazing
at an object; counting the breath; repeating metta (lovingkingdness)
or compassion phrases; focusing on feelings of metta or compassion;
concentrating on a conceptual object, i.e., visualization of deities,
lights, or physical objects.

\item {} 
\sphinxAtStartPar
Ecstatic dancing, whirling, or speaking in tongues.

\end{enumerate}
\begin{enumerate}
\sphinxsetlistlabels{\arabic}{enumi}{enumii}{}{.}%
\setcounter{enumi}{1}
\item {} 
\sphinxAtStartPar
Second gear:
\begin{itemize}
\item {} 
\sphinxAtStartPar
Self\sphinxhyphen{}enquiry as taught in Advaita Vedanta; hua tou as taught in
Chinese Zen (Chan) and Korean Zen (Seon).

\end{itemize}

\item {} 
\sphinxAtStartPar
Third Gear:
\begin{itemize}
\item {} 
\sphinxAtStartPar
Shikantaza (just sitting), as taught in Japanese Soto Zen; turning
toward the “un\sphinxhyphen{}manifest” as in Mahamudra or Dzogchen practices; “Just
stop!” as taught by Advaita teacher Poonja\sphinxhyphen{}ji. Being reminded by a
teacher that you are “already enlightened” or that you “cannot do it
wrong,” as taught by some neo\sphinxhyphen{}advaita teachers, e.g., Ganga\sphinxhyphen{}ji,
Mooji.

\end{itemize}

\end{enumerate}

\sphinxAtStartPar
When I first became interested in contemplative practice, I read a
number of Zen books that made reference to “awakening” or
“enlightenment.” It seemed to be some nebulous sort of wisdom that Zen
masters had. The reader was often encouraged to abandon the quest for
enlightenment, even though enlightenment was clearly the goal. If one
could just adopt the right attitude, enlightenment would arise; but if
you tried to “get” it, you would fail. Paradox was everywhere. The
aspirant must understand that there is “nowhere to go, nothing to get.”
That sort of thing. It was never clear to me how I could duplicate this
highly touted but under\sphinxhyphen{}explained wisdom in my own life. As a westerner
who did not have access to traditional Japanese culture, and who grew up
with the understanding that learning resulted from a fairly
straightforward process of education, I found the Zen approach less than
helpful.

\sphinxAtStartPar
Since I never felt called to put on a black robe and join a Zen center,
I was on my own. I didn’t know how to develop my meditation practice
other than to read books about it and sit for thirty minutes a day
counting my breath from one to ten (a practice I had learned from a Zen
book). I sensed progress in my meditation practice throughout this time,
but I had the distinct feeling that I was missing something and that my
practice was inefficient.

\sphinxAtStartPar
When I met Bill Hamilton in 1990, he told me about the Theravada
Buddhist four paths of enlightenment and the Progress of Insight map.
During this time, I also learned that according to the Pali suttas, the
dying words of the Buddha were “appamadhena sampadetha,” which means
“strive diligently.”

\sphinxAtStartPar
This linear, goal\sphinxhyphen{}oriented approach made sense to me, given my own
cultural influences, and I was immediately able to put this simple
concept to work; the more I practiced, the more I progressed. Thirty
minutes a day was not enough; I practiced more, understanding that
progress was proportional to time spent training. And technique
mattered; Mahasi\sphinxhyphen{}style mental noting, with its built\sphinxhyphen{}in feedback loop
and systematic way of including all aspects of experience, was sure to
be more efficient than simple breath\sphinxhyphen{}counting. “\sphinxstyleemphasis{Aha!}” I thought.
“\sphinxstyleemphasis{There is somewhere to go and something to get.}” It was clear that
the Pali Buddha {[}\sphinxstyleemphasis{Although both the Pali and Sanskrit texts are
ostensibly about the same historical figure, the pictures painted by
these collections of stories diverge; the Buddha of the Pali Canon is
fierce, clear in his communication, and uncompromising in his dedication
to excellence while the Buddha of the Sanskrit texts often appears
easy\sphinxhyphen{}going and vague. This is what I mean when I say “Pali Buddha” or
“Sanskrit Buddha.}{]} wasn’t into this nebulous “you can’t get there from
here” baloney at all. My practice took off like a rocket. Here was a
straightforward, systematic pedagogy, and it worked. Vipassana seemed to
make Zen irrelevant. But that wasn’t the end of the story.

\sphinxAtStartPar
In the early nineties, while living and meditating intensively in a
Burmese\sphinxhyphen{}style Mahasi monastery in Malaysia, I met an American Zen
practitioner who said that the Burmese vipassana approach was
wrong\sphinxhyphen{}headed and that the Zen people had it right after all. He insisted
that the striving that was part and parcel of the Burmese vipassana
scene was just the initial practice and that eventually you had to grow
up, stop banging your head against the wall and let things be as they
are. I spent many hours arguing with this fellow, but it was clear to me
that he had a valid point of view that wasn’t being expressed by my
Burmese teachers. I chewed on this for several years, flip\sphinxhyphen{}flopping
between thinking that the Burmese were right and the Japanese clueless,
and then deciding that the Japanese were right after all, and so on.

\sphinxAtStartPar
In the early and mid 2000s, I became fascinated with Advaita Vedanta and
the process of self\sphinxhyphen{}enquiry as taught by Ramana Maharshi and
Nisargadatta. Here was yet another lens: you didn’t have to pay
attention to anything other than the apparent self, and by asking the
question “who am I?” you could deconstruct this sticky illusion and lose
the sense of self forever, essentially solving all of your problems.
Self\sphinxhyphen{}enquiry had the benefit of simplicity; rather than the myriad
changing objects of vipassana, there was only one. It was arguably
harder to get lost while meditating, since the task was to keep the
attention firmly on the question “who am I?” From the point of view of
Advaita, neither Zen breath\sphinxhyphen{}counting, nor Zen surrender, nor Burmese
vipassana have much to offer;{[}\sphinxstyleemphasis{In all fairness to the vast and
multi\sphinxhyphen{}faceted Zen tradition, self\sphinxhyphen{}enquiry is emphasized in Korean Zen
(Seon or Son), and some schools of Chinese Zen (Chan).}{]} it’s all about
directly investigating the apparent self. All questions are immediately
redirected to self\sphinxhyphen{}enquiry. Who cares what is happening? Only ask to
whom it is happening. The recursion of this approach creates a practice
that is both elegantly simple and completely self\sphinxhyphen{}contained. I liked it,
and I jumped into the practice with both feet; my pendulum swung again
and I became dismissive of both Zen and vipassana. Ramana Maharshi
became my hero and I spat on anyone who wasn’t hip enough to practice
self\sphinxhyphen{}enquiry to the exclusion of all else.

\sphinxAtStartPar
For many years, I was unable to see how these perspectives could be
reconciled; I gravitated toward whatever felt right at the time and
declared it the best and only practice. Again and again I was blinkered
by the narrowness of my own perspective. Gradually, though, my viewpoint
began to broaden. I could no longer deny that all of these seemingly
contradictory systems had value, and more specifically that I had
benefited from all of them.

\sphinxAtStartPar
I needed a conceptual framework big and flexible enough to hold the
striving of the Pali Buddha, the self\sphinxhyphen{}enquiry of Advaita, and the
surrender of Zen. I put them in that order, i.e., 1) “What?” as in “what
is happening?” 2) “Who?” as in “to whom is it happening” and 3) “Stop
practicing because this is it.” The three speed transmission was born.
And by about 2005, I was able to see a way to integrate all three
understandings into a method, using one to scaffold the next. The three
speed transmission allows us to step back and see the bigger picture,
allowing for the possibility that any given perspective can have great
value within its sphere and that there is no one lens to rule them all.

\sphinxAtStartPar
In the end, there is no hierarchy. “Stop practicing because this is it”
is not a higher\sphinxhyphen{}level understanding than the simple reality of the six
sense doors (seeing, hearing, tasting, touching, smelling, and
thinking), as viewed through vipassana noting practice. Nor is
self\sphinxhyphen{}enquiry to be privileged over either of the other lenses. In fact,
the ability to switch fluently among lenses is a hallmark of mature
practice and mature understanding. There is no one lens to rule them
all. As Croatian Buddhist teacher Hokai Sobol says, “every perspective
both reveals and obscures.” Each lens is valid within its sphere, and
effective practice becomes a simple and practical matter of applying the
appropriate lens in any given moment.

\sphinxAtStartPar
The three speed transmission approach encourages you to adopt whatever
lens is most helpful in a particular situation. If you observe the
comings and goings of your own experience, you can see thoughts and body
sensations arising and passing away. All of these phenomena can be
perceived \sphinxstyleemphasis{out there}; I can see the computer in front of me, the man
sitting at the next table in the coffee shop; I can hear the
conversations going on around me, the background music playing through
the sound system; I can taste my coffee; I can recognize my own thoughts
and internal dialogue as I think of these examples. The practice of
looking at the objects of the six senses {[}\sphinxstyleemphasis{Buddhist theory identifies
six senses: seeing, hearing, tasting, touching, smelling, and thinking.
In this system, thinking is the sixth sense. It is valuable to see
experience this way as it counters the tendency to privilege thinking as
somehow more “me” then other phenomena. Ultimately, all phenomena,
including the momentarily arising sense of “I” share equal status; none
is to be privileged over any other.}{]} is vipassana. We can place this in
the first gear category.

\sphinxAtStartPar
In addition to all of the objects arising within experience, there is
often the sense that all of this is happening \sphinxstyleemphasis{to me}. All of this stuff
happening \sphinxstyleemphasis{out there} is being perceived by \sphinxstyleemphasis{me in here}. I’m looking
out at something, so \sphinxstyleemphasis{I} must be the one who is looking. In second gear,
we investigate this sense of subject, this sense of \sphinxstyleemphasis{I} to whom all of
this happening. Any practice that directly targets the apparent sense of
self falls into the category of second gear.

\sphinxAtStartPar
It could be argued that third gear, in its purest form, is not really a
practice at all; it is complete acknowledgement of and surrender to the
situation as it is. Such is the intention behind the “just sitting”
practice of Soto Zen as well as certain practices within the Tibetan
Buddhist traditions of Dzogchen and Mahamudra.

\sphinxAtStartPar
Understanding that the best practice is the one that is most beneficial
in this moment, we can leave behind a big bucket of unnecessary
suffering. When a practitioner laments the fact that she is not able to
sustain herself in third gear all the time, for example, a quick detour
to second gear would call into question this “self” who needs to have a
particular experience. And downshifting to first gear allows for the
invaluable feedback loop of noting (labeling) in order to stay on track
with minimal space\sphinxhyphen{}outs while also reducing experience to its bare
components, devoid of unnecessary rumination and worry.

\sphinxAtStartPar
The gearshift analogy points up the fact that it is possible to get more
traction with noting (1st gear) than with self\sphinxhyphen{}enquiry (2nd gear) or
surrender (3rd gear). Third gear practice is best done when there is
already a good deal of momentum. The automobile transmission idea
initially came from something I heard Shinzen Young say many years ago:
when things were tough, he would downshift to mindfulness of the body
(vipassana) as a kind of “first gear.” Once he got up a head of steam,
he might shift gears to another practice, perhaps Zen \sphinxstyleemphasis{shikantaza} (just
sitting). I found Shinzen’s downshifting idea extremely helpful and
later developed it into a three\sphinxhyphen{}gear model, briefing flirting with a
5\sphinxhyphen{}speed transmission along the way.

\sphinxAtStartPar
The three speed transmission schema ties in with the idea of the yogi
toolbox. There are many powerful practices for training the mind.
Ideally, we collect a toolbox full of effective techniques over a
lifetime. And the most important tool in the box is a kind of
\sphinxstyleemphasis{meta\sphinxhyphen{}tool} that allows you to select the practice most appropriate to
any given moment. This, of course, is in direct opposition to the idea
that you should choose one technique and practice it for a lifetime. I
don’t know anything about that, because I’ve always been eclectic and
experimental in my own practice. This dynamic approach has worked for
me, and this is how I teach.

\sphinxAtStartPar
Like all taxonomies, the three speed transmission is descriptive rather
than prescriptive; it is not intended to tell reality how to be, but
rather to give human beings a conceptual framework for understanding the
reality of meditation practice as it presents itself. As such, the model
is not perfect. You will be able to identify contemplative practices
that do not fit neatly into any of the three categories. In such a case,
the model has done its job; an exception to the model is made possible
by the conceptual framework provided by the model, thus continuing to
build out a scaffold upon which to hang further learning.


\section{The progress of insight map}
\label{\detokenize{main-2:the-progress-of-insight-map}}
\sphinxAtStartPar
{[}\sphinxstyleemphasis{Editor’s note: Introduce and define the Progress of Insight Map before
describing it in detail.}{]} The following is a description of how the
Progress of Insight stages might be experienced by an idealized
meditator.

\sphinxAtStartPar
If the Progress of Insight were plotted on a graph, it would start out
flat, rise until reaching a peak event, descend into a trough,
stabilize, and then resolve.

\noindent\sphinxincludegraphics{{progress-of-insight-plot}.png}
\begin{enumerate}
\sphinxsetlistlabels{\arabic}{enumi}{enumii}{}{.}%
\item {} 
\sphinxAtStartPar
The opening act is the flat line at the left, understanding that the
cycle moves from left to right. (As it is a cycle, this whole process
might be more accurately represented as a circle, but I have
deliberately chosen a linear graph for ease of understanding.) In
traditional language, what I am calling the opening act includes the
first two insight knowledges: Knowledge of Mind and Body and
Knowledge of Cause and Effect.
{[}\sphinxurl{http://www.accesstoinsight.org/lib/authors/mahasi/progress.html}{]}

\item {} 
\sphinxAtStartPar
The ascent. The third insight knowledge, Knowledge of the Three
Characteristics.

\item {} 
\sphinxAtStartPar
The peak. The fourth and fifth insight knowledges, Knowledge of the
Arising and Passing Away of Phenomena and Knowledge of Dissolution,
respectively.

\item {} 
\sphinxAtStartPar
The descent. The 6th through 10th insight knowledges: Fear, Misery,
Disgust, Desire for Deliverance, and Re\sphinxhyphen{}observation. These are
collectively referred to as the dukkha ñanas or the dark night of the
soul.

\item {} 
\sphinxAtStartPar
Consolidation and Resolution. Includes the 11th insight knowledge,
Knowledge of Equanimity, the 12th through 16th insight knowledges,
including Path and Fruition, all five of which are said to happen in
one moment, and the 17th insight knowledge,
Review.

\end{enumerate}

\sphinxAtStartPar
Even though not everyone will recognize all of the stages or experience
them as described, the general arc holds true in most cases. It’s
usually easier to recognize the stages on hindsight.


\subsection{Knowledge of Mind and Body (Stage 1)}
\label{\detokenize{main-2:knowledge-of-mind-and-body-stage-1}}
\sphinxAtStartPar
The opening stage feels solid. When our imaginary idealized meditator
first begins to sit down to meditate, her experience will probably be
fairly pleasant and unremarkable. Soon after starting, she will have her
first insight: seeing that the mind and the body are two separate
things, with each influencing the other. She sees a thought arise as
separate from “herself,” the knower of the thoughts. She may a notice a
sensation such as an itch and recognize that it is perceived in two
parts: the physical sensation itself, and the mental impression of it.

\sphinxAtStartPar
This is the beginning of a meta\sphinxhyphen{}awareness, a stepping back from
experience to be able to dispassionately observe experience, an ability
that will strengthen throughout the meditator’s life.

\sphinxAtStartPar
Our imaginary yogi has reached the first insight knowledge, the aptly
named Knowledge of Mind and Body. She is just beginning to settle into
meditation, which can be pleasant. There’s often a deep sense of calm
and subtle exhilaration upon beginning a meditation practice. Our
meditator’s experience at this point can be described as solid, because
she doesn’t yet have the perceptual resolution and acuity to see things
changing at a fine level of detail. The ability to perceive at the level
of micro\sphinxhyphen{}sensations is the very heart of the vipassana technique and
that which gives it its unique transformative power.

\sphinxAtStartPar
A traditional example can help to illustrate what is meant by solid in
this context, and how objects that initially appear solid can be broken
down into their component parts through careful observation:
\begin{quote}

\sphinxAtStartPar
Imagine that you are walking down a country road and you see what
appears to be rope lying across the road, its ends disappearing into
the brush on either side. As you draw closer, you notice that the
rope is not lying still, as one would expect from a rope. It seems
to be moving ever so slightly. Moving closer still, you realize that
it is not a rope at all, but a line of ants crossing the road in
both directions. Finally, you see that that line is composed of
individual ants, each of which is composed of many constituent parts
constantly in motion. The object of perception, which at first
seemed to be a solid rope, is revealed to be a process rather than a
thing.
\end{quote}

\sphinxAtStartPar
This practice of deconstructing apparently solid objects of perception
into their constituent parts is fundamental to the practice of
\sphinxstyleemphasis{vipassana} {[}\sphinxurl{http://en.wikipedia.org/wiki/Vipassan\%C4\%81}{]}, which is
translated into English as “seeing clearly.”

\sphinxAtStartPar
The meditator at the level of the first insight knowledge, however, has
not yet done this. True vipassana doesn’t begin until the fourth insight
knowledge, Knowledge of the Arising and Passing Away of Phenomena. It is
for this reason that the A \& P, as I call it, is the most important of
the insight knowledges leading up to stream entry. Our imaginary yogi is
not there yet, however; next in the typical sequence of events is the
second insight knowledge, Knowledge of Cause and Effect.


\subsection{Knowledge of Cause and Effect (Stage 2)}
\label{\detokenize{main-2:knowledge-of-cause-and-effect-stage-2}}
\sphinxAtStartPar
The second insight knowledge is the direct, visceral understanding of
what Buddhists call karma, as experienced in the meditator’s own life.
She will feel in her gut the pain of her past unskillful actions and the
joy of past good deeds. She may notice how recalling painful experiences
or even imaginary arguments can lead to unpleasant sensations in the
body. Likewise with pleasant memories: when she remembers the time she
sent flowers to her mother for no reason, she will feel a deep happiness
in mind and body. Our meditator is likely to be slightly less
concentrated here than she was in the first stage, more prone to mind
wandering and reflection, less able to stay focused on the objects of
meditation, whether the sensations of breathing or the choiceless
noting/noticing of various phenomena as they spontaneously arise. Like
the first insight knowledge, this second stage is not necessarily a big
deal in the meditator’s life and may go unnoticed.


\subsection{Knowledge of the Three Characteristics (Stage 3)}
\label{\detokenize{main-2:knowledge-of-the-three-characteristics-stage-3}}
\sphinxAtStartPar
The name of this insight knowledge often leads to confusion. According
to early Buddhism, the three universal characteristics of existence,
also known as the three marks, are unsatisfactoriness (\sphinxstyleemphasis{dukkha}),
impermanence (\sphinxstyleemphasis{anicca}), and not\sphinxhyphen{}self (\sphinxstyleemphasis{anatta}). Therefore, given the
name of this, the third of the insight knowledges according to the
progress of insight map, we might expect to gain deep understanding of
all three characteristics at this stage. In practice, though, this stage
is just unpleasant. The body feels tight, tense, and anxious. This is
the stage of back pain, numb legs, distraction, discomfort, fidgeting,
and boredom.

\sphinxAtStartPar
Our meditator may become obsessed with her posture at this point,
looking for just the right way to sit in order to ease her discomfort.

\sphinxAtStartPar
A common landmark of the third insight knowledge is the experience of
bright, persistent itching. Many mediators report solid, unbearable
itches that seem to last for minutes and become more unpleasant with
attention. I call the sharp, pinpoint itch the “kiss of concentration.”
If you stay with one clear itch and become interested in it, it will
carry you into concentration and eventually into the fourth insight
knowledge, Knowledge of the Arising and Passing Away of Phenomena. If
such an itch arises, become interested in it. If you are doing freestyle
noting, it’s okay to just note “itching” over and over again as you
focus on this one clear object. If you are using an anchor (primary
object) such as the breath, drop the breath entirely and place your
attention on the itch. Become the world’s greatest authority on that
itch. What does it do? Does it get stronger, clearer, brighter? Does it
fade, pulse or strobe? After it fades out, stay in that area of a few
moments and see if it returns. Go back to random noting or your anchor
only after you are certain that you have wrung every bit of useful
information out of the itch (or the pulse or the throb or pain or
whatever is the predominant object).

\sphinxAtStartPar
Eventually everything will open up into champagne bubble\sphinxhyphen{}like
sensations, unitive experiences, rising energy waves, and a general
sense of well\sphinxhyphen{}being, signaling the arrival of the fourth stage, the A\&P.
But you cannot skip over the unpleasantness of the third stage in order
to get to the fourth. Stay with the sensations as they are, whether
pleasant, unpleasant, or neutral, and let nature take its course.

\sphinxAtStartPar
The sticky places along the progress of insight are the third insight
knowledge and the tenth, respectively, the ascent to the crest of the
wave (3rd stage), and the descent into the trough that follows the crest
(10 stage). The insight knowledge is significant in that if it is not
overcome, the yogi will not progress to the all\sphinxhyphen{}important Arising and
Passing Away of Phenomena (4th stage), and will therefore not gain
access to the real fruit of contemplative practice. Having never
penetrated an object of attention, the pre\sphinxhyphen{}4th ñana yogi will remain
forever an outsider, looking in from behind the glass as others have
transformitive experiences that the pre\sphinxhyphen{}4th ñana yogi can only imagine.
Nonetheless, the 3rd ñana in itself does not present anything beyond
ordinary human suffering. The pain is mostly physical, mostly
experienced during formal meditation, and does not significantly affect
the yogi’s life off the cushion. Such pre\sphinxhyphen{}4th ñana yogis, of which there
are many, often become religious, adopting the ideas and trappings of
whatever scene they are in. They may become devoted and much\sphinxhyphen{}valued
members of their spiritual/religious community. But they have not yet
understood the real value of this practice.


\subsection{Knowledge of the Arising and Passing away of Phenomena (Stage 4)}
\label{\detokenize{main-2:knowledge-of-the-arising-and-passing-away-of-phenomena-stage-4}}
\sphinxAtStartPar
The fourth insight knowledge could be said to be the most significant
event in a meditator’s career, and is often the most spectacular. This
is the spiritual opening, often a completely life\sphinxhyphen{}changing event. This
stage often involves unitive experiences, “God\sphinxhyphen{}union,” “the white
light,” mystical visions, and sublime ecstasy. It signals the beginning
of true spirituality, and while it is often mistaken for a culminating
event and heralded as an experience of “enlightenment”, it is really the
germination of the seed that will later come to fruition in stream entry
and further developments over a lifetime.

\sphinxAtStartPar
The A\&P is not a spectacular event for everyone, however; it can be a
more subtle shift, with meditation becoming more pleasant and dynamic.
Even if our meditator does not experience a full\sphinxhyphen{}blown peak experience,
she will notice a change from the rough patch (3rd insight knowledge)
that preceded this stage. She is likely to feel a deep gurgling joy
bubbling up, rising through the body. The A\&P is a very pleasant time in
meditation, bringing with it a kind of orgasmic joy that dwarfs the
pleasantness of the beginning stages.

\sphinxAtStartPar
It is common to experience brightness in the visual field during
meditation in this stage, as if someone just turned on the lights, even
with the eyes closed. Some people feel more energetic throughout the day
and have trouble sleeping. Dreams may be more vivid or intense. A kind
of manic joy may be experienced.

\sphinxAtStartPar
Bill Hamilton used to say that the A\&P stage marks the first time the
meditator has managed to completely “penetrate the object.” To use the
metaphor from earlier, our meditator is now able to see the individual
ants that make up what she previously saw as a rope.

\sphinxAtStartPar
The meditator has managed to reduce a seemingly solid thing to its
component parts. A body sensation that was previously experienced as a
solid pain in her knee while sitting is now experienced as waves of
subtle twitching sensations. The clear, persistent itch from the third
insight knowledge breaks down into pulses and vibrations. Thoughts,
instead of sitting in the mind like stones, are seen to arise, live out
their brief existence, and then vanish cleanly into the nothingness
whence they came.

\sphinxAtStartPar
Sitting is effortless at this stage, and meditators tend to see their
daily hours of formal practice spike upward. It is not unusual for
someone in the throws of the A\&P to sit for several hours a day. For a
few days around the attainment of the fourth insight knowledge, all is
right with the universe. The secular yogi feels enlightened, the
religious yogi feels touched by God, and both expect to live out the
rest of their lives at the crest of this infinite wave.

\sphinxAtStartPar
Waves, however, are not infinite, but temporal and cyclical in nature.
Returning to our graph, we see that the fourth insight knowledge exists
at the very peak of the cycle.

\sphinxAtStartPar
Because following the peak of every wave is a trough, there is trouble
on the horizon. Mercifully, the first part of the descent is pleasant,
though that may be viewed as a knife that cuts both ways as it does not
prepare the meditator for the horror of what is to come. Next in line is
the fifth insight knowledge, Knowledge of Dissolution.


\subsection{Knowledge of Dissolution (Stage 5)}
\label{\detokenize{main-2:knowledge-of-dissolution-stage-5}}
\sphinxAtStartPar
The fifth insight knowledge, Knowledge of Dissolution, is a very
chilled\sphinxhyphen{}out stage, especially compared to the overwhelming joy and
excitement of the previous stage. If the A\&P is orgasmic joy,
dissolution is more akin to post\sphinxhyphen{}coital bliss.

\sphinxAtStartPar
Our meditator is in love with the world and everyone in it, but feels no
compulsion to do anything about it. Our meditator’s experiences in
meditation are noticeably more relaxed than they were in the previous
stage, and she can easily sit for a long periods just grooving on the
cool, diffuse, tingling sensations of the body.

\sphinxAtStartPar
The characteristic mind state of the fifth insight knowledge is bliss
and the characteristic body sensations are coolness on the skin and
tingles. The mental focus is diffuse; it’s possible to feel the skin all
over the body, all at once. This is something that is difficult to do in
any other stage, so when it happens, it’s a good indicator that you are
moving through the dissolution stage. Bill Hamilton used to say of this
stage that you feel like a donut; you can be aware of the edges of an
object, but not the middle. During dissolution, if you try to notice
fine detail within the body, or experience a single sensation clearly,
or zoom in on a small area, you will become frustrated. Although zooming
in to a point would have been easy at an earlier level of development,
at this stage, everything is dissolving and disappearing, hence the name
“dissolution.” The observing mind is only able to perceive the passing
away of objects rather than their arising. If you are able to let this
happen naturally, it will be blissful, but if you fight it, it will be
frustrating. The mind is markedly unproductive at this stage.
Conversations are difficult and it’s hard to concentrate. Attention is
diffuse, often dreamy, and there’s a sense of being out of focus. By the
time a thought is recognized, it is already gone.

\sphinxAtStartPar
This happy stupidity does not last long, however, as the dukkha ñanas
are coming hard on its heels. We are about to enter the true low point
of the cycle, territory so daunting that it has tested the resolve of
many a yogi.


\subsection{The Dukkha Ñanas}
\label{\detokenize{main-2:the-dukkha-nanas}}
\sphinxAtStartPar
{[}\sphinxstyleemphasis{Ñana (pronounced “nyana”) is a word from the Pali language of ancient
India, translated here as (insight knowledge).}{]}

\sphinxAtStartPar
The next five insight knowledges together form the most difficult part
of the Progress of Insight cycle. They are collectively called the
\sphinxstyleemphasis{dukkha ñanas}, the insight knowledges of suffering. I also refer to
them as the dark night of the soul, after the poem by 16th Century
Spanish Christian mystic Saint John of the Cross, which describes his
own spiritual crisis while practicing in a very different context. (The
fact that Saint John of the Cross, among others, has described this
mental territory in a way that is strikingly similar to Buddhist
descriptions is evidence for a developmental process the potential for
which is built\sphinxhyphen{}in to human beings, cutting across time spans,
traditions, and individuals.)

\sphinxAtStartPar
It makes sense to group the five difficult stages of the progress of
insight together as the dukkha ñanas because not every meditator is able
to distinguish the individual stages while going through them. Although
the Progress of Insight map describes a very particular sequence of
unpleasant experiences, many people just experience it as one big blob
of suffering while going through the cycle for the first time or even
after having gone through it many times. It is not necessary to
recognize each of the stages within the dukkha ñanas in order to make
progress. It is, however, important to understand that you are highly
likely to encounter difficult territory at some point. This is the value
of seeing the stages laid out as a graph; meditation does not simply
lead to a linear increase in happiness, and understanding this ahead of
time can save a great deal of confusion. Forewarned is forearmed, and
with a reasonable idea of what to expect as your own process unfolds,
you will be better prepared to deal with difficulty as it arises.


\subsubsection{Knowledge of Fear (Stage 6)}
\label{\detokenize{main-2:knowledge-of-fear-stage-6}}
\sphinxAtStartPar
The name says it all. Following the peak experience of the fourth ñana,
the Arising and Passing Away of Phenomena, our meditator’s world began
to dissolve. But this was not a problem for her, as the deep joy of the
crest of the wave was smoothly replaced by cool bliss. Delicious
tingling sensations ran down the arms and legs and thoughts disappeared
before they could become the objects of obsession. Now, with the onset
of the 6th ñana, Knowledge of Fear, all of that changes. The dissolution
of thoughts and physical sensations continues, but the meditator now
interprets it very differently; she is terrified to see her world
falling apart.

\sphinxAtStartPar
About two weeks into my first three\sphinxhyphen{}month retreat at Insight Meditation
Society in Massachusetts in 1991, having already experienced the high of
the A\&P (4th ñana) and the bliss of Knowledge of Dissolution (5th ñana),
I was passing the time before lunch by doing walking meditation on the
ancient, no\sphinxhyphen{}longer\sphinxhyphen{}used bowling alley of the former manor house when I
was overcome by a wave of abject terror. The hardwood floor of the
bowling alley no longer felt solid beneath my stockinged feet. The stark
colors of the floor and walls punished my eyes and the walls themselves
seemed to writhe and twist as I watched them. I pushed my hand against
the wall beside me, seeking something solid. The wall felt spongy. I
fell to my knees on the hardwood floor, oblivious to other yogis who may
have been passing by, and pushed my fingertips against the oak floor
boards, desperate to find something solid. My fingers seemed to sink
into the floor. Tears streamed down my face and tapped onto the wooden
floor as I found myself overcome by an unspeakable dread that I could
not understand.

\sphinxAtStartPar
This experience, which lasted about ten minutes, was my first full\sphinxhyphen{}blown
taste of the sixth insight knowledge, Knowledge of Fear. As intense as
it was, momentarily plunging me into what seemed like a bad acid trip
from a 1960s anti\sphinxhyphen{}drug propaganda film, it was mercifully brief and
passed cleanly away by early afternoon.

\sphinxAtStartPar
A traditional description of the sixth ñana describes a mother who has
just seen her husband and all but one of her sons executed. As her only
surviving son prepares to suffer the same fate, the dread that his
mother feels is akin to the dread of a yogi who attains to the sixth
ñana. Personally, I find this story a bit over the top, but it certainly
gets one’s attention. And while Knowledge of Fear can indeed be intense,
as it was for me, for some yogis it is not spectacular at all, just
unpleasant.


\subsubsection{Knowledge of Misery (Stage 7)}
\label{\detokenize{main-2:knowledge-of-misery-stage-7}}
\sphinxAtStartPar
The next insight knowledge to arise, the aptly named Knowledge of
Misery, is number seven of the 16 insight knowledges (\sphinxstyleemphasis{ñanas}). The body
writhes, the skin feels like it is crawling with bugs, and the muscles
of the neck and jaw contract unpleasantly, pulling the face into a
rictus. It is hard to sit still on the meditation cushion, as the whole
body feels unsettled. Unpleasant sensations arise quickly and pass away
before the meditator can focus on them, thus taking away one of the
strategies that has served her well until now, that of focusing on
unpleasant body sensations in order to become concentrated. The
experiences I have listed are just some of the many possible ways in
which misery can arise. Each individual will have a unique experience.
The seventh ñana will not last long, perhaps not more than a day or two,
if that.


\subsubsection{Knowledge of Disgust (Stage 8)}
\label{\detokenize{main-2:knowledge-of-disgust-stage-8}}
\sphinxAtStartPar
The ancient ñana\sphinxhyphen{}naming commission once again scores a direct hit; the
eighth insight knowledge, Knowledge of Disgust is just as it sounds.
Food is repellant, the thought of sex is nauseating, and everyone smells
bad. The nose may wrinkle up as it if noticing and unpleasant odor.
Again, this ñana is generally short\sphinxhyphen{}lived.


\subsubsection{Knowledge of Desire for Deliverance (Stage 9)}
\label{\detokenize{main-2:knowledge-of-desire-for-deliverance-stage-9}}
\sphinxAtStartPar
Do you know what it feels like when you are sobbing, completely at wit’s
end, overcome by grief and self\sphinxhyphen{}pity? The body shakes and rocks, and you
feel the release of total surrender to your emotional pain. This is one
way the ninth insight knowledge (9th \sphinxstyleemphasis{ñana}), Knowledge of Desire for
Deliverance, can manifest. One way or the other, though, Desire for
deliverance is just what the name says: you want out. Out of this
situation, out of this experience, even out of this life. There is a
pervasive sadness, and a feeling of hopelessness. Most of all, there is
aversion. But it doesn’t last long and next in line is…


\subsubsection{Knowledge of Re\sphinxhyphen{}Observation (Stage 10)}
\label{\detokenize{main-2:knowledge-of-re-observation-stage-10}}
\sphinxAtStartPar
This is where the ancient Buddhist namers of \sphinxstyleemphasis{ñanas} fell down on the
job. The innocuous\sphinxhyphen{}sounding Knowledge of Re\sphinxhyphen{}Observation, tenth of the
sixteen insight knowledges, is a wolf in sheep’s clothing. Books have
been written about it. It is the stuff legends are made of. A better
name might be Knowledge of Instability. This is the Dark Night of the
Soul, and the Agony in the Garden. Although some yogis are able to pass
through this stage relatively unscathed, it is common for a yogi’s life
to be completely disrupted by the tenth ñana. It is the phase referred
to in Zen as the “rolling up of the mat,” because the yogi has the
intuitive sense that meditation is only adding to his misery, and
abandons the sitting practice. The 10th ñana is St. John of the Cross’
Dark Night of the Soul, a realm of such gut\sphinxhyphen{}wrenching despair that the
yogi may want to abandon all worldly (and otherworldly) pursuits, pull
down the shades, roll up into a ball and die. In more modern terms, the
10th ñana can be indistinguishable from clinical depression.

\sphinxAtStartPar
Although all of the ñanas (insight knowledges) numbered six through
eight are included in the dukkha ñanas, it is the 10th that causes the
real hardship, as the Re\sphinxhyphen{}Observation stage is an iterative rehash of the
Insight Knowledges of Fear, Misery, Disgust, and Desire for Deliverance,
along with some nasty surprises all its own.

\sphinxAtStartPar
When the yogi attains to the crest of the wave in the fourth ñana, she
believes that she has arrived at her destination. From here on in, she
reasons, life should be a breeze. Even if she has been warned, she does
not believe the warnings. She is completely unprepared for what is to
come and is blindsided by the fury of the tenth ñana, which consists of
the four previous ñanas of fear, misery, disgust, and desire for
deliverance repeating themselves in a seemingly endless loop, and worse
with each iteration. In addition, the strong concentration of the fourth
ñana (the A\&P) seems to have disappeared; one cannot escape into a
pleasantly concentrated state, and there is no respite from the
unpleasantness and negativity that flood the body and mind.

\sphinxAtStartPar
Actually, the yogi’s practice is even more concentrated than before, but
she is accessing unstable strata of mind that are not conducive to
restful mind states or happy thoughts. She obsesses about her progress,
is sure that she is back\sphinxhyphen{}sliding, and devises all manner of strategies
to “get back” what she has lost. The meditation teacher tries to
reassure the meditator that she is still on track, but to no avail. The
best approach at this point is to come clean with the yogi, lay the map
on the table, and say “You are here. I know it isn’t easy, but it does
not last forever. If you continue to practice, you will see through
these unpleasant phenomena, just as you have seen through every
phenomenon that has presented itself so far. You are here because you
are a successful yogi, not because you are a failure. Let the momentum
of your practice carry you as you continue to sit and walk and apply the
vipassana technique.”

\sphinxAtStartPar
It is interesting to note that a yogi who is well\sphinxhyphen{}versed in jhana
(pleasant states made possible by high levels of concentration) may
navigate this territory more comfortably than a “dry vipassana” yogi, as
concentration is the juice that can lubricate the practice.

\sphinxAtStartPar
The pre\sphinxhyphen{}4th ñana yogi who repeatedly fails to penetrate the object and
proceed to the Arising and Passing Away of Phenomena is what Sayadaw U
Pandita calls the “chronic yogi.” This yogi can go to retreat after
retreat, over a period of years, and never understand what vipassana
practice is all about. He will, upon hitting the cushion, quickly enter
into a pleasant, hypnogogic state, maybe even discover jhana, but go
nowhere with regard to the insight knowledges. U Pandita’s frequent
exhortations to greater effort and meticulous attention to detail in
noting the objects of awareness are aimed at this “chronic yogi.”

\sphinxAtStartPar
The “dark night yogi,” on the other hand, is Bill Hamilton’s “chronic
achiever.” Having sailed through the all\sphinxhyphen{}important fourth ñana and
subsequent ñanas five through nine, he hits a wall at the tenth, and can
easily spend years there. But even the darkest night ends, and when it
does, dawn is sure to follow. The next stop on the Progress of Insight,
Knowledge of Equanimity, will make everything that came before it seem
worthwhile.


\subsection{Knowledge of Equanimity (Stage 11)}
\label{\detokenize{main-2:knowledge-of-equanimity-stage-11}}
\sphinxAtStartPar
The narrative of the ñanas continues with the 11th ñana, Knowledge of
Equanimity. The equanimity ñana is generally a very happy time for a
yogi. Having suffered through the solid physical pain of the third ñana
and having endured the dark night of the tenth ñana, the yogi wakes up
one day to find that everything is just fine. Dissolution of mind and
body continue, but it is no longer a problem. In fact, nothing is a
problem.

\sphinxAtStartPar
Compared with most of the other insight knowledge phases, the equanimity
ñana is particularly vast and complex, so it’s useful to divide it three
sections. We’ll discuss it in terms of low, mid, and high equanimity,
each with its characteristic sign posts and challenges.


\subsubsection{Low Equanimity}
\label{\detokenize{main-2:low-equanimity}}
\sphinxAtStartPar
I mentioned earlier that the third and tenth ñanas are the only places
where a yogi gets hung up. I should perhaps include the early and middle
stages of the eleventh on that list. In early equanimity, a meditator
can get stalled\sphinxhyphen{}out here for lack of motivation. When everything feels
fine, there seems little reason to meditate. Many of us are motivated to
practice by our own suffering. And since there is very little suffering
in the equanimity phase, it is tempting to stop meditating and enjoy the
passing parade. The challenge, then, in early equanimity, is simply to
keep meditating, whether you feel like it or not.

\sphinxAtStartPar
A typical pattern that I have seen repeated in dozens of meditators is
this: shortly after attaining to the 11th ñana and feeling a great deal
of relief from suffering, especially as contrasted with the difficulties
of the dark night phase, the yogi becomes complacent and stops
practicing regularly. Someone who has maintained a regular practice of
an hour or more of formal sitting per days for months suddenly finds
himself sitting sporadically, perhaps two or three times a week, and
even then for less time than usual. The predictable consequence of this
reduction in practice is to fall back into the dukkha ñanas, at which
point the yogi, once again motivated by suffering, resumes a rigorous
practice schedule, returns to low equanimity, feels relief, stops
practicing again, and falls back into the dukkha ñanas. And so on. There
is no theoretical limit to how many times this can happen. Sooner or
later, the yogi figures it out; the key is to make a firm resolution to
keep practicing systematically until stream entry, \sphinxstyleemphasis{no matter what}.


\subsubsection{Mid Equanimity}
\label{\detokenize{main-2:mid-equanimity}}
\sphinxAtStartPar
Back on a regular practice schedule, it doesn’t take long for our model
meditator to pass from low equanimity to mid equanimity. At this stage,
the challenge is slippery mind. By slippery mind, I mean an inability to
stay focused on one object, and a tendency to drift into pleasant
reverie. At first, this isn’t even noticeable to the meditator as she is
having so much fun feeling calm and free. After a while, though,
slippery mind becomes maddening; even if the meditator makes a firm
resolution to stay with her objects of meditation (in choiceless
vipassana, the objects of meditation are the changing phenomena of mind
and body as they spontaneously arise), another random thought train has
slipped in the back door almost before she has finished making the
resolution. Slippery mind is a natural consequence of a mind that is
unusually quick and nimble, together with the fact that the equanimity
ñana is still part of the dissolution process. In the first stage of
dissolution, the fifth ñana (Knowledge of Dissolution), the focus was on
the passing away of gross physical sensations, so it was experienced as
blissful. In the middle stages of dissolution, the \sphinxstyleemphasis{dukkha ñanas}
(numbers 6\sphinxhyphen{}10), the mind itself was seen to be dissolving, along with
the physical world and even one’s own sense of identity. The fear and
grief induced by the loss of the apparent self were mind\sphinxhyphen{}shattering.
Now, in the eleventh ñana, Knowlege of Equanimity, the yogi has entered
the final stages of dissolution. Even the fear and grief are seen to
disappear as soon as they arise. Things are as they are, and life is
good. But the yogi will have to relearn the art of concentration.

\sphinxAtStartPar
One way to understand what is happening here is to hearken back to the
phases of {\hyperref[\detokenize{back-jhana-nana:chicken-herding}]{\sphinxcrossref{\DUrole{std}{\DUrole{std-ref}{chicken herding}}}}}. In order to master the
equanimity ñana, the yogi has to completely develop the fifth and final
phase of chicken herding. In this phase, the chicken herder has become
one with the flock and is aware of the entire barnyard all at once. This
takes a great deal of momentum, and a great deal of practice, because
you can’t “do” this as much as you can “allow” it; the latter phases of
concentration arise naturally when the momentum is strong. And in order
to have momentum, you must practice. Frustrated by her slippery mind,
however, the yogi may try to hold the objects of meditation too tightly.
This will not work with slippery mind. Holding tightly will not allow
the later phases of concentration to develop, and will result in yet
more frustration.

\sphinxAtStartPar
This is a good place to mention wandering mind and its relationship to
concentration. It is the nature of the mind to wander, and even advanced
meditators have to deal with this phenomenon. Wandering mind cannot be
defeated through brute force, but it can be managed. I once had a
beginning meditation student tell me that she had just finished a
sitting in which she thought about her kids, her husband, the shopping,
her job, and the fact that she was never going to be good at meditation.

\sphinxAtStartPar
“Excellent,” I told her. “Just meditate in between all of that.”

\sphinxAtStartPar
There is no point in trying to will your mind to silence by brute force,
because the effort to do so will make you even more agitated. Instead,
cultivate concentration (the ability to sustain attention on an object
with minimal distraction) a little bit at a time, in the same way that
you would build a muscle by exercising it. As the concentration muscle
gets stronger, you’ll be able to sustain it for ever longer periods of
time. Since the developmental process of awakening is dynamic, it’s
unavoidable that you will have to relearn concentration skills at
various times along the way; every time your perceptual threshold
changes, you gain the ability to notice phenomena you couldn’t see
before. This is a double\sphinxhyphen{}edged sword; life is richer and more
interesting, but there is also more potential for distraction. This
potential for distraction has to be balanced by corresponding increases
in your skill at concentration, which set the stage for yet another
change in perceptual threshold, and so on. Think of it as an ongoing
process rather than a discrete goal with a fixed end point, and be
prepared to keep pushing this edge of development throughout your life.

\sphinxAtStartPar
During any meditation sitting, there are moments when the monkey\sphinxhyphen{}mind
slows down enough that it’s possible to stay with an object for a few
moments, whether the object is the breath, a kasina object, or whatever
it may be. Those few moments of concentration condition the mind in such
a way that there is a little less time before the next window of calm
appears in between the passing storms of monkey\sphinxhyphen{}mind. This momentum, or
snowball effect, where each little bit of calm conditions the next
moment of calm, is an important principle in Buddhist meditation. In
traditional teachings, the Buddhists identify “proximate causes” for
various mental factors. For example, the proximate cause for \sphinxstyleemphasis{metta}
(lovingkindness) is seeing goodness or “loveableness” in another person.
The proximate cause for \sphinxstyleemphasis{mudita} (sympathetic joy at the good fortune of
another) is seeing another’s success. And the proximate cause for
concentration is none other than… concentration! With this in mind, it
is easy to see how important the snowball effect is when you are trying
to steady the mind. And from this point of view, there is no reason to
feel frustrated even when an entire sitting goes by with just a few
brief windows of calm. Every moment of concentration makes it more
likely that the next moment of concentration will arise. Always keep in
mind that it’s all right that you haven’t mastered this yet; you can
learn, you can get better. It’s a process. Awakening itself is the
developmental process of learning to see experience as process. And
awakening, by this definition, is the crown jewel in the collection of
skills, understandings, and developments that, taken together, are
contemplative fitness.

\sphinxAtStartPar
Wandering mind, then, becomes ever more manageable with practice, and
this is good because the later phases of concentration (chicken herding
4 and 5) will not arise if the mind is not still. This does not mean
that thinking stops during deep concentration, but rather that it fades
into the background, slows down, and does not pull the mind away from
its intended target, i.e., the object or objects of meditation. When you
are firmly abiding in a jhana and thinking arises, it is felt as subtle
physical pain as it begins to pull you out of your pleasant state. With
practice, this pain becomes a familiar signal that it’s time to turn the
mind away from thoughts and toward the object of meditation… or face the
consequences. The consequences are simply that you unceremoniously exit
the jhana. The skill to exit a jhana according to the schedule you
decided upon before entering the jhana as opposed to staying too long or
being dumped out prematurely is, as we discussed earlier, the fourth
parameter for mastery of a jhana.

\sphinxAtStartPar
So, how does the yogi get to equanimity in the first place? Why do some
people get hung up for years in the preceding ñana? The key to coming to
terms with the tenth ñana, Knowledge of Re\sphinxhyphen{}observation, is surrender.
Once the yogi surrenders to whatever her practice brings, she is free.
Having surrendered, it does not matter whether the present experience
goes or stays, or whether it is pleasant or unpleasant. It is this attitude
of surrender, along with time on the cushion, that results in the full
development of the strata of mind where fear, misery, and disgust live.
Once those mental strata are developed, or (viewed through another lens)
once the \sphinxstyleemphasis{kundalini energy} is able to move freely through those
chakras, it is as if a groove has been worn through that territory. You
now own that territory and although you move up and down through those
same mental strata every day and in each meditation session, they no
longer create problems in your meditation.

\sphinxAtStartPar
When it becomes obvious that slippery mind is the only thing standing
between you and further progress, there is a specific remedy that you
can apply. The trick is to target thoughts directly. Here are some
effective ways to do this:
\begin{enumerate}
\sphinxsetlistlabels{\arabic}{enumi}{enumii}{}{.}%
\item {} 
\sphinxAtStartPar
Turn toward your thoughts as though addressing another person, and
say, “Speak! I am listening.” Try it now. Notice how thought suddenly
has nothing to say! The mind becomes silent as a church. Do this as
many times as you need to until the mind becomes still.

\item {} 
\sphinxAtStartPar
Repeat to yourself silently or aloud, “I wonder what my next thought
will be.” {[}This highly effective practice comes from Eckhart Tolle’s
\sphinxstyleemphasis{The Power of Now}.{]} Watch your mind the way a cat would watch a
mousehole, alert to the exact moment the mouse (a thought) peeks its
head out of the hole. By directly and continuously objectifying
thought in this way, thoughts will subside, leaving blissful silence
in the mind.

\item {} 
\sphinxAtStartPar
Note (label) your thoughts carefully. Put each thought into a
category. Planning thought, scheming thought, doubting thought,
self\sphinxhyphen{}congratulatory thought, imaging thought, evaluation thought,
self\sphinxhyphen{}loathing thought, reflection thought. By objectifying your
thoughts directly, you turn them into allies; the thoughts themselves
become the object of your meditation rather than a problem.

\item {} 
\sphinxAtStartPar
Count your thoughts. By counting them, you have again made thoughts
the object of your meditation. Thoughts are only a problem when they
are flying under the radar. Light them up with attention and they
cease to cause trouble.

\item {} 
\sphinxAtStartPar
Do a binary note (a noting practice that has just two choices) of
“thinking/not thinking” or “noisy/quiet.”

\end{enumerate}

\sphinxAtStartPar
Always think in terms of co\sphinxhyphen{}opting your enemies. Anything that seems to
be a hindrance in your practice should immediately be taken as the
object of your meditation. In this way, you turn the former hindrance
into an ally in your process of awakening.


\subsubsection{High Equanimity}
\label{\detokenize{main-2:high-equanimity}}
\sphinxAtStartPar
Once thoughts have been clearly objectified and are no longer flying
under the radar, high equanimity naturally emerges. At this stage,
sitting is effortless. The yogi can sit happily for hours at a time. If
pain comes, no problem. Wandering mind, no problem. Objects present
themselves to the mind one after another, obediently posing for
inspection. This is where the yogi really gets a feel for what vipassana
is all about, as she effortlessly deconstructs each thought and
sensation that appears. In high equanimity, the mind is unwilling to
reach out to any object. It doesn’t move toward pleasant objects or away
from unpleasant objects. This is what makes it possible to sit for long
periods of time; when pleasant is not favored over unpleasant, there is
no particular reason to get up.

\sphinxAtStartPar
The mind state of equanimity is inherently appealing. On a hierarchy of
desirable states, joy is higher than exhilaration, bliss is higher than
joy, and equanimity is higher than bliss. Viewed through this
understanding, it’s easy to see the natural logic in how the Progress of
Insight unfolds; notwithstanding the occasional rough patches in the 3rd
ñana and the dukkha ñanas, the progression has moved from the quiet
exhilaration of the 1st ñana through the joy of the 4th, the bliss of
the 5th, and has finally stabilized in the equanimity of the 11th. From
my point of view as a teacher and coach, it’s interesting to track the
hours of formal sitting as a yogi develops through the three phases of
equanimity. When she gets to high equanimity, the hours will usually
spike up. A meditator who has been struggling to find time in her busy
life for two hours of daily meditation may suddenly find herself sitting
three, four, or even five hours a day. Who knows where all these extra
hours come from? People will give up television, reading, time with
friends. They’ll sleep less and take less time eating than usual or
leave aside habitual tasks that don’t really need to be done. When I ask
why they are sitting so much now, students reply that there isn’t
anything else they’d rather do. They just feel like meditating. This
spike in practice hours is phase\sphinxhyphen{}specific and usually only lasts a few
days or weeks. It ends when the yogi reaches stream entry (or the \sphinxstyleemphasis{path}
moment of whichever cycle they happen to be working through at the
moment), at which time their practice hours fall back to a more
sustainable pace. Whenever I see a yogi’s practice hours spike in this
way, I feel confident that they are about to complete a \sphinxstyleemphasis{path} and I
tell them so. This particular trick of prognostication has proven
remarkably accurate and I marvel every time this process unfolds as
predicted by a 2,000 year\sphinxhyphen{}old map of human development.


\subsection{Path and Fruition}
\label{\detokenize{main-2:path-and-fruition}}
\sphinxAtStartPar
Let’s briefly review what we’ve seen so far:

\sphinxAtStartPar
Theravada Buddhism identifies Four Paths of enlightenment. The first of
these Four Paths can be subdivided into 16 “insight knowledges” or
\sphinxstyleemphasis{ñanas}. These ñanas arise one after the other, in invariable order, as
a result of vipassana meditation (objectifying, investigating, and
deconstructing the changing phenomena of mind and body). Most of the
heavy lifting is done in the first three ñanas; taken together, the
first three insight knowledges can be thought of as the pre\sphinxhyphen{}vipassana
phase. During this first phase of practice, it’s as though the yogi is
rubbing two sticks together in an effort to start a fire.{[}\sphinxstyleemphasis{Thanks to
Shinzen Young for this image of rubbing two sticks together to start a
fire, thereby releasing the potential energy contained in the wooden
sticks.}{]} When the fire takes hold in earnest, the 4th ñana, the
all\sphinxhyphen{}important Arising and Passing of Phenomena (A\&P) has been attained.
From this point on, the practice is more about constancy than heroics.
Patience and trust are important; at times it is necessary to avoid the
temptation to push too hard, understanding that just as you can’t force
a young plant to grow by pulling on its stalk, you can’t force yourself
to develop through the ñanas.

\sphinxAtStartPar
My hypothesis is that this process of development is hardwired into the
human organism. Everyone has the potential to develop along this
particular axis, and in order to do so one must simply follow the
instructions for accessing and deconstructing each new layer of mind as
it arises.


\subsubsection{Stages 12\sphinxhyphen{}16}
\label{\detokenize{main-2:stages-12-16}}
\sphinxAtStartPar
We now continue to track the progress of our idealized yogi. It’s
tempting to make a big deal out of the Path moment (the moment in which
stream entry is attained). So much emphasis is put on attaining stream
entry that we imagine there is some secret to it; surely there is some
special bit of knowledge or some extra bit of effort required to “get us
over this last hump.” In fact, it’s not like that at all. Just as all
the previous insight knowledges arose, in order, on cue, the Path moment
shows up out of nowhere when you least expect it. It’s a little bit like
chewing and swallowing; when you put food into your mouth, you begin to
chew. At some point, when sufficient chewing has taken place, you
swallow. It’s an involuntary reflex. You don’t have to obsess about
whether swallowing will occur or try to control the process. If you do,
chances are you will just get in the way. Similarly, when you meditate
according to the instructions, the various strata of mind are
automatically accessed, the apparently solid phenomena are automatically
deconstructed, the information is naturally processed, and you
automatically move from one insight knowledge to the next without having
to orchestrate the process at all. In just this way, our yogi is sitting
there one day (or walking, or standing), and there is a momentary
discontinuity in her stream of consciousness. It’s not a big deal. But,
immediately afterward, she asks herself, “Was that it?” It seems that
something has changed, but it’s very subtle. She feels lighter than
before. Maybe she begins to laugh. “Was that it? Ha! I thought it was
going to be a big deal. That was hardly anything. And yet…”

\sphinxAtStartPar
Something is somehow different. It would be very difficult to say
exactly what. In many ways, things feel exactly the same.

\sphinxAtStartPar
In Mahasi Sayadaw’s classic \sphinxstyleemphasis{The Progress of Insight}, he explains that
insight knowledges 12\sphinxhyphen{}16 happen all together, in a single instant.
Stages 12\sphinxhyphen{}15 are one\sphinxhyphen{}time events signaling stream entry, while stage 16,
fruition, can be re\sphinxhyphen{}experienced later as many times as desired. Mahasi’s
descriptions, based on the 5th Century Buddhist commentary the
\sphinxstyleemphasis{Vissudhimagga} (Path of Purification), itself based on an even earlier
text, the \sphinxstyleemphasis{Vimuddhimagga}, are interesting and well worth the
read.{[}\sphinxurl{http://www.accesstoinsight.org/lib/authors/mahasi/progress.html\#ch7.17}{]}
From the point of view of the yogi, however, it’s much simpler; she
develops through the first eleven insight knowledges and then something
changes in her practice, completing the Progress of Insight. From a
simple, subjective point of view, then, there are just twelve stages:
the first eleven, including equanimity, and the Path and Fruition event,
which somehow resets the clock and completes the cycle. Fruition is
technically the 16th insight knowledge, and we’ll preserve that
numbering system, although I will gloss over insight knowledges 12\sphinxhyphen{}15,
understanding them as theoretical structures intended to explain changes
the yogi will notice after the fact as opposed to discrete stages the
yogi experiences is real time as they occur. In fact, the yogi, by
definition, experiences nothing whatsoever during the momentary blip
that is the Path and Fruition moment.

\sphinxAtStartPar
Later, as the days and weeks go by, it becomes more and more clear that
the event was indeed First Path (stream entry). First of all, the
practice is different now. Instead of having to sit for 10 or 15 minutes
in order to work herself up to the 4th ñana, every sitting \sphinxstyleemphasis{begins} with
the 4th ñana or A\&P. From there, it takes only a short time, sometimes a
few minutes, sometimes just seconds, to get to equanimity.

\sphinxAtStartPar
Second, our yogi may suddenly find that she has access to four or more
clearly delineated jhanas, or “realms of absorption.” She may find that
she can navigate these states simply by inclining her mind toward them,
jumping between them and manipulating them at the speed of thought.

\sphinxAtStartPar
Third, there is the possibility of re\sphinxhyphen{}experiencing the 16th ñana,
Knowledge of Frution; a yogi can learn to call up fruition, which, in
classical terms, is said to be the direct apprehension of nibbana
(nirvana), at will. There are said to be three doors to nibbana, namely
the dukkha (suffering), anicca (impermanence), and anatta (no\sphinxhyphen{}self)
doors. Each of these modes of accessing cessation leads to a slightly
different experience of entering and exiting nibbana. The fascinating
exploration known as fruition practice is only available to post\sphinxhyphen{}stream
entry yogis and consists of systematically calling up, becoming familiar
with, and comparing these phenomena.

\sphinxAtStartPar
And finally, there is the 17th ñana, “knowledge of review.” It is
possible to learn to call up each of the ñanas 1\sphinxhyphen{}11 in addition to the
16th ñana of fruition and re\sphinxhyphen{}experience them as a kind of laboratory for
understanding what the insight knowledges feel like and what insights
they bring. (Ñanas 12\sphinxhyphen{}15 are one\sphinxhyphen{}time events marking the attainment of
Path and as such cannot be reviewed.) The ability to review previously
attained ñanas is especially helpful for those who plan to become
meditation teachers, but is interesting and useful for everyone because
the ñanas will continue to cycle throughout a yogi’s lifetime and it’s
very empowering to be able to identify them as they arise. This ability
to see sensations, thoughts, and mind states as process rather than
identifying with them is part of the larger process of awakening. When
we objectify (take as the object of attention) something that was
previously seen as self, we move to more and more subtle forms of
identification and ultimately come to the place where everything in
experience can be seen as process, impersonal and ever\sphinxhyphen{}changing.


\subsection{Knowledge of Review}
\label{\detokenize{main-2:knowledge-of-review}}
\sphinxAtStartPar
Question:
\begin{quote}

\sphinxAtStartPar
Kenneth, I’m curious about the phenomenon called “cycling” and how
that manifests. I relate well to the part of your commentary that
explains the initial run\sphinxhyphen{}up to stream entry. I relate well to your
explanation of a yogi’s practice and how it changes after achieving
stream entry. In MCTB
{[}\sphinxurl{https://www.amazon.com/Mastering-Core-Teachings-Buddha-Unusually/dp/1904658407}{]}
Daniel Ingram makes reference to the part concentration plays
in recognizing progress and he explains that a person with less
concentration (attention?) will be less clued in to where they are
and what’s going on. Your description below hints at the same kind
of thing: “Third, there is the \sphinxstylestrong{possibility} of re\sphinxhyphen{}experiencing
the 16th ñana, frution; a yogi can \sphinxstylestrong{learn} to call up fruition,
which is said to be the direct apprehension of nibbana (nirvana) at
will. There are three doors to nibbana, namely the dukkha
(suffering), anicca (impermanence), and anatta (no\sphinxhyphen{}self) doors. Each
of these modes of accessing cessation leads to a slightly different
experience of entering and exiting nibbana. The fascinating
exploration known as fruition practice is only available to
post\sphinxhyphen{}stream entry yogis and consists of systematically calling up,
becoming familiar with, and comparing these phenomena.” (I added the
emphasis to highlight the parts of your comments I was referring to
in the above.)

\sphinxAtStartPar
Can you elaborate on the role concentration plays at this stage? I
have not been paying very close attention to where I am according to
the four path model (or any model) and I think I’m missing some
important information due to my self\sphinxhyphen{}induced ignorance. I experience
fruition, but it occurs infrequently and on the cushion. Is it
possible to miss the experience of fruition if it happens during a
meeting, walking along, driving, what have you? Will increasing my
concentration help me recognize it?
\end{quote}

\sphinxAtStartPar
Answer: A high level of concentration is required in order to complete
the 16 ñanas and attain stream entry, but I wouldn’t say that
concentration is the deciding factor in whether a yogi recognizes and
can effectively review the territory; it would seem that attitude and
training are more important. Here is an example that might help to make
the point:

\sphinxAtStartPar
A Zen student attains stream entry. This happens in spite of the fact
that neither ñanas nor Paths are mentioned in Zen training and is not
surprising given that the ñana/Path model is just one way to describe
and map a natural, organic process of human development. Having
traversed the territory, though, the Zen student has no meta\sphinxhyphen{}perspective
that will allow him to conceive of what he has been through. In fact,
throughout the Zen training, the various phenomena that arise during
meditation are actively invalidated by the teacher; all of the pleasant
and unpleasant experiences are considered “makyo” (hallucination). A
good Zen student learns very quickly not to attempt to make sense of
meditative phenomena for fear of incurring the ire of the teacher. In
this case, both attitude (the belief that thinking about or assigning
importance to meditative experiences is dangerous) and a lack of
training in identifying and systematically accessing various states
conspire to prevent the Zen yogi from mastering this aspect of practice
even though he has shown that he has sufficient concentration to access
them.

\sphinxAtStartPar
In cases like this, a bit of remediation is in order for those who would
like to understand and master the mental territory that has become
available with the advent of stream entry. This is similar to the
situation you now find yourself in, so I’ll bring this back to specifics
and offer a prescription that is tailored to you.

\sphinxAtStartPar
You have already taken several important steps toward understanding your
experience; you have begun to educate yourself about the phenomena by
reading about the maps, you have identified fruition as a recurring
phenomenon in your own experience, and you have made a commitment to
learn more. The next step is to notice patterns in how your experience
manifests both during a sitting and over a period of hours and days.
Notice, for example, that a sitting will often follow a predictable
pattern; beginning with very little concentration, you become more and
more concentrated until you reach a climax of concentration, sometimes
culminating in a fruition or series of fruitions, after which you become
less concentrated again and have to work your way up to a concentrated
state again.

\sphinxAtStartPar
Using more technical language, a stream\sphinxhyphen{}enterer’s sitting begins with
the 4th ñana, progresses through ñanas 5\sphinxhyphen{}11, then leaps to the 16th
ñana, fruition, often a momentary event and experienced as a blip\sphinxhyphen{}out or
discontinuity of conscious awareness. After that, it resets to the 4th
ñana and repeats the pattern. You can enhance your ability to notice the
various states as they arise by keeping a journal of each sitting. Over
time you see a pattern. For example, here’s a typical sequence of events
that might unfold during a single sitting for a meditator in the review
phase after stream entry:
\begin{itemize}
\item {} 
\sphinxAtStartPar
I started the sitting with my mind a jumble (the mind is not yet
settled enough to access any Insight Knowledge).

\item {} 
\sphinxAtStartPar
Almost immediately, my mind settled down and I felt pleasant tingling
and vibrating in my leg, along with a feeling of well\sphinxhyphen{}being and
lightness (4th ñana, Arising and Passing of Phenomena).

\item {} 
\sphinxAtStartPar
Next, there were subtle, cool tingles all over my skin and I felt
bliss (5th ñana, Dissolution).

\item {} 
\sphinxAtStartPar
Next, I heard a sudden noise and was startled, frightened, and
disoriented (6th ñana, Fear).

\item {} 
\sphinxAtStartPar
Next, my jaw and neck started to tighten and writhe, and I felt
itches on my skin (7th ñana, Misery).

\item {} 
\sphinxAtStartPar
Next, I began thinking about snails and worms and ugly people, and my
face pulled involuntarily into a sneer (8th ñana, Disgust).

\item {} 
\sphinxAtStartPar
Next, my chest became tight, my breathing shallow, and I started
thinking “Let me out of here!” (9th ñana, Desire for Deliverance).

\item {} 
\sphinxAtStartPar
Next, my mind was full of all kinds of negativity, my concentration
went to hell, and I began thinking I was wasting my time and I might
as well get up and have another cup of coffee or watch some
television. I started thinking about the argument I once had with
someone, and how I had definitely been in the right (10th ñana,
Knowledge of Re\sphinxhyphen{}observation).

\item {} 
\sphinxAtStartPar
Finally, my mind settled down once again, the field of attention
expanded to include the entire environment around me, and sitting was
effortless. There was a pain in my leg, but it was no problem; I
experienced it as a flow of sensations, some pleasant, some
unpleasant, but none of it was a problem (11th ñana, Knowledge of
Equanimity).

\item {} 
\sphinxAtStartPar
I became even calmer. Then, when I wasn’t expecting anything, there
was a momentary discontinuity in my awareness, followed by a deep
breath and a feeling of bliss (16th ñana, Knowledge of Fruition).

\item {} 
\sphinxAtStartPar
After that, I sat up straight, feeling energy returning to my body
and mind and realized I was back at the beginning of the cycle (4th
ñana, Knowledge of The Arising and Passing Away of Phenomena).

\end{itemize}

\sphinxAtStartPar
Sometimes these stages go by very quickly. You may get just a momentary
taste of each ñana as you quickly move through it to the next.
Nonetheless, with repeated observations, you can see that the mind is
moving through a series of layers or strata as it becomes more
concentrated throughout the sitting. Also remember that “concentrated”
does not mean “focused on one small area or object.” Rather, it means
“remaining undistracted with the mind resting in the object or objects
of meditation.” In fact, as concentration deepens throughout the
sitting, the movement is toward an ever\sphinxhyphen{}more\sphinxhyphen{}diffuse field of awareness.

\sphinxAtStartPar
Once you have a feeling for what each state or stage entails, you can
make a resolution (Pali: \sphinxstyleemphasis{adhitthana}) to call up each state and review
it in isolation. You can call up any state in any order in this way.
This becomes your laboratory for really understanding and identifying
each of the ñanas. The formal resolution does not have to be elaborate;
it can be as simple as “May I review the 4th ñana now,” or “OK, I wanna
do some fruitions.” The more you work with resolutions, the more
confidence you have in them, until it becomes clear to you that all
these states are available to you instantaneously by simply deciding to
go there. Finally, the answer to the question “how do you get to
such\sphinxhyphen{}and\sphinxhyphen{}such a ñana or such\sphinxhyphen{}and\sphinxhyphen{}such a jhana”? becomes as simple as the
question “how do you get to the kitchen from the living room”? You just
go there. You don’t even think about it. That level of proficiency with
jhanas and ñanas is a realistic goal for anyone who has the interest and
the willingness to train systematically toward it. Taken together, this
kind of training is called \sphinxstyleemphasis{adhitthana} practice, and is usually
undertaken during the 17th ñana (Knowledge of Review), review phase but
can be done any time after First (or any other) Path.

\sphinxAtStartPar
The first time through the complete cycle usually takes months,
sometimes years, and, by definition, results in the first path of
enlightenment, using traditional Theravada Buddhist language. The first
path attainment is also referred to as stream entry, which is what we
will call it in this book. Having made it all the way through the cycle
once, you now “own” it, and can learn to review all of the territory you
have covered. Even without further practice, you will naturally cycle
through this territory. At some point, though, the mind seems to get
tired of this first package and moves on. You find yourself at the
beginning of a new progress of insight, another run through the cycle
you had completed to attain stream entry.

\sphinxAtStartPar
The cycles continue to occur whether or not the meditator pays attention
to them. They become a natural part of life, like the breath, or the
sleep cycle, or the seasons and the year. Just as one doesn’t get tired
of breathing or sleeping, but surrenders to the natural rhythms of life,
someone who has continued to practice beyond stream entry has integrated
the cycles into his or her daily rhythms. And although we may get tired
of the seasons, as we do when it is very hot in the summer or very cold
in the winter, we are deeply confident that they will soon change.


\section{The map is not the territory}
\label{\detokenize{main-2:the-map-is-not-the-territory}}\begin{quote}

\sphinxAtStartPar
“The map is not the territory.” \sphinxhyphen{}Alfred Korzybski
\end{quote}

\sphinxAtStartPar
It would be tempting the imagine that one could walk through the insight
stages exactly as they are described on the Progress of Insight map,
with each phenomenon showing up exactly on cue. It is reassuring to see
yourself as the perfect example of meditative development. As with any
map of human development, however, there are infinite individual
variations. It is one thing to accept that there is an overall sweep of
progress that moves through predictable stages, each building upon the
other, and it is another thing entirely to expect to see each
developmental landmark the same way someone else saw it.

\sphinxAtStartPar
We must also remember that the Theravada Progress of Insight map and the
Theravada practice approach and techniques reinforce one another; if you
are practicing vipassana, you are training yourself to meticulously
observe your own moment by moment experience. In traditions that
emphasize the Progress of Insight map, you may also be trained to
observe patterns over time. In such a program, you are likely to see
things in a developmental sequence and you are likely to have the
perceptual and conceptual tools to map your own experience and to be
able to compare it to a standard map of development. In other systems,
this may not be the case. In some Zen systems, for example, meditators
are taught to see all temporal phenomena as illusion.
{[}\sphinxurl{http://en.wikipedia.org/wiki/Makyo}{]} They may be encouraged to ignore
any thoughts they may have about the order in which experiences unfold.
This would be a profoundly anti\sphinxhyphen{}mapping approach. The emphasis in such a
case is on the momentary experience, to the exclusion of all else. A
meditator might become very advanced in such a system and yet have no
ability to track his or her progress through time.

\sphinxAtStartPar
Although maps are not necessary to progress, they are extremely useful
for many people. This speaks to the fundamental assumption of pragmatic
dharma: do what works. Maps work. We understand from modern methods of
education that if you can clearly articulate a goal, subdivide that goal
into smaller, attainable sub\sphinxhyphen{}goals, and provide clear feedback all along
the way, the student has a high probability of success. Using the
Progress of Insight map is consistent with what we now know about how
people learn. Zen, on the other hand, is not. It should be no surprise,
then that it is a truism in Zen that only one student in a hundred
succeeds. This is a romantic notion that may appeal to a certain macho
sense of specialness, but it is terrible pedagogy. So, we use the maps,
understanding their limitations, because to date we have nothing else
that approaches the success rate of pragmatic dharma.

\sphinxAtStartPar
Here’s what every student can reasonably expect:

\sphinxAtStartPar
In the beginning stages of a meditation practice, it is often possible
to calm the mind after a few minutes of noting or following the breath.
At this stage, students expect that the more they meditate, the calmer
they will become, and that this will continue forever. But it doesn’t
happen that way. Instead, after an initial period of success, meditation
gets harder; meditators encounter rough patches and become discouraged.
This happens to almost everyone, although details vary. For some, the
distraction is caused by itching; for others, back pain; for still
others, it is wandering mind. But the pattern of starting out optimistic
and later getting discouraged almost always holds.

\sphinxAtStartPar
For those who continue with good technique and good coaching, there is a
noticeable shift from this first difficult stage, in which meditation
becomes effortless again. For some people, this will be a spectacular
event, with lights, unitive experiences, and life\sphinxhyphen{}changing shifts in
perspective… for others, it will be much less dramatic; they’ll simply
say they are enjoying their meditation again. This is a good example of
how the overall map holds while individual expression of the
developmental phases varies. After some days or weeks in this new, easy
and pleasant phase, meditation gets hard yet again. Then it gets easy
again. And so on. Experienced teachers look for these changes, and are
ready to give encouragement or suggest tweaks in technique or attitude
to counter predictable challenges. One criticism of maps in general is
that students will learn the maps and imagine themselves into the
various states in order to convince themselves or the teacher that they
are making progress. This is a legitimate concern, but it isn’t a big
deal. In the first place, it is hard to fool an experienced teacher. For
another thing, who cares if the student is mistaken about his progress?
These things tend to work themselves out over time. When life kicks your
ass again, it’s back to the cushion.

\sphinxAtStartPar
A metaphor is useful here; we see that a human being follows a typical
developmental arc, from infant to toddler, to pre\sphinxhyphen{}adolescent,
adolescent, young adult, mature adult, to old age, and finally death.
This general sequence is a reliable predictor of the arc of a human
life, albeit with infinite variations in the way an individual will
experience these changes. With this in mind, it need not be so
surprising that we develop through meditative insight in a more or less
predictable sequence; humans have a lot in common with each other, and
often develop along predictable lines.

\sphinxAtStartPar
Finally, understand that you will spend a lifetime learning at ever
deeper levels that the map is not the territory. A map is a concept, an
embarrassingly incomplete summary made possible by the extraordinary
human powers of pattern recognition. The map will help you orient
yourself, normalize your experience, and find motivation to practice.
But your experience will be uniquely your own, rich and complex beyond
imagination, and ultimately impervious to even the most sophisticated
efforts at cartography. One of the last things Bill Hamilton said to me
while on his death bed in a Seattle hospital in 1999 was, “All maps
eventually fail.”

\sphinxAtStartPar
Use the maps wisely, accept that they will fail, and understand that
your own experience supersedes any concept. The map is not the
territory.
\begin{quote}

\sphinxAtStartPar
“When the bird and the book disagree, always believe the bird.” \sphinxhyphen{}
John James Audubon
\end{quote}


\section{Concentration, Mindfulness, and Investigation}
\label{\detokenize{main-2:concentration-mindfulness-and-investigation}}
\sphinxAtStartPar
\sphinxstyleemphasis{Concentration}, as we are using the word here, is the ability to
sustain the mind on an object with minimal distraction. Concentration is
the opposite of mind\sphinxhyphen{}wandering. The focus of the mind during
concentration can be narrow and laser\sphinxhyphen{}like, zooming in, for example, on
a single point of body sensation, or it can be wide and diffuse, where
the entire environment is the object of attention. So it is important to
understand that \sphinxstyleemphasis{non\sphinxhyphen{}distractedness} is the key to concentration. In
Buddhist theory, concentration (\sphinxstyleemphasis{samadhi}) is one of the seven factors
of enlightenment, the seven mental factors that are said to come into
balance during a moment of awakening. Concentration, then, is essential
to contemplative fitness, both for the attainment of altered states and
for the ability to see experience as process, aka awakening or
enlightenment.

\sphinxAtStartPar
To get a picture of what concentration looks like, think of a cat
sitting on the front lawn, watching a gopher hole. The cat is completely
focused on the task; it may sit alertly for ten or twenty minutes,
patiently staring at the hole in the ground, waiting for the gopher to
pop its head out. This is concentration. From this image, we can also
see why concentration alone is not enough to gain enlightenment; cats
are not enlightened, notwithstanding their prodigious concentration
skills. So concentration is but one of the skills required. This point
comes further into focus if we compare and contrast concentration and
mindfulness (\sphinxstyleemphasis{sati}). I like to define \sphinxstyleemphasis{mindfulness} as noticing that
you are noticing. While a cat has wonderful concentration, it is hard to
imagine that there is much self\sphinxhyphen{}awareness there. The cat does not notice
that it is noticing, and hence will never become enlightened; among the
inhabitants of this planet, the ability to balance mindfulness and
concentration is probably unique to humans.

\sphinxAtStartPar
It is a truism in Buddhist theory that concentration alone will never
lead to awakening. In the Mahasi tradition, monks can be positively
derisive about people who sit around for hours in highly concentrated
states but never investigate the objects of attention. These theoretical
concentration junkies are objects of mild pity. So, we must be careful
not to fall into the concentration trap. But let’s not let this pendulum
swing too far!

\sphinxAtStartPar
Concentration is, after all, one of the seven factors of enlightenment,
and without it a meditator cannot stay with any one object long enough
to investigate and deconstruct the object. Furthermore, “dry” vipassana
practice (investigation without concentration) can be painful at times.
Concentration is the juice that lubricates your practice, keeps you
interested, brings pleasant experiences, and therefore motivation to
practice. For all these reasons, it is well worthwhile to become adept
at concentration. The section on pure concentration will introduce
targeted practices to develop concentration in isolation, understanding
that eventually your facility with concentration will be integrated into
all the other practices you do, forming a whole that is greater than the
sum of the parts. For now, though, let’s continue to contrast
concentration with other important mental factors.

\sphinxAtStartPar
Imagine that I hand you a package. It’s a box, all wrapped in colored
paper and bows. A pure concentration approach would be to put your hand
on the package and leave it there, maintaining just enough contact with
the tactile sensation or visual appearance of the package to focus your
mind on it to the exclusion of everything else. Whenever your attention
wanders away from the package, bring it back. You want to become so
focused on the package that you merge with it. You don’t know what’s in
the package; you don’t care. A vipassana approach, on the other hand,
would be to take the package from my hand, shake it, prod, poke and
palpate it, give it a thorough visual examination from all angles and
distances, and finally to begin tearing off the paper and the box, layer
by layer, to discover what is inside. Vipassana requires that you
balance concentration with mindfulness and investigation. Remember that
mindfulness is noticing that you are noticing. Investigation is just as
it sounds; what is this thing? I must investigate to find out.

\sphinxAtStartPar
Vipassana leads to the deconstruction of initially solid\sphinxhyphen{}seeming objects
into their component parts. With pure concentration, the object becomes
more solid as you apply the technique, whereas with vipassana, the
object becomes less solid, and more fluid. When awakening is the goal,
the ideal is a dynamic balance of concentration, mindfulness, and
investigation. This balance allows you to maintain focus on the objects
of attention with minimal distraction while also deconstructing the
object.

\sphinxstepscope


\chapter{Book Three: Method}
\label{\detokenize{main-3:book-three-method}}\label{\detokenize{main-3::doc}}

\section{Quick Start Guide}
\label{\detokenize{main-3:quick-start-guide}}
\sphinxAtStartPar
What would I say if I had just five minutes to give comprehensive
instructions for awakening?

\sphinxAtStartPar
You are unenlightened to the extent that you are embedded in your
experience. You think that your experience is you. You must dis\sphinxhyphen{}embed.
Do this by taking each aspect of experience as object (looking at it and
recognizing it) in a systematic way. Then, surrender entirely.

\sphinxAtStartPar
Do these practices, exactly as written:

\sphinxAtStartPar
First Gear:
\begin{enumerate}
\sphinxsetlistlabels{\arabic}{enumi}{enumii}{}{.}%
\item {} 
\sphinxAtStartPar
Objectify body sensations. If you can name them, you
aren’t embedded there. Notice sensations and note to yourself:
“Pressure, tightness, tension, release, coolness, warmth, softness,
hardness, tingling, itching, burning, stinging, pulsing, throbbing,
seeing, tasting, smelling, hearing.” If I am looking at something it is
not “I”.

\item {} 
\sphinxAtStartPar
Objectify feeling\sphinxhyphen{}tone. Are sensations pleasant, unpleasant,
or neutral? Every time you note pleasant, unpleasant, or neutral, you
are dis\sphinxhyphen{}embedding from experience.

\item {} 
\sphinxAtStartPar
Objectify mind states. Call them
out as they occur. “Investigation, curiosity, happiness, anxiety,
amusement, sadness, joy, anger, frustration, annoyance, irritation,
aversion, desire, disgust, fear, worry, calm, embarrassment, shame,
self\sphinxhyphen{}pity, compassion, love, contentment, dullness, sleepiness, bliss,
exhilaration, triumph, self\sphinxhyphen{}loathing.” Name them and be free. These mind
states are not “you;” if there is a “you” it must the one who is
looking, rather than what is being looked at. Below, we will challenge
the notion that there is any “you” at all.

\item {} 
\sphinxAtStartPar
Objectify thoughts.
Categorize them: “Planning thought, anticipating thought, worrying
thought, imaging thought, remembering thought, rehearsing thought,
scenario spinning thought, fantasy thought, self\sphinxhyphen{}recrimination thought.”
Come up with your own vocabulary and see your thoughts as though they
belonged to someone else. The content of your thoughts is not relevant
except to the extent that it helps you to label and therefore objectify
them.

\end{enumerate}

\sphinxAtStartPar
Second Gear:
\begin{enumerate}
\sphinxsetlistlabels{\arabic}{enumi}{enumii}{}{.}%
\item {} 
\sphinxAtStartPar
Objectify the apparent subject. Who am “I”? Turn the
light of attention back on itself. Who knows about this experience? To
whom is this happening? Spoiler: you will not find a self to whom this
is happening. Keep looking until this becomes second nature.

\end{enumerate}

\sphinxAtStartPar
Third Gear:
\begin{quote}

\sphinxAtStartPar
1. Surrender entirely. Let it be. Good. Now go beyond even
surrender, to the simple acknowledgement that this moment is as it is,
with or without your approval. This does not mean that you must be
passive. Surrender also to activity. You are not in charge. You are the
little kid in the back seat with the plastic steering wheel. Relax and
enjoy the ride.
\end{quote}


\section{Introduction to the Method}
\label{\detokenize{main-3:introduction-to-the-method}}
\sphinxAtStartPar
The method described here is a synthesis of the lessons of my practice
and teaching over the past thirty years. My aim is to present a
systematic, reproducible method for the development of contemplative
fitness. The method works; hundreds of students have used this program
to develop high levels of contemplative fitness. But it is not the only
effective method, nor is it necessarily the best for every individual.
Think of it as similar to a gym routine from a personal fitness trainer;
you can expect it to perform as advertised if you do the work, but it is
not the only way to work out.

\sphinxAtStartPar
While we are developing contemplative excellence, let us also develop a
bit of emotional maturity; I would like to take the moralism out of
meditation. You would never think that your personal fitness trainer is
a better person than you are simply because he can bench press heavier
weights than you; he just works out more and consequently has a skill
set and a fitness level that you do not yet have. This does not make him
a saint. Contemplative development is morally neutral in the same way;
being an expert meditator or being “awakened” does not make you a better
person. Rather, a high level of contemplative fitness means that you
have pumped enough mental iron to develop a set of skills and
competencies that most people do not have. If you want to be a good
person, you must behave like one; simply meditating won’t do it for you.

\sphinxAtStartPar
There are many possible variations on contemplative fitness. This method
is one that matches my values and has consistently proven successful in
helping my students develop elite contemplative skills. It trains a
variety of skills and understandings that are valuable on their own, and
can serve as a starting point for further exploration and specialization
depending on individual interests. The course draws on techniques and
concepts I have found useful from Theravada Buddhism as well as various
traditions including Zen, Tibetan Buddhism, Advaita Vedanta, Neo\sphinxhyphen{}Advaita
and Christian mysticism. You will also find a healthy dose of my unique
contributions; when possible, I will make an effort to point out which
is which in order to avoid confusion.

\sphinxAtStartPar
Depending on his or her interests, the student may or may not decide to
follow the method all the way through. I recommend stream entry, as
described in the chapter by that name, as a wonderful goal for any
meditator. For a more casual meditator, simply reading the following
chapter on the three speed transmission and the techniques compendium in
the appendix may be enough. For the student interested in mapping the
experiences of the contemplative path and gaining facility with altered
states, the later chapters of the method will be of interest. For
someone who seeks elite levels of contemplative fitness, aka spiritual
enlightenment, I recommend that you practice the program in its
entirety. The surest way to arrive at contemplative excellence is to
build a robust practice by triangulating from many different directions.

\sphinxAtStartPar
This is a course for a lifetime of contemplative development. When
someone asks me how long it takes to reach stream entry or some other
mile\sphinxhyphen{}marker of progress, I point out that a similarly unanswerable
question would be “how long does it take to be able to do twenty
pushups?” For some people, it is trivial; they can already do twenty
pushups. For others, doing twenty pushups is a big deal, and some people
may never be able to do it in their lifetime. Similarly, with
contemplative fitness there is a great deal of individual variation in
the time it takes to make progress, depending on what you’re bringing to
the table and how much time and energy you are willing to invest. Based
on my experience working with students, we can model a bell curve for
how long it takes to get stream entry, the first goal I recommend to my
students and to the readers of this book. Most people who take on the
project are likely to get stream entry within a year or two. On the tail
ends of the curve, I know people who have been working seriously towards
stream entry for several years and haven’t yet gotten it, and I also
know people who managed it within a month or two of getting serious
about their meditation practice. Having attained stream entry, you are
likely to find that there is more to do and that you are more interested
in your meditation practice than ever. Ultimately, there is no end to
contemplative development. Like evolution, it adapts forever, always
changing and moving into new spaces, never resting or growing stale.
Plan on practicing for the rest of your life and falling more deeply in
love with your practice with each passing year.


\section{Course Objectives}
\label{\detokenize{main-3:course-objectives}}
\sphinxAtStartPar
After completing this course, you will know:
\begin{itemize}
\item {} 
\sphinxAtStartPar
The difference between pure concentration and vipassana meditation.

\item {} 
\sphinxAtStartPar
Basic theory of the Three Speed Transmission

\item {} 
\sphinxAtStartPar
Three basic skills of
\begin{enumerate}
\sphinxsetlistlabels{\arabic}{enumi}{enumii}{}{.}%
\item {} 
\sphinxAtStartPar
concentration,

\item {} 
\sphinxAtStartPar
perceptual acuity, and

\item {} 
\sphinxAtStartPar
perceptual resolution.

\end{enumerate}

\item {} 
\sphinxAtStartPar
Basic developmental theory of contemplative fitness.

\item {} 
\sphinxAtStartPar
Basic theory of 20 strata of mind.

\item {} 
\sphinxAtStartPar
Multiple meditation techniques useful in formal meditation and in daily life, including vipassana, concentration (\sphinxstyleemphasis{samatha}), self\sphinxhyphen{}inquiry, and choiceless awareness.

\end{itemize}

\sphinxAtStartPar
After completing this course, you will be able to:
\begin{itemize}
\item {} 
\sphinxAtStartPar
Deconstruct the objects of attention using the vipassana technique.

\item {} 
\sphinxAtStartPar
Recognize, navigate, and objectify a variety of mind states.

\item {} 
\sphinxAtStartPar
Access 20 strata of mind, including the Insight Knowledges from the Progress of Insight and thirteen jhanas (altered states of consciousness brought about by meditative absorption).

\item {} 
\sphinxAtStartPar
Practice meditation interactively with other people.

\item {} 
\sphinxAtStartPar
See your experience as process, at least some of the time.

\end{itemize}

\sphinxAtStartPar
Trajectory of the course
\begin{itemize}
\item {} 
\sphinxAtStartPar
Balance concentration and investigation to progress through the
Progress of Insight and attain stream entry.

\item {} 
\sphinxAtStartPar
You’ve learned to navigate and objectify a wide variety of mind
states and experiences, in formal practice and in daily life.

\item {} 
\sphinxAtStartPar
Use concentration to develop facility with jhanas.

\item {} 
\sphinxAtStartPar
You’ve learned to access a variety of blissful absorption states that
are fun, interesting, and conducive to tranquility.

\item {} 
\sphinxAtStartPar
Use 6th jhana to scaffold 2nd gear (self\sphinxhyphen{}enquiry) and dwell as the
Witness.

\item {} 
\sphinxAtStartPar
You’ve become less distractible by learning to sustain attention on
one object instead of many.

\item {} 
\sphinxAtStartPar
You’ve learned another way of moving practice into daily life.

\item {} 
\sphinxAtStartPar
You’ve learned another valuable perspective: to look at your own
experience from a dispassionate point of view.

\item {} 
\sphinxAtStartPar
See through the Witness by investigating it or letting it run its
course to scaffold 3rd gear

\item {} 
\sphinxAtStartPar
You’ve learned to see your life as process.

\item {} 
\sphinxAtStartPar
You’ve leveled the playing field and learned that there is no
ultimate state; there are many lenses or perspectives of equal
status.

\end{itemize}

\sphinxAtStartPar
Three Basic Skills
\begin{enumerate}
\sphinxsetlistlabels{\arabic}{enumi}{enumii}{}{.}%
\item {} 
\sphinxAtStartPar
Concentration.

\item {} 
\sphinxAtStartPar
Increased perceptual resolution.

\item {} 
\sphinxAtStartPar
Increased perceptual acuity.

\end{enumerate}

\sphinxAtStartPar
To understand the difference between perceptual acuity and perceptual
resolution, imagine watching a movie. Higher acuity relates to clarity
and sharpness. With high acuity, can see the images more clearly, see
the colors as rich and saturated, and see what the figures in the movie
are doing in great detail.

\sphinxAtStartPar
Perceptual resolution allows you to drill down to see pixels rather than
a solid shape (this is spatial resolution), and also allows you to see
that, in reality, a movie is a series of still frames projected in quick
succession, creating the illusion of movement (temporal resolution).

\sphinxAtStartPar
Much of the training we will do is designed to strengthen the three
basic skills of concentration, perceptual acuity, and perceptual
resolution. These three skills build the foundation for the entire
program. More complex skills arise naturally when these simple building
blocks are well developed.


\section{Unit 1: Get Stream Entry}
\label{\detokenize{main-3:unit-1-get-stream-entry}}

\subsection{Introduction to Stream Entry}
\label{\detokenize{main-3:introduction-to-stream-entry}}
\sphinxAtStartPar
We begin our training by working towards stream entry, a classical
attainment from Buddhism. There are many ways of interpreting stream
entry, and some traditions don’t discuss it at all. My interpretation is
rooted in the tradition of the late Mahasi Sayadaw, a Burmese meditation
teacher known for bringing meditation beyond the walls of the monastery
to make it available to the common people. In his book, \sphinxstyleemphasis{The Progress of
Insight}, Mahasi outlines stages that a meditator typically experiences
while practicing a specific kind of meditation. {[}\sphinxstyleemphasis{Mahasi was expanding
on the Visudimagga, a 5th century commentary by Buddaghosa. The
Visudimagga, in turn, was likely influenced by the Vimuttimagga, an
earlier text by Upatissa.}{]} In this chapter, I will present my
interpretation of this map, modified to remove unnecessary jargon and to
describe the way these stages might be experienced today.

\sphinxAtStartPar
Theravada Buddhism, the Buddhism of Southeast Asia, identifies four
“paths” or levels of enlightenment. These are seen as sequential
attainments, with each path building upon the previous. The first of the
four paths is called stream entry. Of the four stages, stream entry or
first path is the easiest to describe, especially if you use the map
from the Burmese tradition of Mahasi Sayadaw. Within this system, first
path is itself subdivided into sixteen stages. Remarkably, people today
continue to experience this predictable sequence of events and it is
possible to track a meditator’s progress as he moves through this
process; this allows a teacher to give encouragement and targeted
guidance all along the way.

\sphinxAtStartPar
The general arc of development goes like this: Meditation is easy, then
it gets hard, then it really catches fire, then it all goes to hell, and
then it stabilizes for a while. It is from this platform of stability
that stream entry (first of the four paths of enlightenment) is reached.
Even a meditator who knows nothing about the maps is likely to go
through these stages. I have spoken with people who found these maps
later in their practice and looking back were able to recognize having
been through the stages described. Because the development through these
stages is not one of linear increases in happiness, knowing about the
maps can help manage the difficult parts of the process. Since anyone
who practices meditation seriously is likely to go through these stages,
it’s helpful to know about them.

\sphinxAtStartPar
The stages encompass the whole spectrum of experience, from the physical
discomfort of sitting down to meditate for the first time, to ecstatic
joy, bliss, fear and misery, and finally, equanimity. Having been
through this roller coaster of highs and lows, a meditator gains a new
level of confidence.

\sphinxAtStartPar
This newfound confidence is one of the benefits of stream entry. Sayadaw
U Pandita once expressed it like this: “When you get to stream entry,
you will be like a \sphinxstyleemphasis{bobo doll}. There is a kind of inflatable doll that
is round on the bottom and weighted with sand so that you can punch it
and knock it over, but it will always pop back up. I do not know what
you call it in your country, but in my country we call it a \sphinxstyleemphasis{bobo
doll}.” {[}\sphinxstyleemphasis{The bobo doll analogy came while U Pandita was speaking to a
group of foreign (non\sphinxhyphen{}Burmese) yogis on long term silent retreat at his
monastery in Rangoon.}{]}

\sphinxAtStartPar
The most efficient way to attain stream entry is by systematically
investigating the experience of this moment. The most foolproof way to
systematically investigate the experience of this moment is vipassana
meditation via the noting technique. Noting is foolproof because it
provides a real\sphinxhyphen{}time feedback loop to keep you on track; you know you
are doing it right because the noting (labeling, either silently or
aloud) is not possible unless you are investigating your experience.

\sphinxAtStartPar
In the four paths model, stream entry is the first stage (path) of
enlightenment and is referred to as “classical enlightenment” by some
teachers, but people often overestimate its effects. Bill Hamilton used
to say that “after all, there are four paths, so how good could the
first one be? Stream entry is only a quarter of the way there.”
Attaining stream entry will not solve all of your problems, but by the
time you have it, you will have gained mastery of a variety of skills
and techniques that lead to more freedom over time, including the
ability to objectify and dis\sphinxhyphen{}embed from all kinds of phenomena. Most
meditation systems do not talk about measurable, achievable goals of any
kind, but I strongly encourage all of my students to work towards stream
entry. Having a systematic plan of attack is highly motivating and is
conducive to making progress when learning any skill, including
meditation. Notes:


\subsection{Introduction to Noting Meditation}
\label{\detokenize{main-3:introduction-to-noting-meditation}}\begin{quote}

\sphinxAtStartPar
“Labeling technique helps us to perceive clearly the actual
qualities of our experience, without getting immersed in the
content.” (Sayadaw U Pandita, \sphinxstyleemphasis{In This Very Life})
\end{quote}

\sphinxAtStartPar
The common thread among all meditation techniques is the activity of
bringing attention to experience. One way of doing this is to label
(note) your experience, silently or aloud, as it happens. This is the
premise of the noting technique, the powerhouse of the yogi toolbox. In
noting, we label our experience using one or two\sphinxhyphen{}word notes, at a
consistent pace. Whatever has our attention in any moment can be noted,
and I recommend noting at a pace of between one and three seconds per
note.

\sphinxAtStartPar
Here’s a quick experiment to give you a taste of noting. Ask yourself
which of the senses is predominant in this moment. Is it seeing,
hearing, tasting, touching, or smelling? Whichever it is, label it as
such. If you hear a car go by, say “hearing”. If you see these words,
say “seeing”. If you feel the weight of your body on the chair, say
“feeling”. I’m simply asking you to notice which of the senses is
predominant in your experience in this moment. Continue to note in this
way every one to three seconds: seeing, hearing, tasting, touching, or
smelling. You may soon notice another aspect of experience: thinking.
Buddhism identifies six sense doors, including the five body senses
along with thinking as the sixth sense. When thinking comes up, notice
that it’s just more sensory input; thinking happens automatically and
without your control, just like seeing or hearing.

\sphinxAtStartPar
The beauty of the noting technique is that as long as you are noting
continuously, every few seconds, you are paying attention to something
happening in the present moment. You’re meditating! If you notice that
more than a few seconds have passed since your last note, that is simply
a reminder to resume noting.

\sphinxAtStartPar
Noting has three main functions:
\begin{enumerate}
\sphinxsetlistlabels{\arabic}{enumi}{enumii}{}{.}%
\item {} 
\sphinxAtStartPar
Noting keeps you on track by giving
you a feedback loop. (If you stop noting, that in itself can be a sign
that you have wandered off track, giving you the heads\sphinxhyphen{}up that it’s time
to refocus.)

\item {} 
\sphinxAtStartPar
Noting helps ensure that you have clearly objectified
and and therefore dis\sphinxhyphen{}embedded from whatever you are experiencing.

\item {} 
\sphinxAtStartPar
Noting keeps the mind engaged to the point where there is very little
processing power left for needless suffering in the form of rumination
or worry.

\end{enumerate}

\sphinxAtStartPar
Noting harnesses the power of the feedback loop, allowing you to stay on
track throughout the meditation session; less time spent in drifting or
mind\sphinxhyphen{}wandering results in more efficient use of precious practice time,
something that is especially important for those of us who practice in
daily life. Noting can be done both on and off the cushion. Noting is
failure\sphinxhyphen{}proof; it doesn’t matter what you’re noting, as long as you are
noting. The benefits of noting are realized irrespective of whether your
meditation is pleasant or unpleasant. A session spent noting boredom,
irritation, frustration, and aversion to noting is considered as
successful as a session where everything is groovy and pleasant. The
objective is not to have nothing but pleasant experiences, but rather to
clearly objectify whatever the experience happens to be.

\sphinxAtStartPar
I first learned the noting technique from Bill Hamilton and continued to
refine it under the direction my Asian teachers from the Mahasi lineage.
The traditional Mahasi instructions for noting while following the
breath are valuable and are presented later in this chapter, but most of
my students have better luck with choiceless noting at first. I
recommend beginning with “four categories” noting and experimenting with
other styles once you feel comfortable with the basic technique.


\subsection{Noting with 5 Senses (with video)}
\label{\detokenize{main-3:noting-with-5-senses-with-video}}
\sphinxAtStartPar
The five senses are seeing, hearing, touching, tasting, and smelling.
Here is a one\sphinxhyphen{}minute video demonstrating the technique:

\begin{center}

\sphinxhref{https://www.youtube.com/watch?v=3vBmNlJ0O7I}{\sphinxincludegraphics[width=0.500\linewidth]{{3vBmNlJ0O7I}.jpg}}

   \url{https://www.youtube.com/watch?v=3vBmNlJ0O7I}
\end{center}


\subsection{Noting with 6 Senses}
\label{\detokenize{main-3:noting-with-6-senses}}
\sphinxAtStartPar
The six sense doors are seeing, hearing, touching, tasting, smelling,
and thinking. Anything we experience must necessarily fall under one of
these categories. The brain forms a holistic experience from all of the
streams of sensory experience, but it is possible to zoom in and notice
which of the senses is taking center stage in any given moment.

\sphinxAtStartPar
When noting at the level of the six sense doors, it isn’t necessary to
drill down to specifics. We just notice which of the sense doors is
predominant in this moment. Simply note “seeing,” “hearing,” “touching
(or feeling),” “tasting,” “smelling,” or “thinking,” silently or aloud
every few seconds.

\appendix
% move PDF bookmarks to the top leve
\bookmarksetup{startatroot}
% demote sections again, same as in frontmatter
\let\part\chapter
\let\chapter\section
\let\section\subsection
\let\subsection\subsubsection

\sphinxstepscope


\chapter{Glossary}
\label{\detokenize{back-glossary:glossary}}\label{\detokenize{back-glossary::doc}}\begin{itemize}
\item {} 
\sphinxAtStartPar
\sphinxstyleemphasis{concentration}: The ability to sustain attention on an object (or
objects) with minimal distraction.

\item {} 
\sphinxAtStartPar
\sphinxstyleemphasis{ñana}: (Pali) Insight knowledge.

\item {} 
\sphinxAtStartPar
\sphinxstyleemphasis{objectify}: Turn toward. Look at. See clearly. Take as object.
Identify.

\item {} 
\sphinxAtStartPar
\sphinxstyleemphasis{samadhi}: (Pali) The condition or state of concentration, i.e.,
sustaining attention on an object (or objects) with minimal
distraction.

\item {} 
\sphinxAtStartPar
\sphinxstyleemphasis{samatha}: (Pali) Activity of concentration, i.e., sustaining
attention on an object (or objects) with minimal distraction.

\item {} 
\sphinxAtStartPar
\sphinxstyleemphasis{yogi}: Person who meditates.

\end{itemize}

\sphinxstepscope


\chapter{Jhana and Ñana}
\label{\detokenize{back-jhana-nana:jhana-and-nana}}\label{\detokenize{back-jhana-nana::doc}}
\sphinxAtStartPar
(January 2009)


\section{Five phases of concentration}
\label{\detokenize{back-jhana-nana:five-phases-of-concentration}}\label{\detokenize{back-jhana-nana:chicken-herding}}
\sphinxAtStartPar
Concentration means different things to different people. I’ll first
explain what I mean by concentration, then talk about current western
Buddhist ideas and misconceptions about it, misconceptions that I
believe have contributed greatly to the glass ceiling effect.

\sphinxAtStartPar
By concentration, I mean the focusing of the mind. This can be a very
tight focus or a very diffuse focus, but in either case, the mind is
gathered together in one place or direction. One way to illustrate this
idea is to think of herding chickens. Chickens are interesting creatures
in that, although they naturally tend to move together as a flock, they
will not hesitate to scatter when they are agitated or startled. I will
describe the five phases of herding chickens. Maybe someday I’ll draw
the five chickenherding pictures, ha, ha.

\sphinxAtStartPar
Phase one: With great effort and determination, you thrust yourself into
the middle of the flock with the intention of focusing on just one
chicken. The chickens, however, are much quicker than you, and scatter
in all directions. You chase them, but the minute you get one of them
in your sights, it veers off and scurries away. You turn your attention
to the next nearest bird and continue the chase. You are not able to
follow any one bird for more than a moment. The very act of singling out
an individual chicken causes it to flee. You feel anxious and
frustrated. This tendency of a chicken to flee when pursued, however, is
built into the dynamics of chicken herding. It doesn’t mean you are
doing it wrong, it’s all part of the natural unfolding of the process.
Although you may be tempted to abandon chicken herding as futile, do not
despair. With perseverance, phase two will eventually arise.

\sphinxAtStartPar
Phase two: You are able, through continuous focus and the application of
just the right amount of effort (learned through trial and error) to
single out one bird and stick to it like glue. Your eyes do not waver
from the target. Wherever it goes, you are sure to follow. If it speeds
up, you speed up. If it slows down, you slow down. When it turns left or
right, you are right on its tail, at just the proper following distance.
A subtle exhilaration arises and you feel happy and alert.

\sphinxAtStartPar
Phase three: Chickens are, after all, flock animals, and they like
nothing better than to run together as a group. If you relax your gaze
just slightly from the chicken in front of you, you will notice that you
are now in the midst of an entire flock of chickens that are moving as
one. You are part of the flock now. You let yourself sink into this
experience, absorbing into and becoming one with the flock. There is
much less effort required here than at stage two, which, in turn,
required less than stage one. You feel a deep joy, a sense of unity
with the world.

\sphinxAtStartPar
Phase four: Your attention becomes even more diffuse and you become
aware of the edges of the flock. The bird in front of you almost
disappears. You are now noticing the entire flock, all at once. Any
effort to tighten the focus of awareness, or single out an individual
bird would pull you back to the earlier stages. You surrender to the
diffuse, almost effortless experience of the fourth stage of chicken
herding. You feel a profound bliss. Having accomplished your end, you
are now free to relish the fruits of your labor. It is good to be alive,
surrendered to the flock.

\sphinxAtStartPar
Phase five: You are fully integrated with the flock, and have become
just another chicken. You are effortlessly aware of not only the
chickens, but of the entire barnyard. Whether standing or sitting,
running uphill or down, happily grubbing for worms or painfully
tripping over chicken wire, you have no preference. Everything is fine
with you. This is the final phase of chicken herding.

\sphinxAtStartPar
The five phases of chicken herding correspond to the five phases of
concentration. My wife pointed out to me the other day that if I wanted
to talk about concentration I should carefully explain what I mean by
the word, as many people think that concentration only refers to the
very tight focus that I refer to as phase two. She is right that this
must be very explicitly taught, because if a yogi believes that only a
very tight focus qualifies as true concentration he or she will never
relax enough to let the higher phases develop.

\sphinxAtStartPar
How does a yogi know whether to practice samatha or vipassana?

\sphinxAtStartPar
There are two very different instructions, depending on whether a yogi
is pre\sphinxhyphen{} or post\sphinxhyphen{} fourth ñana. A pre\sphinxhyphen{} fourth ñana yogi, i.e. one who has
not attained to the level of the Arising and Passing Away of Phenomena,
must put his focus on penetrating the object. A post\sphinxhyphen{} fourth ñana yogi
must concentrate. It’s that simple. And the reason, in my opinion, that t
he western dharma scene has been so spectacularly unsuccessful in
producing high levels of attainment in its students is that western
dharma teachers give beginning instruction to intermediate and advanced
students; they tell post\sphinxhyphen{} fourth ñana students to ratchet up the
intensity of their vipassana, when they should be telling them to
concentrate their behinds off.

\sphinxAtStartPar
This, in my opinion tragic situation, is due to a misunderstanding that
arose out of a cultural difference. The western vipassana scene, as
exemplified by Insight Meditation Society, is influenced primarily by
Burmese Mahasi\sphinxhyphen{}style vipassana. It seems that Burmese people, by and
large, concentrate so well that it is difficult for them to learn
vipassana. This, at least, is the conventional wisdom, and my experience i
n Burma in the early and mid\sphinxhyphen{}‘90’s led me to believe that it is,
although a stereotype, generally accurate. Burmese yogis very quickly
attain a deeply concentrated state and it is all the teachers can do to
get them to look clearly at an object. Westerners, on the other hand,
have no concentration whatsoever. We watch television, drink coffee, and
obsess endlessly about our careers and our relationships. We are so
goal\sphinxhyphen{}oriented that if you so much as suggest to us that there is
something to gain by striving we will strive from here to eternity. When
Burmese monks give instructions that were designed for Burmese yogis to
American yogis, the result is too much effort and too little
concentration. Without concentration, the strata of mind that contain
advanced insight are never reached. This leads to the chronic achiever,
as Bill Hamilton put it, the yogi that has attained to the all important
fourth ñana, but is unable, year after year, to attain to the Paths.

\sphinxAtStartPar
Once a yogi, whether American, Asian, or otherwise, reaches the fourth
ñana, it is imperative that the teacher recognize this and change the
instruction from effort to concentration. A post 4th ñana yogi is in no
danger of becoming “lost in concentration.” He or she has all the tools
to deconstruct whatever object presents itself to the mind. The
important thing now is to access the relevant mental strata. These
strata are accessed through concentration. There are various techniques
to encourage the development of concentration. Two of my favorites are
counting the breath from one to ten, and kasina practice.


\section{Two techniques for developing concentration}
\label{\detokenize{back-jhana-nana:two-techniques-for-developing-concentration}}

\subsection{Counting the breath from one to ten}
\label{\detokenize{back-jhana-nana:counting-the-breath-from-one-to-ten}}
\sphinxAtStartPar
This deceptively simple but powerful practice is one of my favorites.
While walking, sitting, or reclining, count each exhalation of the
breath. When you arrive at ten, start over. The beauty of this practice
is that it has a built\sphinxhyphen{}in feedback monitor. If the mind wanders, you
will keep counting past ten, or lose count entirely. When that happens,
start over at one. I like to do this practice while walking, and often
use it as a warmup for sitting. If, for example, I plan to do kasina
practice (described below), I find it helpful to attain a concentrated
state before sitting down. This saves me the usual five to ten minutes
of fidgeting and allows me to get directly to work. How do I know I am
concentrated enough? Because I was able to count to ten two or three
consecutive times without losing count. It takes the guesswork out of
concentration.


\subsection{Kasina practice}
\label{\detokenize{back-jhana-nana:kasina-practice}}
\sphinxAtStartPar
This is the gold standard practice for attaining “hard” concentration
and jhana. A kasina is a colored disk that is used as a visual object.
It doesn’t much matter what color, but I favor pastels or earth tones. I
use a cereal bowl. For years I carried around one of those cheap plastic
bowls they use for bathing from tanks in Burma. Mine happened to be
brown, about 8 inches in diameter. You prop the bowl up against the
wall, sit a comfortable distance from it (about 5 to 8 feet) and stare
at it. That’s it. Your mind will go through the five chicken herding
stages described above. At some point you will enter jhana. You will
recognize it as an altered state of consciousness that feels very stable
and very pleasant. Note that the first four jhanas correspond to chicken
herding stages 2 through 5. Each jhana develops in the five stages, so
it is like nested Russian dolls. Jhanas 5\sphinxhyphen{}8 are a subset of jhana 4, so
there is always this nested relationship.

\sphinxAtStartPar
I have found both counting and kasina practices to be applicable to both
retreats and daily practice at home. The more I go back and forth
between deep concentration states and daily life activities, the easier
it gets to make a quick and easy transition between them. In fact, there
is a thing I sometimes do for my dharma friends that I call my “parlor
trick,” in which I sit down and cycle through all eight of the material
and immaterial jhanas in less than two minutes. It doesn’t look like
much; I just sit there and shake and roll my eyes up into my head,
holding up fingers to signal jhana numbers. (Although in the higher
jhanas, I always forget which fingers to hold up and the signal system
breaks down.) So they have to take my word for it that I attained all
those jhanas. But I began doing it as a way to show people that jhanas
aren’t something abstract, or something for other people, but rather for
ordinary people like us; they can be learned and cultivated to high
levels and called up instantly, even during daily life. Also, I must
admit, I began doing it as a way to rebel against a western Buddhist
culture that teaches that it is wicked or shameful to admit that you
“have the power of jhana.” What rubbish.


\section{Jhana, ñana, and Path}
\label{\detokenize{back-jhana-nana:jhana-nana-and-path}}
\sphinxAtStartPar
There is a relationship between jhana, ñana and Path. In 1995, I spent
two months at Sayadaw U Kundala’s monastery in Rangoon. U Kundala, a
former disciple of the late Mahasi Sayadaw, is a senior monk, much
beloved, and widely reputed to be an arahat. A few weeks into the
retreat, I began reporting to U Kundala that I was experiencing hundreds
of little flashes of cessation each day, like the winking out of
consciousness for a moment. They came singly or in waves, and I could
induce them at will. On the third day of my trying to explain this to
him through the interpreter, a woman who spoke rather limited English, U
Kundala’s eyes lit up as he said “Oh! That is Magga Phala! (Path and
Fruition, the culmination of one of the Four Paths of Enlightenment).”

\sphinxAtStartPar
“Yes,” I said. “And it’s not the first time this has happened. It also
happened a couple of years ago in Malaysia, but I had to go through the
whole Progress of Insight again.” (As an aside, this is typical of U
Kundala’s openness in speaking to students about their progress, an
attitude that spilled over into the entire community. During our
interviews, U Kundala would talk to me about Second Path. Someone would
overhear and spread the word, and soon people were coming from all over
town to stare lovingly at the western yogi who was making such progress.
People I didn’t know would stop by my room to give me gifts, hoping to
“gain merit” in so doing. One Burmese man took me home (with U Kundala’s
permission) to meet his family, and then drove me around the countryside
exploring Buddhist temples. Throughout the day, he and his cousin asked
me discreet questions about what it was like to have attained Second
Path. After my retreat, everyone treated me like royalty, and one of the
board members of the monastery volunteered to drive me to the airport.
Once at the airport, we did not wait in the queue with the hundreds of
others at the airport, but walked to the head of the line. The board
member, obviously an important man, said a word in Burmese to the
policeman at customs, who waived me through to the empty waiting room at
the gate without so much as checking my ID. As I walked toward the gate,
the man I was with shouted across the crowded airport, “You got two!
Come back for a third!” One can easily see how this sort of thing could
be a distraction, but I tell the story to illustrate how different the
attitude is in some Burmese dharma communities from that of the American
mushroom factory.)

\sphinxAtStartPar
U Kundala was very pleased with this development, and worked with me
over the next few weeks to explore the new territory. He showed me that
I could, by making a resolution, review the Fruition of either First or
Second Path, and compare them side by side. Before attaining Second
Path, however, I had had an exchange with U Kundala that completely
changed my understanding of the ñanas (insight knowledges). I reported
that I found myself able to call up any of the ñanas that I had
experienced so far on the retreat and re\sphinxhyphen{}experience them in real time.

\sphinxAtStartPar
“Yes,” he said. “Any jhanic experience can be reviewed by inclining the
mind toward it.”

\sphinxAtStartPar
Jhanic experiences? I was talking about insight knowledges. Was he
saying that ñanas are jhanas? Yes, that is exactly what he was saying.
Ñanas are jhanas, i.e. discrete concentrated states that are hardwired
into our minds. This is why all yogis have similar ideas and insights
when meditating, and they have them in an invariable sequence. There is
an underlying structure, common to all humans, that can be developed
through meditation. A yogi who has developed the first 16 of the insight
knowledges (ñanas) for the first time has attained First Path. It’s
actually quite mechanical, predictable, and not particularly mystical
when seen as a simple matter of human development.

\sphinxAtStartPar
As ñanas are jhanas, they can be lined up alongside the traditional pure
concentration jhanas in order to better understand the territory. As the
yogi develops the mind through insight and concentration, he is moving
through a series of layers, or strata, of mind. Each layer has its own
characteristics and contains within it the blueprint for a particular
insight. The first ñana, for example, corresponds to the first jhana.
That is, the stratum of mind being accessed is the same. To access that
stratum with pure concentration results in the first jhana, a highly
concentrated and pleasant absorption of mind. To access that same
stratum using the investigative technique of vipassana results in the
first insight knowledge, Knowledge of Mind and Body. Below is a list of
all 16 ñanas, along with their corresponding jhanas:
\begin{enumerate}
\sphinxsetlistlabels{\arabic}{enumi}{enumii}{}{.}%
\item {} 
\sphinxAtStartPar
ñana: Mind and Body (corresponds to 1st jhana)

\item {} 
\sphinxAtStartPar
ñana: Cause and Effect

\item {} 
\sphinxAtStartPar
ñana: Three Characteristics

\item {} 
\sphinxAtStartPar
ñana: Arising and Passing (corresponds to 2nd jhana)

\item {} 
\sphinxAtStartPar
ñana: Dissolution (corresponds to 3rd jhana)

\item {} 
\sphinxAtStartPar
ñana: Fear

\item {} 
\sphinxAtStartPar
ñana: Misery

\item {} 
\sphinxAtStartPar
ñana: Disgust

\item {} 
\sphinxAtStartPar
ñana: Desire for Deliverance

\item {} 
\sphinxAtStartPar
ñana: Re\sphinxhyphen{}observation

\item {} 
\sphinxAtStartPar
ñana: Equanimity (corresponds to 4th jhana)

\item {} 
\sphinxAtStartPar
ñana: Adaptation (one\sphinxhyphen{}time event)

\item {} 
\sphinxAtStartPar
ñana: Change of Lineage (one\sphinxhyphen{}time event)

\item {} 
\sphinxAtStartPar
ñana: Path (one\sphinxhyphen{}time event)

\item {} 
\sphinxAtStartPar
ñana: Fruition (corresponds to cessation, not considered a jhana)

\item {} 
\sphinxAtStartPar
ñana: Review

\end{enumerate}

\sphinxAtStartPar
Notice that only four of the 16 ñanas have corresponding jhanas. (The
immaterial jhanas 5\sphinxhyphen{}8 are a subset of the 4th jhana.) This is because
the other ñanas, although jhanic states, are not stable. They are
nexuses of energy where, for some reason, the energy roils around and
does not rest comfortably. Being unstable (or as in the case of ñanas
12\sphinxhyphen{}14, one\sphinxhyphen{}time events), they are not places where a yogi can rest his
mind. It is no coincidence that the pleasant ñanas have correponding
samatha jhanas, whereas the upleasant ñanas do not. Stability is
pleasant. Instability leads to fear, misery, disgust, etc. The system I
am presenting here is my own contribution to the literature. While many
agree that the jhanas and ñanas cover the same territory, the usual
practice, following U Pandita, is to lump a bunch of ñanas together
under the heading of one jhana and call it a “vipassana jhana.” I prefer
the method presented here as it is more precise, and because I believe
it better represents the actual situation.

\sphinxAtStartPar
The 15th ñana, Fruition, is stable but is not considered a jhana.
According to Theravada Buddhism, it is the direct apprehension of
Nibbana. In any case, it is very pleasant and restorative to
re\sphinxhyphen{}experience Fruition, and it is one of the benefits of attaining to
any of the Four Paths of Enlightenment. Furthermore, far from being some
esoteric practice only available to robed ascetics, it can be cultivated
to the point where only a few seconds of concentration are required to
get a taste of it. Waiting in line in the supermarket, for example, is
one of my favorite places to experience cessation/fruition.

\sphinxAtStartPar
Bill Hamilton once said that First Path is not like a pot of gold at the
end of the rainbow. It’s more like you’ve been picking up gold pieces all
along the way. First Path is just a pot to keep them in. (This applies
to subsequent Paths as well.) One way to think of it is to consider that
once you attain First Path, you “own” all of the states leading up to
it, and can learn to call them up whenever you want. Whereas before Path
even a yogi who has experienced the Arising and Passing once or many
times is subject to falling below that level once his concentration
weakens (as between retreats), the Sotaphanna, or Stream Enterer, cannot
fall below the level of fourth ñana. This then becomes the platform upon
which to begin building the scaffolding of jhanas and ñanas that lead to
Second Path, and so on. Upon the attainment of Fourth Path, or
arahatship, all of the nexes of energy have been developed, all of the
strata of mind have been accessed and penetrated, and the physioenergetic
development process is complete. From now on, the energy will
recirculate in a stable pattern, and the yogi will feel no further pull
toward this type of energetic development. He has unfettered access to
all strata of mind, and is limited only by his concentration and his
experience of navigating this territory. Needless to say, although there
is a finite number of strata, the permutations and combinations of so
many nexes of energy working in combination are effectively infinite and
no one will ever master all there is to see and feel. The arahat is far
from static. More importantly, the considerable energy that previously
went into ascending the ladder is now freed up for other pursuits, be
they mundane or sublime. Chop wood, carry water, anyone?

\sphinxAtStartPar
As a practical matter, having easy and immediate access to a variety of
jhanas is not only fun and pleasant, it also supports non\sphinxhyphen{}dual practice
and living\sphinxhyphen{}in\sphinxhyphen{}the\sphinxhyphen{}world practice, which, unlike physio\sphinxhyphen{}energetic
development, have no end.

\sphinxAtStartPar
“Full enlightenment,” then, as defined by the Theravada Buddhists, is
not a mysterious process. It is purely a matter of accessing a finite
number of strata of mind and seeing them clearly. Set ‘em up and knock
‘em down. The “seeing clearly” is automatic, or at least not difficult
for anyone who has crossed the first Arising \& Passing of Phenomena (4th
ñana). So concentration is the whole game for an intermediate or
advanced meditator. For those of a poetic or mystical bent, it could
even be a disappointment to learn that we are dealing with such a
mechanistic process. Nevertheless, such is the situation as I see it. In
any case, the subjective experience is far from dry, and there is no
need to abandon the infinitely mysterious non\sphinxhyphen{}dual practice while
developing the jhanas.



\renewcommand{\indexname}{Index}
\printindex
\end{document}